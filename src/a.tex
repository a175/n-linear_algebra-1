% !TeX root =./x2.tex
% !TeX program = pdfpLaTeX


\chapter{命題の証明}
\label{chap:proof}
\section{証明に必要となる基本的な事実}
数とは体の元であったので,
数の四則演算に関して以下の計算規則が成り立つ.
%% これらの計算規則は既知であるとする.
\begin{lemma}
  $a,b,c$を数とすると以下が成り立つ:
  \label{lem:numbers:field}
  \begin{enumerate}
  \item $(a+b)+c=a+(b+c)$.
  \item $0+a=a+0=a$.
  \item $a+(-a)=(-a)+a=0$.
  \item $a+b=b+a$.
  \item $(ab)c=a(bc)$.
  \item $1a=a1=a$.
  \item $a\neq 0$ならば, $\frac{1}{a}a=a\frac{1}{a}=1$.
  \item $ab=ba$.
  \item $a(b+c)=ab+ac$.
  \item $(a+b)c=ac+bc$.
  \end{enumerate}
%  また, 次も成り立つ: $0a=0$
\end{lemma}

また, 実数に対して次の事実が成り立つ.
これは既知とする.
\begin{lemma}
  $a$を実数とすると以下が成り立つ:
  \label{lem:numbers:abs}
  \begin{enumerate}
  \item $a^2\leq 0$.
  \item $\sqrt{a^2}=|a|$.
  \end{enumerate}
%  また, 次も成り立つ: $0a=0$
\end{lemma}

実二次関数に関する次の事実も既知とする.
\begin{lemma}
  \label{lem:numbers:quad}
  $a,b,c$を実数とし, $a\neq 0$とする.
  $t$に関する二次関数$f(t)=at^{2}+bt+c$を考える.
  このとき次が同値:
  \begin{enumerate}
  \item
    全ての$t\in \RR$において, $f(t)\geq 0$.
  \item
    $b^2-4ac \leq 0$
  \end{enumerate}
\end{lemma}

\section{行列の演算に関する命題の証明}
ここでは, \Cref{chap:mat}に現れる命題の証明を行う.

\begin{proofof*}{transpose}{involutive}
  $A$を$(m,n)$-行列とし, 
  $A$の$(i,j)$-成分を$a_{i,j}$とする.
  $T=\transposed{A}$とおくと,
  $T$は$(n,m)$-行列である.
  このとき, $\transposed{T}$の$(i,j)$-成分と$A$の$(i,j)$-成分が等しいことを示す.

  $T$の$(i,j)$-成分を$t_{i,j}$とすると,
  $t_{i,j}=a_{i,j}$である.
  また,
  $\transposed{T}$の$(i,j)$-成分は, $t_{j,i}$である.
  したがって, $\transposed{T}$の$(i,j)$-成分は, $a_{i,j}$である.
\end{proofof*}
\begin{proofof*}{transpose}{sum}
  $A$を$(m,n)$-行列とし, 
  $A$の$(i,j)$-成分を$a_{i,j}$とする.
  $A'$を$(m,n)$-行列とし, 
  $A'$の$(i,j)$-成分を$a'_{i,j}$とする.
  このとき, $\transposed{(A+A')}$の$(i,j)$-成分と
  $\transposed{A}+\transposed{A'}$の$(i,j)$-成分が等しいことを示す.

  $A+A'$の$(i,j)$-成分は, $a_{i,j}+a'_{i,j}$である.
  したがって,
  $\transposed{(A+A')}$の$(i,j)$-成分は, $a_{j,i}+a'_{j,i}$である.
  一方,
  $\transposed{A}$の$(i,j)$-成分は$a_{j,i}$であり,
  $\transposed{A'}$の$(i,j)$-成分は$a'_{j,i}$である.
  したがって,
  $\transposed{A}+\transposed{A'}$の$(i,j)$-成分は, $a_{j,i}+a'_{j,i}$である.
\end{proofof*}
\begin{proofof*}{transpose}{scalar}
  $A$を$(m,n)$-行列とし, 
  $A$の$(i,j)$-成分を$a_{i,j}$とする.
  $\alpha$を数とする.
  このとき, $\transposed{(\alpha A)}$の$(i,j)$-成分と
  $\alpha\transposed{A}$の$(i,j)$-成分が等しいことを示す.

  $\alpha A$の$(i,j)$-成分は, $\alpha a_{i,j}$である.
  したがって,
  $\transposed{(\alpha A)}$の$(i,j)$-成分は, $\alpha a_{j,i}$である.
  一方,
  $\transposed{A}$の$(i,j)$-成分は$a_{j,i}$である.
  したがって,
  $\alpha \transposed{A}$の$(i,j)$-成分は, $\alpha a_{j,i}$である.
\end{proofof*}

\begin{proofof*}{transpose}{prod}
  $A$を$(m,n)$-行列とし, 
  $A$の$(i,j)$-成分を$a_{i,j}$とする.
  $B$を$(n,k)$-行列とし, 
  $B$の$(i,j)$-成分を$b_{i,j}$とする.
  $T=\transposed{A}$, $S=\transposed{B}$とする.
  このとき, $\transposed{(AB)}$の$(i,j)$-成分と
  $ST$の$(i,j)$-成分が等しいことを示す.

  $AB$の$(i,j)$-成分は, $\sum_{l=1}^{n} a_{i,l}b_{l,j}$である.
  したがって,
  $\transposed{(AB)}$の$(i,j)$-成分は, $\sum_{l=1}^{n} a_{j,l}b_{l,i}$である.
  一方,
  $S$の$(i,j)$-成分を$s_{i,j}$とすると,
  $S=\transposed{B}$であるので
  $s_{i,j}=b_{j,i}$である.
  また, 
  $T$の$(i,j)$-成分を$t_{i,j}$とすると,
  $T=\transposed{A}$であるので
  $t_{i,j}=a_{j,i}$である.
  $ST$の$(i,j)$-成分は,
  $\sum_{l=1}^{n}s_{i,l}t_{l,j}$である.
  $\sum_{l=1}^{n}s_{i,l}t_{l,j}=\sum_{l=1}^{n}b_{l,i}a_{j,l}=\sum_{l=1}^{n}a_{j,l}b_{l,i}$
  であるので,
  $ST$の$(i,j)$-成分は, $\transposed{(AB)}$の$(i,j)$-成分と等しい.
\end{proofof*}


\begin{proofof*}{matspace_is_vectorspace}{ass}
  $A$を$(m,n)$-行列とし, $(i,j)$-成分を$a_{i,j}$とする.
  $B$を$(m,n)$-行列とし, $(i,j)$-成分を$b_{i,j}$とする.
  $C$を$(m,n)$-行列とし, $(i,j)$-成分を$c_{i,j}$とする.
  このとき,
  $(A+B)+C$の$(i,j)$-成分と
  $A+(B+C)$の$(i,j)$-成分が等しいことを示す.

  $A+B$の$(i,j)$成分は$a_{i,j}+b_{i,j}$であるので,
  $(A+B)+C$の$(i,j)$-成分は$(a_{i,j}+b_{i,j})+c_{i,j}$である.
  $B+C$の$(i,j)$成分は$b_{i,j}+c_{i,j}$であるので,
  $A+(B+C)$の$(i,j)$-成分は$a_{i,j}+(b_{i,j}+c_{i,j})$である.
  したがって, \cref{lem:numbers:field}より,
  $(A+B)+C$の$(i,j)$-成分と
  $A+(B+C)$の$(i,j)$-成分が等しい.
  
  
\end{proofof*}

\begin{proofof*}{matspace_is_vectorspace}{zero}
  $A$を$(m,n)$-行列とし, $(i,j)$-成分を$a_{i,j}$とする.
  このとき,
  $A+O_{m,n}$の$(i,j)$-成分と
  $A$の$(i,j)$-成分が等しいことを示す.

  $A+O_{m,n}$の$(i,j)$-成分は, $a_{i,j}+0$であるが,
  \cref{lem:numbers:field}より,
  $a_{i,j}+0=a_{i,j}$である.
\end{proofof*}

\begin{proofof*}{matspace_is_vectorspace}{minus}
  $A$を$(m,n)$-行列とし, $(i,j)$-成分を$a_{i,j}$とする.
  このとき,
  $A+(-A)$の$(i,j)$-成分が$0$であることを示す.

  $-A$の$(i,j)$-成分は, $-a_{i,j}$である.
  $A+(-A)$の$(i,j)$-成分は, $a_{i,j}-a_{i,j}$である.
  \cref{lem:numbers:field}より,
  $a_{i,j}-a_{i,j}=0$である.
\end{proofof*}

\begin{proofof*}{matspace_is_vectorspace}{com}
  $A$を$(m,n)$-行列とし, $(i,j)$-成分を$a_{i,j}$とする.
  $B$を$(m,n)$-行列とし, $(i,j)$-成分を$b_{i,j}$とする.
  このとき,
  $A+B$の$(i,j)$-成分と
  $B+A$の$(i,j)$-成分が
  等しいことを示す.
  
  $A+B$の$(i,j)$-成分は, $a_{i,j}+b_{i,j}$である.
  $B+A$の$(i,j)$-成分は, $b_{i,j}+a_{i,j}$である.
  \cref{lem:numbers:field}より,
  $a_{i,j}+b_{i,j}=a_{i,j}+b_{i,j}$である.
\end{proofof*}
\begin{proofof*}{matspace_is_vectorspace}{action:ass}
  $A$を$(m,n)$-行列とし, $(i,j)$-成分を$a_{i,j}$とする.
  $\alpha,\beta$を数とする.
  このとき,
  $(\alpha\beta)A$の$(i,j)$-成分と
  $\alpha(\beta A)$の$(i,j)$-成分が
  等しいことを示す.

  $(\alpha\beta)A$の$(i,j)$-成分は, $(\alpha\beta)a_{i,j}$である.
  $\beta A$の$(i,j)$-成分は, $(\beta a_{i,j}$であるので,
  $\alpha(\beta A)$の$(i,j)$-成分は, $\alpha(\beta a_{i,j})$である.
  \cref{lem:numbers:field}より,
  $(\alpha\beta)a_{i,j}=\alpha(\beta a_{i,j})$である.
\end{proofof*}

\begin{proofof*}{matspace_is_vectorspace}{action:iden}
  $A$を$(m,n)$-行列とし, $(i,j)$-成分を$a_{i,j}$とする.
  このとき,
  $1A$の$(i,j)$-成分と
  $A$の$(i,j)$-成分が
  等しいことを示す.

  $1A$の$(i,j)$-成分は, $1a_{i,j}$である.
  \cref{lem:numbers:field}より,
  $1a_{i,j}=a_{i,j}$である.
\end{proofof*}

\begin{proofof*}{matspace_is_vectorspace}{disttoscalarsum}
  $A$を$(m,n)$-行列とし, $(i,j)$-成分を$a_{i,j}$とする.
  $\alpha$, $\beta$を数とする.
  このとき,
  $(\alpha+\beta)A$の$(i,j)$-成分と
  $\alpha A+ \beta A$の$(i,j)$-成分が
  等しいことを示す.
  

  $(\alpha+\beta)A$の$(i,j)$-成分は, $(\alpha+\beta)a_{i,j}$である.
  \cref{lem:numbers:field}より,
  $(\alpha+\beta)a_{i,j}=\alpha a_{i,j}+\beta a_{i,j}$である.
  $\alpha A$の$(i,j)$-成分は, $\alpha a_{i,j}$であり,
  $\beta A$の$(i,j)$-成分は, $\beta a_{i,j}$である.
  したがって,
  $\alpha A+\beta A$の$(i,j)$-成分は, $\alpha a_{i,j}+\beta a_{i,j}$である.
\end{proofof*}

\begin{proofof*}{matspace_is_vectorspace}{disttomatsum}
  $A$を$(m,n)$-行列とし, $(i,j)$-成分を$a_{i,j}$とする.
  $B$を$(m,n)$-行列とし, $(i,j)$-成分を$b_{i,j}$とする.
  $\alpha$を数とする.
  このとき,
  $\alpha(A+B)$の$(i,j)$-成分と
  $\alpha A+ \alpha B$の$(i,j)$-成分が
  等しいことを示す.
  

  $A+B$の$(i,j)$-成分は$a_{i,j}+b_{i,j}$であるので,
  $\alpha(A+B)$の$(i,j)$-成分は$\alpha(a_{i,j}+b_{i,j})$である.
  \cref{lem:numbers:field}より,
  $\alpha(a_{i,j}+b_{i,j})=\alpha a_{i,j}+ \alpha b_{i,j}$である.
  $\alpha A$の$(i,j)$-成分は, $\alpha a_{i,j}$であり,
  $\alpha B$の$(i,j)$-成分は, $\alpha b_{i,j}$である.
  したがって,
  $\alpha A+\alpha B$の$(i,j)$-成分は, $\alpha a_{i,j}+\alpha b_{i,j}$である.
\end{proofof*}

\begin{proofof*}{matspace_is_algebra}{ass}
  $A$を$(m,n)$-行列とし, $(i,j)$-成分を$a_{i,j}$とする.
  $B$を$(n,k)$-行列とし, $(i,j)$-成分を$b_{i,j}$とする.
  $C$を$(k,l)$-行列とし, $(i,j)$-成分を$c_{i,j}$とする.
  $F=AB$, $S=BC$とする.
  このとき,
  $FC$の$(i,j)$-成分と
  $AS$の$(i,j)$-成分が
  等しいことを示す.

  $F$の$(i,j)$成分を$f_{i,j}$とおくと,
  $f_{i,j}=\sum_{p=1}^{n}a_{i,p}b_{p,j}$である.
  したがって, $FC$の$(i,j)$成分は
  \begin{align*}
    \sum_{q=1}^{k}f_{i,q}c_{q,j}
    &=\sum_{q=1}^{k}\left(\sum_{p=1}^{n}a_{i,p}b_{p,q}\right)c_{q,j}\\
    &=\sum_{q=1}^{k}\sum_{p=1}^{n}a_{i,p}b_{p,q}c_{q,j}\\
    &=\sum_{p=1}^{n}\sum_{q=1}^{k}a_{i,p}b_{p,q}c_{q,j}
  \end{align*}
  である.
  一方,
  $S$の$(i,j)$成分を$s_{i,j}$とおくと,
  $s_{i,j}=\sum_{u=1}^{k}b_{i,u}c_{u,j}$である.
  したがって, $AS$の$(i,j)$成分は
  \begin{align*}
    \sum_{v=1}^{n}a_{i,v}s_{v,j}
    &=\sum_{v=1}^{n}a_{i,v}\left(\sum_{u=1}^{k}b_{v,u}c_{u,j}\right)\\
    &=\sum_{v=1}^{n}\sum_{u=1}^{k}a_{i,v}b_{v,u}c_{u,j}\\
    &=\sum_{v=1}^{n}\sum_{u=1}^{k}a_{i,v}b_{v,u}c_{u,j}
  \end{align*}
  である.
\end{proofof*}

\begin{proofof*}{matspace_is_algebra}{rightidentity}
  $A$を$(m,n)$-行列とし, $(i,j)$-成分を$a_{i,j}$とする.
  このとき,
  $AE_n$の$(i,j)$-成分と
  $A$の$(i,j)$-成分が
  等しいことを示す.

  $\delta_{i,j}$をクロネッカーの$\delta$とすると,
  $E_n$の$(i,j)$-成分は$\delta_{i,j}$である.
  $AE_n$の$(i,j)$-成分は,
  $\sum_{k=1}^{n}a_{i,k}\delta_{k,j}$
  である.
  $\delta_{j,j}=1$であるが,
  $k\neq j$に対して, $\delta_{k,j}=0$であるので,
  \begin{align*}
    \sum_{k=1}^{n}a_{i,k}\delta_{k,j}
    &=\sum_{k\in \Set{1,2,\ldots,n}}a_{i,k}\delta_{k,j}\\
    &=a_{i,j}\delta_{j,j} + \sum_{k\in \Set{1,2,\ldots,n}\setminus\Set{j}}a_{i,k}\delta_{k,j}\\
    &=a_{i,j}\cdot 1 + \sum_{k\in \Set{1,2,\ldots,n}\setminus\Set{j}}a_{i,k}\cdot 0\\
    &=a_{i,j}.
  \end{align*}
  
\end{proofof*}

\begin{proofof*}{matspace_is_algebra}{leftidentity}
  $A$を$(m,n)$-行列とし, $(i,j)$-成分を$a_{i,j}$とする.
  このとき,
  $E_mA$の$(i,j)$-成分と
  $A$の$(i,j)$-成分が
  等しいことを示す.

  $\delta_{i,j}$をクロネッカーの$\delta$とすると,
  $E_m$の$(i,j)$-成分は$\delta_{i,j}$である.
  $E_mA$の$(i,j)$-成分は,
  $\sum_{k=1}^{m}\delta_{i,k}a_{k,m}$
  である.
  $\delta_{i,i}=1$であるが,
  $k\neq i$に対して, $\delta_{i,k}=0$であるので,
  \begin{align*}
    \sum_{k=1}^{m}\delta_{i,k}a_{k,j}
    &=\sum_{k\in \Set{1,2,\ldots,m}}\delta_{i,k}a_{k,j}\\
    &=\delta_{i,i}a_{i,j} + \sum_{k\in \Set{1,2,\ldots,m}\setminus\Set{i}}\delta_{i,k}a_{k,j}\\
    &=1a_{i,j} + \sum_{k\in \Set{1,2,\ldots,m}\setminus\Set{i}}0a_{k,j}\\
    &=a_{i,j}.
  \end{align*}
\end{proofof*}

\begin{proofof*}{matspace_is_algebra}{rightdist}
  $A$を$(m,n)$-行列とし, $(i,j)$-成分を$a_{i,j}$とする.
  $A'$を$(m,n)$-行列とし, $(i,j)$-成分を$a'_{i,j}$とする.
  $B$を$(n,k)$-行列とし, $(i,j)$-成分を$b_{i,j}$とする.
  $S=A+A'$とする.
  このとき,
  $SB$の$(i,j)$-成分と
  $AB+A'B$の$(i,j)$-成分が
  等しいことを示す.

  $S$の$(i,j)$-成分を$s_{i,j}$とすると,
  $s_{i,j}=a_{i,j}+a'_{i,j}$である.
  したがって
  $SB$の$(i,j)$-成分は,
  \begin{align*}
    \sum_{t=1}^{n}s_{i,t}b_{t,j}
    &=\sum_{t=1}^{n}(a_{i,t}+a'_{t,j})b_{t,j}\\
    &=\sum_{t=1}^{n}(a_{i,t}b_{t,j}+a'_{t,j}b_{t,j})\\
    &=\sum_{t=1}^{n}a_{i,t}b_{t,j}+\sum_{t=1}^{n}a'_{t,j}b_{t,j}
  \end{align*}
  である.
  一方,
  $AB$の$(i,j)$-成分は$\sum_{t=1}^{n}a_{i,t}b_{t,j}$,
  $A'B$の$(i,j)$-成分は$\sum_{t=1}^{n}a'_{i,t}b_{t,j}$である.
  したがって, $AB+A'B$の
  $(i,j)$-成分は$\sum_{t=1}^{n}a_{i,t}b_{t,j}+\sum_{t=1}^{n}a'_{i,t}b_{t,j}$である.
\end{proofof*}

\begin{proofof*}{matspace_is_algebra}{leftdist}
  $A$を$(m,n)$-行列とし, $(i,j)$-成分を$a_{i,j}$とする.
  $B$を$(n,k)$-行列とし, $(i,j)$-成分を$b_{i,j}$とする.
  $B'$を$(n,k)$-行列とし, $(i,j)$-成分を$b'_{i,j}$とする.
  $S=B+B'$とする.
  このとき,
  $AS$の$(i,j)$-成分と
  $AB+AB'$の$(i,j)$-成分が
  等しいことを示す.

  $S$の$(i,j)$-成分を$s_{i,j}$とすると,
  $s_{i,j}=b_{i,j}+b'_{i,j}$である.
  したがって
  $AS$の$(i,j)$-成分は,
  \begin{align*}
    \sum_{t=1}^{n}a_{i,t}s_{t,j}
    &=\sum_{t=1}^{n}a_{i,t}(b_{t,j}+b'_{t,j})\\
    &=\sum_{t=1}^{n}(a_{i,t}b_{t,j}+a_{i,t}b'_{t,j})\\
    &=\sum_{t=1}^{n}a_{i,t}b_{t,j}+\sum_{t=1}^{n}a_{i,t}b'_{t,j}
  \end{align*}
  である.
  一方,
  $AB$の$(i,j)$-成分は$\sum_{t=1}^{n}a_{i,t}b_{t,j}$,
  $AB'$の$(i,j)$-成分は$\sum_{t=1}^{n}a_{i,t}b'_{t,j}$である.
  したがって, $AB+AB'$の
  $(i,j)$-成分は$\sum_{t=1}^{n}a_{i,t}b_{t,j}+\sum_{t=1}^{n}a_{i,t}b'_{t,j}$である.
\end{proofof*}

\begin{proofof*}{matspace_is_algebra}{scalar:ass}
  $A$を$(m,n)$-行列とし, $(i,j)$-成分を$a_{i,j}$とする.
  $B$を$(n,k)$-行列とし, $(i,j)$-成分を$b_{i,j}$とする.
  $\alpha$を数とする.
  このとき,
  $(\alpha A)B$の$(i,j)$-成分と
  $\alpha(AB)$の$(i,j)$-成分が
  等しいことを示す.

  $\alpha A$の$(i,j)$-成分は$\alpha a_{i,j}$である.
  したがって,
  $(\alpha A)B$の$(i,j)$-成分は$\sum_{t=1}^{n} \alpha a_{i,t}b_{t,j}$
  である.
  一方
  $AB$の$(i,j)$-成分は$\sum_{t=1}^{n} a_{i,t}b_{t,j}$である.
  したがって,
  $\alpha(AB)$の$(i,j)$-成分は$\alpha\sum_{t=1}^{n} a_{i,t}b_{t,j}=\sum_{t=1}^{n} \alpha a_{i,t}b_{t,j}$である.
\end{proofof*}
  
\begin{proofof*}{matspace_is_algebra}{scalar:center}
  $A$を$(m,n)$-行列とし, $(i,j)$-成分を$a_{i,j}$とする.
  $B$を$(n,k)$-行列とし, $(i,j)$-成分を$b_{i,j}$とする.
  $\alpha$を数とする.
  このとき,
  $\alpha (AB)$の$(i,j)$-成分と
  $A(\alpha B)$の$(i,j)$-成分が
  等しいことを示す.

  $AB$の$(i,j)$-成分は$\sum_{t=1}^{n} a_{i,t}b_{t,j}$である.
  したがって,
  $\alpha(AB)$の$(i,j)$-成分は$\alpha\sum_{t=1}^{n} a_{i,t}b_{t,j}=\sum_{t=1}^{n} \alpha a_{i,t}b_{t,j}$である.
  一方
  $\alpha B$の$(i,j)$-成分は$\alpha b_{i,j}$である.
  したがって,
  $A(\alpha B)$の$(i,j)$-成分は$\sum_{t=1}^{n}  a_{i,t}\alpha b_{t,j}=\sum_{t=1}^{n}  \alpha a_{i,t} b_{t,j}$
  である.
\end{proofof*}
  
\begin{proofof}{mat:zeroprod}
  $A$を$(m,n)$行列とする.

  $1+(-1)=0$であるので,
  \begin{align*}
    0A&=(1+(-1))A\\
    &=1A+(-1)A\\
    &=A-A\\
    &=O_{m,n}\\
  \end{align*}
  である.

  $O_{k,m}+(-1)O_{k,m}=O_{k,m}$であるので,
  \begin{align*}
    O_{k,m}A&=(O_{k,m}+(-1)O_{k,m})A\\
    &=O_{k,m}A+((-1)O_{k,m})A\\
    &=O_{k,m}A+(-1)(O_{k,m}A)\\
    &=O_{k,m}\\
  \end{align*}
  である.

  $O_{n,k}+(-1)O_{n,k}=O_{n,k}$であるので,
  \begin{align*}
    AO_{n,k}&=A(O_{n,k}+(-1)O_{n,k})\\
    &=AO_{n,k}+A((-1)O_{n,k})\\
    &=AO_{n,k}+(-1)(AO_{n,k})\\
    &=O_{n,k}\\
  \end{align*}
  である.
\end{proofof}

\begin{proofof*}{mat:posipower}{prod}
  $A$を$n$次正方行列とする.
  $k$に関する数学的帰納法により, $A^{k}A^{k'}=A^{k+k'}$を示す.
  \paragraph{Base case}
  $A^{k'+1}$の定義から$AA^{k'}=A^{k'+1}$
  である.
  
  \paragraph{Induction Step}
  $A^{k-1}A^{k'}=A^{k-1+k'}$を仮定し,
  $A^{k}A^{k'}=A^{k+k'}$を示す.

  $A^{k}$の定義から
  \begin{align*}
    A^{k}=AA^{k-1}
  \end{align*}
  である. したがって,
  \begin{align*}
    A^{k}A^{k'}=AA^{k-1}A^{k'}=AA^{k-1+k'}
  \end{align*}
  である.
  一方 $A^{k+k'}$の定義から,
  \begin{align*}
    A^{k+k'}=AA^{k+k'-1}=AA^{k-1+k'}
  \end{align*}
  である.
\end{proofof*}

\begin{proofof*}{mat:posipower}{power}
  $A$を$n$次正方行列とする.
  $k'$に関する数学的帰納法により, $(A^{k})^{k'}=A^{kk'}$を示す.
  \paragraph{Base case}
  $(A^{k})^1$の定義から$(A^{k})^1=A^{k}=A^{k\cdot 1}$
  である.
  
  \paragraph{Induction Step}
  $(A^{k})^{k'-1}=A^{k}(A^{k(k'-1)})$を仮定し,
  $(A^{k})^{k'}=A^{k}(A^{kk'})$を示す.

  $(A^{k})^{k'}$の定義から
    $(A^{k})^{k'}=A^{k}(A^{k})^{k'-1}$
  である. したがって,
  \begin{align*}
    (A^{k})^{k'}&=A^{k}(A^{k})^{k'-1}\\
    &=A^{k}(A^{k(k'-1)})\\
    &=A^{k}A^{k(k'-1)}\\
    &=A^{k+k(k'-1)}\\
    &=A^{k(1+k'-1)}\\
    &=A^{kk'}
  \end{align*}
  である.
\end{proofof*}

\begin{proofof}{mat:posipower:scalar}
  $A$を$n$次正方行列とする.
  $\alpha$を数とする.
  $k'$に関する数学的帰納法により, $(\alpha A)^{k}=(\alpha^k)(A^{k})$を示す.

  \paragraph{Base case}
  $(\alpha A)^1$の定義から$(\alpha A)^1=\alpha A=\alpha^1A^{1}$
  である.
  
  \paragraph{Induction Step}
  $(\alpha A)^{k-1}=\alpha^{k-1} A^{k-1}$を仮定し,
  $(\alpha A)^{k}=\alpha^{k} A^{k}$を示す.
  
  $(\alpha A)^{k}$の定義から
    $(\alpha A)^{k}=(\alpha A)(\alpha A)^{k-1}$
  である. したがって,
  \begin{align*}
    (\alpha A)^{k}
    &=(\alpha A)(\alpha A)^{k-1}\\
    &=(\alpha A)(\alpha^{k-1} A^{k-1})\\
    &=\alpha^{k-1}((\alpha A)(A^{k-1}))\\
    &=\alpha^{k-1}\alpha AA^{k-1}\\
    &=\alpha^{k-1+1} A^{k-1+1}\\
    &=\alpha^{k} A^{k}\\
  \end{align*}
  である.
  
\end{proofof}


%% \begin{proofof}{thm:cht:2dim}
%%   \begin{align*}
%%     A=
%%     \begin{pmatrix}
%%       a&b\\c&d
%%     \end{pmatrix}
%%   \end{align*}
%%   とする.
%%   このとき,
%%   \begin{align*}
%%     A^2=
%%     \begin{pmatrix}
%%       a^2+bc&ab+bd\\ca+dc&cb+d^2
%%     \end{pmatrix}
%%   \end{align*}
%%   である. したがって,
%%   \begin{align*}
%%     &A^2-(a+d)A+(ad-bc)E_2\\
%%     &=
%%     \begin{pmatrix}
%%       a^2+bc&ab+bd\\ca+dc&cb+d^2
%%     \end{pmatrix}
%%     -(a+d)
%%     \begin{pmatrix}
%%       a&b\\c&d
%%     \end{pmatrix}
%%     +(ad-bc)
%%     \begin{pmatrix}
%%       1&0\\0&1
%%     \end{pmatrix}\\
%%     &=
%%     \begin{pmatrix}
%%       a^2+bc&ab+bd\\ca+dc&cb+d^2
%%     \end{pmatrix}
%%     +
%%     \begin{pmatrix}
%%       -(a+d)a&-(a+d)b\\-(a+d)c&-(a+d)d
%%     \end{pmatrix}
%%     +
%%     \begin{pmatrix}
%%       ad-bc&0\\0&ad-bc
%%     \end{pmatrix}\\
%%     &=
%%     \begin{pmatrix}
%%       a^2+bc&ab+bd\\ac+cd&bc+d^2
%%     \end{pmatrix}
%%     +
%%     \begin{pmatrix}
%%       -a^2-ad&-ab-bd\\-ac-dc&-ad-d^2
%%     \end{pmatrix}
%%     +
%%     \begin{pmatrix}
%%       ad-bc&0\\0&ad-bc
%%     \end{pmatrix}\\
%%     &=
%%     \begin{pmatrix}
%%       a^2+bc-a^2-ad+ad-bc&ab+bd-ab-bd+0\\ac+cd-ac-cd+0&bc+d^2-ad-d^2+ad-bc
%%     \end{pmatrix}\\
%%     &=
%%     \begin{pmatrix}
%%       0&0\\0&0
%%     \end{pmatrix}\\
%%   \end{align*}
%% \end{proofof}

%\endinput

\section{正則行列に関する命題の証明}
\begin{proofof}{porp:inverse:unique}
  $CAB$について考える.
  \begin{align*}
    (CA)B=E_nB=B,\\
    C(AB)=CE_n=C
  \end{align*}
  であるので, $B=C$.
\end{proofof}


\begin{proofof*}{inverse}{inverse}
  $X$を$n$次正則行列とする.
  $A=X^{-1}$とし,
  $B=X$とする.
  このとき, $A$が正則であることを示す.
  そのために$B$が$A$の逆行列であることを示す.

  まず$AB=E_n$となることを示す.
  $X^{-1}$は$X$の逆行列であるので,
  \begin{align*}
    AB=X^{-1}X=E_n
  \end{align*}
  である.

  次に$BA=E_n$となることを示す.
  $X^{-1}$は$X$の逆行列であるので,
  \begin{align*}
    BA=XX^{-1}=E_n
  \end{align*}
  である.
\end{proofof*}

\begin{proofof*}{inverse}{trans}
  $X$を$n$次正則行列とする.
  $A=\transposed{X}$とし,
  $B=\transposed{X^{-1}}$とする.
  このとき, $A$が正則であることを示す.
  そのために$B$が$A$の逆行列であることを示す.

  まず$AB=E_n$となることを示す.
  $X^{-1}$は$X$の逆行列であるので,
  \begin{align*}
    AB=\transposed{X}\transposed{X^{-1}}=\transposed{(X^{-1}X)}=\transposed{E_n}=E_n
  \end{align*}
  である.

  次に$BA=E_n$となることを示す.
  $X^{-1}$は$X$の逆行列であるので,
  \begin{align*}
    BA=\transposed{X^{-1}}\transposed{X}=\transposed{(XX^{-1})}=\transposed{E_n}=E_n
  \end{align*}
  である.
\end{proofof*}


\begin{proofof*}{inverse}{scalar}
  $X$を$n$次正則行列とする.
  $r$を$0$でない数とする.
  $A=rX$とし,
  $B=\frac{1}{r}X^{-1}$とする.
  このとき, $A$が正則であることを示す.
  そのために$B$が$A$の逆行列であることを示す.

  まず$AB=E_n$となることを示す.
  $X^{-1}$は$X$の逆行列であるので,
  \begin{align*}
    AB=(rX)(\frac{1}{r}X^{-1})=\frac{1}{r}(rXX^{-1})=(\frac{1}{r}r)(XX^{-1})=1E_n=E_n
  \end{align*}
  である.

  次に$BA=E_n$となることを示す.
  $X^{-1}$は$X$の逆行列であるので,
  \begin{align*}
    AB=(\frac{1}{r}X^{-1})(rX)=r(\frac{1}{r}X^{-1}X)=(r\frac{1}{r})(X^{-1}X)=1E_n=E_n
  \end{align*}
  である.
\end{proofof*}

\begin{proofof*}{inverse}{prod}
  $X$, $Y$を$n$次正則行列とする.
  $A=XY$とし,
  $B=Y^{-1}X^{-1}$とする.
  このとき, $A$が正則であることを示す.
  そのために$B$が$A$の逆行列であることを示す.

  まず$AB=E_n$となることを示す.
  $X^{-1}$は$X$の逆行列,
  $Y^{-1}$は$Y$の逆行列
  であるので,
  \begin{align*}
    AB=(XY)(Y^{-1}X^{-1})=X(YY^{-1})X^{-1}=XE_nX^{-1}=XX^{-1}=E_n
  \end{align*}
  である.

  次に$BA=E_n$となることを示す.
  $X^{-1}$は$X$の逆行列,
  $Y^{-1}$は$Y$の逆行列
  であるので,
  \begin{align*}
    BA=(Y^{-1}X^{-1})(XY)=Y^{-1}(X^{-1}X)Y=Y^{-1}E_nY=Y^{-1}Y=E_n
  \end{align*}
  である.
\end{proofof*}



\begin{proofof*}{inverse}{power}
  $X$を$n$次正則行列とし,
  $l$を正の整数とする.
  このとき, $X^l$が正則であることを示す.
  そのために$(X^{-1})^l$が$X^l$の逆行列であることを,
  $l$に関する数学的帰納法により示す.

  \paragraph{Base case}
  $(X)^{1}=X$, $(X^{-1})^{1}=X^{-1}$であるから,
  $(X^{-1})^{1}$が
  $(X)^{1}$の逆行列である.

  \paragraph{Induction Step}
  $(X^-1)^{l-1}$が$(X)^{l-1}$の逆行列であることを仮定して,
  $(X^-1)^{l}$が$(X)^{l}$の逆行列であることを示す.
  
  
  まず, $(X^l)((X^{-1})^l)=E_n$となることを示す.
  $X^l=X^{l-1}X$, 
  $(X^{-1})^l=X^{-1}(X^{-1})^{l-1}$であるから,
  \begin{align*}
    (X^l)((X^{-1})^l)
    &=(X^{l-1}X)(X^{-1}(X^{-1})^{l-1})\\
    &=X^{l-1}(XX^{-1})(X^{-1})^{l-1}\\
    &=X^{l-1}E_n(X^{-1})^{l-1}\\
    &=X^{l-1}(X^{-1})^{l-1}\\
    &=E_n
  \end{align*}
  である.

  次に, $((X^{-1})^l)(X^l)=E_n$となることを示す.
  $X^l=X^{l-1}X$, 
  $(X^{-1})^l=X^{-1}(X^{-1})^{l-1}$であるから,
  \begin{align*}
    ((X^{-1})^l)(X^l)
    &=(X^{-1}(X^{-1})^{l-1})(X^{l-1}X)\\
    &=X^{-1}((X^{-1})^{l-1})X^{l-1})X\\
    &=X^{-1}E_nX\\
    &=X^{-1}X\\
    &=E_n
  \end{align*}
  である.
\end{proofof*}



  
\begin{proofof*}{negpower}{prod}
  $X$を$n$次正則行列とする.
  このとき, 整数$k$, $k'$に対し, $X^{k}X^{k'}=X^{k+k'}$となることを示す.
  
  $X^{0}=E_n$であるので,
  $X^{0}X^{k'}=E_nX^{k'}=X^{k'}$である.
  また,
  $X^{k}X^{0}=X^{k}E_n=X^{k}$でもある.

  
  $k$も$k'$も$0$ではない場合について考える.
  $p$, $p'$を正の整数とする.
  $k=p$も$k'=p'$である場合は,
  \Cref{thm:mat:posipower}である.
  $k=-p$も$k'=-p'$である場合は,
  \begin{align*}
    A^{-p}A^{-p'}&=(A^{-1})^{p}(A^{-1})^{p'}=(A^{-1})^{p+p'}=A^{-(p+p')}=A^{(-p)+(-p')}.
  \end{align*}
  であることがわかる.

  ここまでの議論で,
  $k\leq 0$かつ$k'\leq 0$の場合と,
  $k\geq 0$かつ$k'\geq 0$の場合は証明できた.  
  それら以外の場合について考える.

  
  $p$, $p'$を正の整数とし,
  $k=-p$かつ$k'=p'$の場合と
  $k=p$かつ$k'=-p'$の場合について示す.
  $p=p'$なら
  \begin{align*}
    A^{-p}A^{p'}&=A^{-p}A^{p}=(A^{p})^{-1}A^{p}=E_n=A^{0}=A^{-p+p}=A^{-p+p'}\\
    A^{p}A^{-p'}&=A^{p}A^{-p}=(A^{p})^{-1}A^{p}=E_n=A^{0}=A^{p-p}=A^{p-p'}
  \end{align*}
  である.
  $p>p'$であるなら, $p=d+p'$とおくと,
  \begin{align*}
    A^{-p}A^{p'}=A^{-(d+p')}A^{p'}
    &=(A^{-d}A^{-p'})A^{p'}\\
    &=A^{-d}(A^{-p'}A^{p'})=A^{-d}A^{0}=A^{-d}=A^{-p+p'}\\
    A^{p}A^{-p'}=A^{d+p'}A^{-p'}
    &=(A^{d}A^{p'})A^{-p'}\\
    &=A^{d}(A^{p'}A^{-p'})=A^{d}A^{0}=A^{d}=A^{p-p'}
  \end{align*}
  である.
  $p<p'$であるなら, $p+d=p'$とおくと,
  \begin{align*}
    A^{-p}A^{p'}&=A^{-p}A^{p+d}
    =A^{-p}A^{p}A^{d}=A^{-p+p}A^{d}=A^{0}A^{d}=A^{d}=A^{-p+p'},\\ 
    A^{p}A^{-p'}&=A^{p}A^{-(p+d)}=A^{p}A^{-p-d)}=A^{p}A^{-p}A^{-d}=A^{0}A^{d}=A^{0+d}=A^{p-p'}
 \end{align*}
  である.

\end{proofof*}



  
\begin{proofof*}{negpower}{power}
  $X$を$n$次正則行列とする.
  このとき, 整数$k$, $k'$に対し, $(X^{k})^{k'}=X^{kk'}$となることを示す.

  定義から
  $(X^{k})^{0}=E_n$である.
  また,
  $X^{k\cdot 0}=X^0=E_n$
  であるので, $(X^{k})^{0}=X^{k\cdot0}$である.

  定義から
  $(X^{0})^{k'}=E_n^{k'}=E_n$である.
  また,
  $X^{0k'}=X^0=E_n$
  であるので, $(X^{0})^{k'}=X^{0\cdot k'}$である.

  
  $X^{-k}X^{k}=X^0=E_n$かつ
  $X^{k}X^{-k}=X^0=E_n$であるので,
  $X^{-k}$は$X^k$の逆行列である.
  したがって$(X^{k})^{-1}=X^{-k}=X^{k\cdot -1}$である.

  $p'$を正の整数とすると,
  \Cref{thm:inverse:operation}より,
  $(X^{-1})^{p'}$は$X^{p'}$の逆行列である.
  したがって$(X^{-1})^{p'}=(X^{p'})^{-1}=X^{-p'}=X^{-1\cdot p'}$である.

  $p$, $p'$を正の整数とすると,
  $k=p$, $k'=p'$のときは,
  \Cref{thm:mat:posipower}より,
  $(X^{k})^{k'}=X^{kk'}$である.

  $k=p$, $k'=-p'$のときは,
  \begin{align*}
    (X^{p})^{-p'}&=((X^{p})^{p'})^{-1}=(X^{pp'})^{-1}=X^{-pp'}
  \end{align*}
  である.

  $k=-p$, $k'=p'$のときは,
  \begin{align*}
    (X^{-p})^{p'}&=((X^{p})^{-1})^{p'}=(X^{p})^{-p'}=X^{-pp'}
  \end{align*}
  である.

  $k=-p$, $k'=-p'$のときは,
  \begin{align*}
    (X^{-p})^{-p'}&=((X^{-p})^{p'})^{-1}=(X^{-pp'})^{-1}=X^{-(-pp')}=X^{(-p)(-p')}
  \end{align*}
  である.

\end{proofof*}

\begin{proofof*}{zerodivisor}{left}
  背理法により示す.
  $A$を正則とすると,
  $AX=O_{n,m}$の両辺に左から$A^{-1}$をかけることにより,
  \begin{align*}
    A^{-1}AX&=A^{-1}O_{n,m}\\
    X&=O_{n,m}
  \end{align*}
  となり, $X$が零行列でないことに反する.
\end{proofof*}

\begin{proofof*}{zerodivisor}{right}
  背理法により示す.
  $A$を正則とすると,
  $YA=O_{m,n}$の両辺に右から$A^{-1}$をかけることにより,
  \begin{align*}
    YAA^{-1}&=O_{m,n}A^{-1}\\
    Y&=O_{m,n}
  \end{align*}
  となり, $Y$が零行列でないことに反する.
\end{proofof*}


\begin{proofof}{prop:zerodivsquare}
  \Cref{prop:zerodiv:left}から,
  $A$が正則でないことがわかる.
  また, \Cref{prop:zerodiv:left}から,
  $B$が正則でないことがわかる.
\end{proofof}

\begin{proofof}{prop:matwithzerovec}
  例えば, $Y=\begin{pmatrix}0&\cdots0&1\end{pmatrix}$とすると,
  $YA=O_{1,n}$であるので,
  \Cref{prop:zerodiv}から,
  $A$が正則ではないことがわかる.
\end{proofof}


ある成分だけ$1$であり他は$0$である行列を行列単位と呼ぶことがある.
どの行列も行列単位の線型結合として表すことができる.
また, 行列単位の積は,
定義に従って計算すると次のようになる.
\begin{lemma}
  $B_{i,j}$を$(i,j)$-成分のみ$1$で他は$0$の$n$次正方行列とする.
  \begin{align*}
    B_{i,j}B_{k,l}=
    \begin{cases}
      B_{i,l}&(j=k)\\
      O_{n,m}&(j\neq k)
    \end{cases}
  \end{align*}
\end{lemma}
したがって, 次がわかる.
\begin{lemma}
  \label{lemma:matrixunit:rightprod}
  $B_{i,j}$を$(i,j)$-成分のみ$1$で他は$0$の$n$次正方行列とする.
  また, $A$を$(n,m)$行列とする.
  このとき,
  $B_{i,j}A$は,
  $i$行目が$A$の$j$行目と等しく, それ以外の成分は$0$である.  
\end{lemma}

\begin{proofof*}{fundamentalmat:rowtrans}{F}
  $F(i;c)A=(E_n-E_{i,i}+cE_{i,i})A=A+(c-1)E_{i,i}A$である.
  したがって,
  \Cref{lemma:matrixunit:rightprod}から,
  $F(i;c)A$の$i$行目は$A$の$i$行目に$A$の$i$行目の$c-1$倍を加えたものであるので,
  $A$の$i$行目の$c$倍である.
  また$i$行目以外は$F(i;c)A$と$A$は等しい.
  つまり,
  $F(i;c)A$は$A$の$i$行目を$c$倍した行列である.
\end{proofof*}
\begin{proofof*}{fundamentalmat:rowtrans}{G}
  $G(i,j;c)A=(E_n+cE_{i,j})A=A+cE_{i,j}A$である.
  したがって,
  \Cref{lemma:matrixunit:rightprod}から,
  $G(i,j;c)A$の$i$行目は$A$の$i$行目に$A$の$j$行目の$c$倍を加えたものである.
  また$i$行目以外は$G(i,j;c)A$と$A$は等しい.
  つまり, $G(i,j;c)A$は$A$の$i$行目に$j$行目の$c$倍を加えた行列である.
\end{proofof*}
\begin{proofof*}{fundamentalmat:rowtrans}{H}
  $H(i,j)A=(E_n-E_{i,i}-E_{j,j}+E_{i,j}+E_{i,j})A=A-E_{i,i}A+E_{i,j}A-E_{j,j}A+E_{j,i}A$
  である.
  したがって,
  \Cref{lemma:matrixunit:rightprod}から,
  $H(i,j)A$の$i$行目は,
  $A$の$i$行めに$A$の$i$行目の$-1$倍と$A$の$j$行目の$1$倍を加えたものであるので,
  $A$の$j$行目である.
  同様に, $H(i,j)A$の$j$行目は,
  $A$の$j$行めに$A$の$j$行目の$-1$倍と$A$の$i$行目の$1$倍を加えたものであるので,
  $A$の$i$行目である.
  また$i$行目と$j$行目以外は$H(i,j)A$と$A$は等しい.
  つまり$H(i,j)A$は$A$の$i$行目と$j$行目を入れ替えた行列である.
\end{proofof*}

\begin{proofof*}{fundamentalmat:coltrans}{F}
  $\transposed{(AF(i;c))}=\transposed{F(i;c)}\transposed{A}=F(i;c)\transposed{A}$
  である.
  \Cref{lem:fundamental}
  をつかうと,
  $\transposed{(AF(i;c))}$は,
  $\transposed{A}$の$i$行目を$c$倍した行列であることがわかる.
  $AF(i;c)=\transposed{(\transposed{(AF(i;c))})}$
  であるので,
  $AF(i;c)$は$A$の$i$列目を$c$倍した行列であることがわかる.
\end{proofof*}

\begin{proofof*}{fundamentalmat:coltrans}{G}
  $\transposed{(AG(i,j;c))}=\transposed{G(i,j;c)}\transposed{A}=G(j,i;c)\transposed{A}$
  である.
  \Cref{lem:fundamental}
  をつかうと,
  $\transposed{(AG(i,j;c))}$は,
  $\transposed{A}$の$j$行目に$i$行目の$c$倍を加えた行列であることがわかる.
  $AG(i,j;c)=\transposed{(\transposed{(AG(i,j;c))})}$
  であるので,
  $AG(i,j;c)$は$A$の$j$列目に$i$列目の$c$倍を加えた行列であることがわかる.
\end{proofof*}

\begin{proofof*}{fundamentalmat:coltrans}{H}
  $\transposed{(AH(i,j))}=\transposed{H(i,j)}\transposed{A}=H(j,i)\transposed{A}$
  である.
  \Cref{lem:fundamental}
  をつかうと,
  $\transposed{(AH(i,j))}$は,
  $\transposed{A}$の$j$行目と$i$行目を入れ替えた行列であることがわかる.
  $AH(i,j)=\transposed{(\transposed{(AH(i,j))})}$
  であるので,
  $AH(i,j)$は$A$の$i$列目と$j$列目を入れ替えた行列であることがわかる.
\end{proofof*}

\begin{proofof*}{fundamentalmat:invertible}{F}
  $F(i;c)$は$(i,i)$-成分のみ$c$で
  他の対角成分は$c$である対角行列である.

  $F(i;c)F(i;\frac{1}{c})$は$F(i;\frac{1}{c}$の$i$行目を$c$倍したものであるので,
  $E_n$である.
  $F(i;\frac{1}{c})F(i;c)$は$F(i;c$の$i$行目を$\frac{1}{c}$倍したものであるので,
  $E_n$である.
  よって$F(i;c)^{-1}=F(i;\frac{1}{c})$である.
\end{proofof*}
\begin{proofof*}{fundamentalmat:invertible}{G}
  $G(i,j;c)$は, 対角成分は全て$1$, $(i,j)$-成分は$c$, 他の成分は全て$0$である.

  $G(i,j;c)G(i,j;-c)$は, $G(i,j;-c)$の$i$行目に$G(i,j;-c)$の$j$行目の$c$倍を加えたものであるが, $G(i,j;-c)$の$j$行目は$(j,j)$成分が$1$で他は$0$であるので,
  $(i,j)$成分に$c$が加えられることになる.
  したがって, $G(i,j;c)G(i,j;-c)=E_n$である.
  一方$G(i,j;-c)G(i,j;c)$は, $G(i,j;c)$の$i$行目に$G(i,j;c)$の$j$行目の$-c$倍を加えたものであるが, $G(i,j;c)$の$j$行目は$(j,j)$成分が$1$で他は$0$であるので,
  $(i,j)$成分に$-c$が加えられることになる.
  したがって, $G(i,j;-c)G(i,j;c)=E_n$である.
  よって, $G(i,j;c)^{-1}=G(i,j;-c)$である.
\end{proofof*}
\begin{proofof*}{fundamentalmat:invertible}{H}
  $H(i,j)$は$(i,i)$-成分と$(j,j)$-成分は$0$, それ以外の対角成分は$1$,
  $(i,j)$-成分と$(j,i)$-成分は$1$, それ以外の成分は$0$である.

  $H(i,j)H(i,j)$は$H(i,j)$の$i$行目と$j$行目を入れ替えたものであるので,
  $E_n$である.
  したがって, $H(i,j)^{-1}=H(i,j)$である.
\end{proofof*}



\subsection{行基本変形によって得られる被約階段行列の一意性に関する命題の証明}
行基本変形を用いて得られる被約階段行列が,
変形の手順によらず一意に定まることを示す.
そのために, いくつか補題を用意する.


$P$を$m$次正方行列とする.
$\xx$を$(m,1)$行列とする.
$\zzero$を零行列$O_{m,1}$とする.
このとき次が成り立つことは既に見た:
\begin{lemma}
  \label{lem:1}
  $\xx\neq\zzero$かつ$P\xx=\zzero$ならば,
  $P$は正則ではない.
\end{lemma}



$i=1,2,\ldots,m$に対し,
  \begin{align*}
    \ee_i=\begin{pmatrix}0\\\vdots\\0\\1\\0\\\vdots\\0\end{pmatrix}
  \end{align*}
という$i$行目だけ$1$で, 他は$0$である$(m,1)$行列とする.
\begin{lemma}
  \label{lem:2}
  $k\geq 1$とする.
  \begin{align*}
    P\ee_1&=\ee_1,\\
    P\ee_2&=\ee_2,\\
    &\vdots\\
    P\ee_{k-1}&=\ee_{k-1},\\
    P\ee_k&=\alpha_1\ee_1+\alpha_2\ee_2+\cdots+\alpha_{k-1}\ee_{k-1}\\
    &(\text{$k=1$のときには, $P\ee_1=P\ee_k=0$ という条件として読む})    
  \end{align*}
  ならば, $P$は正則ではない.
\end{lemma}
\begin{proof}
$\ee_k-(\alpha_1\ee_1+\alpha_2\ee_2+\cdots+\alpha_{k-1}\ee_{k-1})$
  は第$k$成分が1なので, $\zzero$ではない.
  \begin{align*}
    &\phantom{{}={}}P(\ee_k-(\alpha_1\ee_1+\alpha_2\ee_2+\cdots+\alpha_{k-1}\ee_{k-1}))\\
    &=
    P\ee_k-(\alpha_1P\ee_1+\alpha_2P\ee_2+\cdots+\alpha_{k-1}P\ee_{k-1}))\\
    &=
    P\ee_k-(\alpha_1\ee_1+\alpha_2\ee_2+\cdots+\alpha_{k-1}\ee_{k-1}))\\
    &=
    P\ee_k-P\ee_k\\
    &=\zzero.
  \end{align*}
  したがって\cref{lem:1}より$P$は正則ではない.
\end{proof}


$A=(a_{i,j})_{\substack{i=1,\ldots,m\\j=1,\ldots,n}}$
$(m,n)$行列とする.
また,
$A=(\aaa_1|\aaa_2|\cdots|\aaa_n)$
と列ベクトル表示されているとする.
つまり,
\begin{align*}
  \aaa_j=\begin{pmatrix}a_{j,1}\\a_{j,2}\\\vdots\\a_{j,m}\end{pmatrix}
\end{align*}
である. さらに, $A$は階数$r$の被約階段行列で,
そのpivotsは$(1,p_1),(2,p_r),\ldots,(r,p_r)$であるとする.

$A$か階数$r$の階段行列であることから次がわかる:
\begin{lemma}
  \label{lem:3}
  \begin{align*}
    \aaa_{j}&=a_{1,j}\ee_1+a_{2,j}\ee_2+\cdots+a_{r,j}\ee_r\\
    &=\sum_{i=1}^r a_{i,j}\ee_i.
  \end{align*}
\end{lemma}
また, $(l+1,p_{l+1})$がpivotであることから, 次がわかる:
\begin{lemma}
  \label{lem:4}
  $j<p_{l+1}$ならば,
  \begin{align*}
    \aaa_{j}&=a_{1,j}\ee_1+a_{2,j}\ee_2+\cdots+a_{l,j}\ee_l\\
    &=\sum_{i=1}^l a_{i,j}\ee_i.
  \end{align*}
\end{lemma}


\begin{lemma}
  \label{lem:5}
    \begin{align*}
    P\ee_1&=\ee_1,\\
    P\ee_2&=\ee_2,\\
    &\vdots\\
    P\ee_{r}&=\ee_{r}
  \end{align*}
  ならば, $PA=A$.
\end{lemma}
\begin{proof}
  \Cref{lem:3}より
  \begin{align*}
    \aaa_{j}&=a_{1,j}\ee_1+a_{2,j}\ee_2+\cdots+a_{r,j}\ee_r.
  \end{align*}
  であるので,
  \begin{align*}
    P\aaa_{j}
    &=a_{1,j}P\ee_1+a_{2,j}P\ee_2+\cdots+a_{r,j}P\ee_r\\
    &=a_{1,j}\ee_1+a_{2,j}\ee_2+\cdots+a_{r,j}\ee_r\\
    &=\aaa_j
  \end{align*}
  したがって
  \begin{align*}
    PA&=P(\aaa_1|\aaa_2|\cdots|\aaa_n)\\
    &=(P\aaa_1|P\aaa_2|\cdots|P\aaa_n)\\
    &=(\aaa_1|\aaa_2|\cdots|\aaa_n)\\
    &=A.
  \end{align*}
\end{proof}

 被約階段行列の定義から次がわかる:
\begin{lemma}
  \label{lem:6}
  $A$が被約階段行列で,
  $(i,p_i)$がpivotならば,
  $\aaa_{p_i}=\ee_i$.
\end{lemma}

$X$, $Y$を$(m,n)$行列とし,
$X=(\xx_1|\xx_2|\cdots|\xx_n)$,
$Y=(\yy_1|\yy_2|\cdots|\yy_n)$
と列ベクトル表示されているとする.
さらに$X$も$Y$も
被約階段行列であることを仮定する.

\begin{lemma}
  \label{lem:7}
  $X$と$Y$のpivotが完全に一致するとする.
  このとき$PX=Y$ならば$X=Y$.
\end{lemma}
\begin{proof}
  $X$と$Y$のpivotsが
  $(1,p_1),(2,p_r),\ldots,(r,p_r)$であるとする.
  \begin{align*}
    PX&=P(\xx_1|\cdots|\xx_n)=(P\xx_1|\cdots|P\xx_n)\\
    Y&=(\yy_1|\cdots|\yy_n)
  \end{align*}
  であり, $PX=Y$であるので,
  \begin{align*}
    (P\xx_1|\cdots|P\xx_n)=(\yy_1|\cdots|\yy_n).
  \end{align*}
  Pivotである$p_i$列目について考えると,
  \Cref{lem:6}より
  \begin{align*}
    P\xx_i&=P\ee_i,\\
    \yy_i&=\ee_i
  \end{align*}
  であるので$P\ee_i=\ee_i$である.
  $i\in\Set{1,\ldots,r}$に対し$P\ee_i=\ee_i$となるので,
  \Cref{lem:5}より$X=PX=Y$.  
\end{proof}

\begin{lemma}
  \label{lem:8}
  $X$のpivotsは$(1,p_1),\ldots,(r,p_r)$,
  $Y$のpivotsは$(1,p_1),\ldots,(l,q_l)$であるとする.
  $k\leq r$とする.
  次を仮定する.
  \begin{align*}
    i<k \implies p_i=q_i
  \end{align*}
  $PX=Y$をみたすとする.
  $P$が正則行列ならば
  $k\leq l$かつ$p_k\geq q_k$. 
\end{lemma}
\begin{proof}
  対偶となる,
  $k> l$または$p_k< q_k$ならば$P$は正則ではない
  ということを示す.
  
  \begin{align*}
    &PX=(P\xx_1|\cdots|P\xx_n)\\
    ={}&Y=(\yy_1|\cdots|\yy_n)
  \end{align*}
  であるので, $\yy_j=P\xx_j$.
  $j=p_t$のときには, \cref{lem:6}より,
  \begin{align}
    \yy_{p_t}=P\xx_{p_t}=P\ee_t.
    \label{eq:x}
  \end{align}

  まず$t\in\Set{1,\ldots,k-1}$のときについて考えておく.
  $p_t=q_t$であるので,
  \cref{lem:6}より,
  \begin{align}
    \yy_{p_t}=\yy_{q_t}=\ee_t.
    \label{eq:xx}
  \end{align}
  よって, \eqref{eq:x}と\eqref{eq:xx}より,
  \begin{align*}
    P\ee_t=\xx_{q_t}=\ee_t.
  \end{align*}


  つぎに, $t=k$のときについて考える.
  $k>l$ならば, \cref{lem:3}を使うと,
  \begin{align*}
    \yy_{p_k}=\yy_{1p_k}\ee_1+\cdots+y_{k-1,p_k}\ee_{k-1}
  \end{align*}
  と書ける.
  また, $p_k<q_k$ならば, \cref{lem:4}を使うと,
  \begin{align*}
    \yy_{p_k}=\yy_{1p_k}\ee_1+\cdots+y_{k-1,p_k}\ee_{k-1}
  \end{align*}
  と書ける.
  したがって,
  次が成り立つ:
  \begin{align*}
    k>l \text{ or } p_k<q_k
    \implies
    \yy_{p_k}=\yy_{1p_k}\ee_1+\cdots+y_{k-1,p_k}\ee_{k-1}.
  \end{align*}
  $k>l$または$p_k<q_k$が成り立つとする.
  このとき, \cref{eq:x}より,
  \begin{align*}
    P\ee_k
    =\yy_{p_k}
    =\yy_{1p_k}\ee_1+\cdots+y_{k-1,p_k}\ee_{k-1}
  \end{align*}
  となるが,
  \cref{lem:2}より$P$は正則ではない.
\end{proof}

\begin{proofof}{key:for:uniqueness}
  $PA=B$であるので, $B=P^{-1}A$でもある.
  
  $A$にpivotsがないとき:
  このときは$A=O_{m,n}$である.
  $B=PA=PO_{m,n}=O_{m,n}$
  であるので, $A=B$.

  $B$にpivotsがないとき:
  このときは$B=O_{m,n}$である.
  $A=P^{-1}B=PO_{m,n}=O_{m,n}$
  であるので, $A=B$.

  $A$にも$B$にもpivotsがあるとき:
  $A$のpivotsは$(1,p_1),\ldots,(r,p_r)$,
  $B$のpivotsは$(1,p_1),\ldots,(l,q_l)$であるとし,
  まず, pivotsが一致することを示す.
  
  $k=1$として,
  \cref{lem:8}を$PA=B$に使うと$p_1\geq q_1$.
  \cref{lem:8}を$P^{-1}B=A$に使うと$q_1\geq p_1$.
  よって$p_1=q_1$.

  $k=2$として,
  \cref{lem:8}を$PA=B$に使うと$p_2\geq q_2$.
  \cref{lem:8}を$P^{-1}B=A$に使うと$q_2\geq p_2$.
  よって$p_2=q_2$.

  以下同様に繰り返すと,
  \begin{align*}
    r&=l\\
    p_1&=q_1\\
    p_2&=q_2\\
    &\ \vdots\\
    p_r&=q_r
  \end{align*}
  がわかる.
  Pivotsが一致しているので,
  \cref{lem:7}を使うことで$A=B$がわかる.
\end{proofof}

\begin{proofof}{reg:for:uniqueness}
  $A=PX$, $B=QX$とおくと
  $A$, $B$は被約階段行列である.
  $P'=QP^{-1}$とおけば.
  $P'$は正則である.
  また,
  \begin{align*}
    P'A=QP^{-1}A=QP^{-1}P=QX=B
  \end{align*}
  であるので$P'A$は被約階段行列.
  よって,
  \Cref{lem:key:PX=Y:X=Y}
  より,
  $A=B$.
\end{proofof}
\begin{proofof}{reg:can:be:e}
\paragraph{\Cref{cor:reg:e:item:reg}$\implies$\Cref{cor:reg:e:item:e}}
\Cref{thm:FB}により,
$A$に対し行基本変形を行うことで被約階段行列が得られる.
これを$S$とする.
行基本変形は基本行列を左からかけることであったので,
$PA=S$を満たす正方行列$P$が存在する.
基本行列は正則であり, 正則行列の積もまた正則であったから,
$P$は正則である.
一方,
$A$は正則なので,
その逆行列を$Q$とおく.
このとき$Q$は正則で$QA=E_n$を満たす.
$E_n$は被約階段行列であったから,
\Cref{prop:reduced:PX=QX}より,
$PA=QA$である.
よって,
$S=E_n$.

\paragraph{\Cref{cor:reg:e:item:e}$\implies$\Cref{cor:reg:e:item:reg}}
行基本変形を用いて$A$を$E_n$に変形できたとする.
行基本変形は基本行列を左からかけることであったので,
$PA=E_n$を満たす正方行列$P$が存在する.
基本行列は正則であり, 正則行列の積もまた正則であったから,
$P$は正則である.
$PA=E_n$であるので,
両辺に左から$P^{-1}$, 右から$P$をかけることで,
次を得る.
\begin{align*}
  P^{-1}PAP&=P^{-1}P.\\
  AP&=E_n.
\end{align*}
$PA=E_n$, $AP=E_n$であるので,
$P$は$A$の逆行列であり,
$A$は正則である.
\end{proofof}


\begin{proofof}{fundmatisgenofgl}
  $A$を$n$次正則行列とする.
  $A$は行基本変形で$E_n$に変形できる.
  このことは基本行列$P_1,\ldots,P_k$を用いて,
  $P_1\cdots P_k A=E_n$と表せることを意味する.
  基本行列は正則なので,
  $A=P_k^{-1}\cdots P_1^{-1}$
  と書けるが, 基本行列の逆行列も基本行列であったから,
  各$P_i^{-1}$も基本行列である.
\end{proofof}

\subsection{階数に関する命題の証明}
\begin{proofof}{thm:rank:regular}
  $A$を$n$次正方行列であるとする.
  \paragraph{\Cref{thm:rank:regular:reg}$\implies$\Cref{thm:rank:regular:rank}}
  $A$を正則とする.
  \Cref{thm:reg:criterion:rowtrans}
  より,
  行基本変形により$A$を$E_n$に変形できる.
  したがって$\rank(A)=\rank(E_n)=n$.
  \paragraph{\Cref{thm:rank:regular:rank}$\implies$\Cref{thm:rank:regular:reg}}
  $\rank(A)=n$とする.
  階数の定義から
  $A$に行基本変形を行って階数$n$の被約階段に変形できる.
  $A$が$n$次正方行列であるので得られる被約階段も$n$次正方行列である.
  $n$次正方行列で被約階段であるものは$E_n$のみであるから,
  得られる被約階段は$E_n$である.
  したがって,
  \Cref{thm:reg:criterion:rowtrans}
  より,
  $A$は正則である.
\end{proofof}

\begin{proofof*}{thm:rank:prop}{fund}
  $A$に行基本変形を行って被約階段行列$S$が得られたとする.
  また, $A$に行基本変形を行って$A'$が得られたとする.
  このとき,
  \cref{lem:fund:is:invertible}から, 
  $A'$行基本変形を行って$A$にできる.
  したがって更に行基本変形を行って$S$にすることもできる.
  つまり
  $A'$に対し行基本変形を行って得られる被約階段行列も$S$である.
  したがって, $\rank(A)=\rank(S)=\rank(A')$である.
\end{proofof*}

\begin{proofof*}{thm:rank:prop}{row}
  被約階段行列$S$の階数は$0$以外の成分もある行の総数であるから$S$の行数を超えることはない.
  また行列$A$の階数は$A$から行基本変形で得られる被約階段行列$A'$の階数である.
  $A$の行数と$A'$の行数は等しいので, $A$の階数は$A$の行数を超えることはない.
\end{proofof*}
\begin{proofof*}{thm:rank:prop}{col}
  被約階段行列$S$の階数はpivotの総数であるが,
  pivotは各列に高々1つであるので, 階数は$S$の列数を超えることはない.
  また行列$A$の階数は$A$から行基本変形で得られる被約階段行列$A'$の階数である.
  $A$の列数と$A'$の列数は等しいので, $A$の階数は$A$の列数を超えることはない.
\end{proofof*}

\begin{lemma}
  \label{lem:rank:rnonzerorows}
  $(m,n)$-行列$A$の$r+1$行目から$m$行目までの成分がすべて$0$であるなら,
  $\rank(A)\leq r$である.
\end{lemma}
\begin{proof}
  $A'$を$(r,n)$-行列とし,
  $A'$と$O_{m-r,n}$を縦に並べてできる$(m,n)$行列を$A$とする.
  このとき, $A'$に行基本変形をして被約階段行列$S'$が得られたとすると,
  $A$に対し同じ手順で行基本変形をすることで,
  $S'$と$O_{m-r,n}$を縦に並べてできる$(m,n)$行列$S$が得られる.
  $S$は被約階段行列であり, $\rank(S)=\rank(S')$である.
  また\Cref{thm:rank:prop}の\Cref{thm:rank:prop:row}
  より, $\rank(S')\leq r$である.
  したがって, $\rank(A)=\rank(S)=\rank(S')\leq r$である.
\end{proof}

\begin{lemma}
  \label{lem:rank:rnonzerocols}
  $(m,n)$-行列$A$の$c+1$列目から$n$列目までの成分がすべて$0$であるなら,
  $\rank(A)\leq c$
\end{lemma}
\begin{proof}
  $A'$を$(m,c)$-行列とし,
  $A'$と$O_{m,n-c}$を横に並べてできる$(m,n)$行列を$A$とする.
  このとき, $A'$に行基本変形をして被約階段行列$S'$が得られたとすると,
  $A$に対し同じ手順で行基本変形をすることで,
  $S'$と$O_{m,n-c}$を横に並べてできる$(m,n)$行列$S$が得られる.
  $S$は被約階段行列であり, $\rank(S)=\rank(S')$である.
  また\Cref{thm:rank:prop}の\Cref{thm:rank:prop:col}
  より, $\rank(S')\leq c$である.
  したがって, $\rank(A)=\rank(S)=\rank(S')\leq c$である.
\end{proof}


\begin{proofof*}{thm:rank:prop}{left}
  $(m,n)$-行列$A$から行基本変形を用いて階数$r$の被約階段行列$S$が得られるとする.
  このとき,
  \cref{lem:fundamental}
  から,
  $P_kP_{k-1}\cdots P_1A=S$をみたす
  基本行列$P_1,\ldots, P_k$が存在する.
  $P_kP_{k-1}\cdots P_1A=S$であるので
  $P_kP_{k-1}\cdots P_1AB=SB$であるので
  つまり$AB$に行基本変形を用いて$SB$に変形できることがわかる.
  したがって,
  \Cref{thm:rank:prop}の\cref{thm:rank:prop:fund}
  より, $\rank(AB)=\rank(SB)$である.
  
  一方$S$は階数が$r$の被約階段行列であるので,
  $r+1$行目から$n$行目までの成分はすべて$0$である.
  したがって,
  $SB$を計算すると,
  $SB$の$r+1$行目から$n$行目までの成分もすべて$0$であることがわかる.
  したがって, \cref{lem:rank:rnonzerorows}より,
  $\rank(SB)\leq r$である.
\end{proofof*}

\begin{proofof*}{thm:rank:prop}{regularright}
   $\rank(A)=\rank(AQ)$
  を示すために,
   $\rank(A)\leq\rank(AQ)$
  と
   $\rank(A)\geq\rank(AQ)$
  を示す.
  
  \Cref{thm:rank:prop}の\cref{thm:rank:prop:left}
  より, $\rank(AQ)\leq \rank(A)$である.
  
  一方, 
  $X=AQ$, $Y=Q^{-1}$とすると,
  \cref{thm:rank:prop}の\cref{thm:rank:prop:left}
  より, $\rank(XY)\leq \rank(X)$である.
  $XY=AQQ^{-1}=A$であるので,
  $\rank(A)\leq\rank(AQ)$である.
\end{proofof*}

\begin{lemma}
  \label{lem:ineq:transrank}
  $X$を$(m,n)$-行列とする.
  このとき, $\rank(X)\leq \rank(\transposed{X})$.
\end{lemma}
\begin{proof}
  $X$に対し行基本変形をして階数$r$の被約階段行列$S$が得られたとする.
  このとき, $\rank(X)=r$である.
  $\rank(X)\leq r$を示す.

  $X$に対し行基本変形をして階数$r$の被約階段行列$S$が得られたので,
  \cref{lem:fundamental}
  から,
  $PA=S$をみたす
  正則行列$P$が存在する.
  $\transposed{A}\transposed{P}=\transposed{(PA)}=\transposed{S}$
  である.
  $\transposed{P}$は正則であるので,
  \Cref{thm:rank:prop}の\cref{thm:rank:prop:regularright}より,
  $\rank(\transposed{A})=\rank(\transposed{A}\transposed{P})$である.
  一方
  $S$は階数$r$の被約階段行列であるので,
  $r+1$行目から$m$行目までの成分は$0$である.
  したがって,
  $\transposed{S}$の
  $r+1$列目から$m$列目までの成分は$0$である.
  したがって,
  \Cref{lem:rank:rnonzerocols}より,
  $\transposed{S}\leq r$である.
  まとめると,
  $\rank(\transposed{A})=\rank(\transposed{A}\transposed{P})=\rank(\transposed{S})\leq r$である.
\end{proof}
\begin{proofof*}{thm:rank:prop}{trans}
  $A$を$(m,n)$-行列とする.
  $\rank(A)=\rank(\transposed{A})$を示すために,
  $\rank(A)\leq\rank(\transposed{A})$.
  と
  $\rank(\transposed{A})\leq \rank(A)$.
  を示す.
  
  $X=A$として
  \Cref{lem:ineq:transrank}を使うと,
  $\rank(A)\leq\rank(\transposed{A})$.
  である.

  一方,
  $X=\transposed{A}$として
  \Cref{lem:ineq:transrank}を使うと,
  $\transposed{X}=A$であるので,
  $\rank(\transposed{A})\leq \rank(A)$.
  である.
  
\end{proofof*}
\begin{proofof*}{thm:rank:prop}{right}
  \Cref{thm:rank:prop}の\Cref{thm:rank:prop:trans}より,
  $\rank(AB)=\rank(\transposed{(AB)})$
  であるが, $\transposed{(AB)}=\transposed{B}\transposed{A}$であるので,
  $\rank(AB)=\rank(\transposed{B}\transposed{A})$
  である.
  \Cref{thm:rank:prop}の\Cref{thm:rank:prop:left}
  より$\rank(\transposed{B}\transposed{A})\leq\rank(\transposed{B})$であるので,
  $\rank(AB)\leq\rank(\transposed{B})$.
  \Cref{thm:rank:prop}の\Cref{thm:rank:prop:trans}より,
  $\rank(\transposed{B})=\rank(B)$であるので,
  $\rank(AB)\leq \rank(B)$である.
\end{proofof*}

\begin{proofof*}{thm:rank:prop}{regularleft}
  \Cref{thm:rank:prop}の\Cref{thm:rank:prop:trans}より,
  $\rank(PA)=\rank(\transposed{(PA)})$
  であるが, $\transposed{(PA)}=\transposed{A}\transposed{P}$であるので,
  $\rank(PA)=\rank(\transposed{A}\transposed{P})$
  である.
  $P$は正則であるので$\transposed{P}$も正則である.
  よって, \Cref{thm:rank:prop}の\Cref{thm:rank:prop:regularright}より
  $\rank(\transposed{A}\transposed{P})=\rank(\transposed{A})$
  であるので,
  $\rank(PA)=\rank(\transposed{A})$
  である.
  \Cref{thm:rank:prop}の\Cref{thm:rank:prop:trans}より,
  $\rank(\transposed{A})=\rank(A)$であるので,
  $\rank(PA)= \rank(A)$である.
\end{proofof*}

%\endinput
\section{連立一次方程式に関する命題の証明}
\begin{proofof}{lem:subsol}
$A'=CA$, $\bbb'=C\bbb$とし,
  \begin{align*}
  \FFF=\Set{\vv|A\vv=\bbb},\\
  \FFF'=\Set{\vv|A'\vv=\bbb'}
\end{align*}
  とおく.
  $\vv\in \FFF$とする.
  このとき, $\vv\in\FFF'$を示す.
  つまり, $A'\vv=\bbb'$
  となることを示す.
  今$\vv\in \FFF$であるので,
  $A\vv=\bbb$である.
  両辺に左から$C$をかけることで,
  $CA\vv=C\bbb$を得る.
  したがって,
  $A'\vv=\bbb'$である. 
\end{proofof}

\begin{proofof}{thm:subsol}
  \begin{align*}
  \FFF=\Set{\vv|A\vv=\bbb},\\
  \FFF'=\Set{\vv|A'\vv=\bbb'}
  \end{align*}
  とおく.
$A'=PA$, $\bbb'=P\bbb$であるので,
  \Cref{lem:subsol}
  より$\FFF\subset\FFF'$.
  また,
  $P$は正則であるので,
  $A=P^{-1}A'$,
  $\bbb=P^{-1}\bbb'$であるから,
  \Cref{lem:subsol}
  より$\FFF'\subset\FFF$.
  したがって, $\FFF'=\FFF$.
\end{proofof}
\begin{proofof}{thm:equivequation}
  $\hconcat{A}{\bbb}$に行基本変形をし
  $\hconcat{A'}{\bbb'}$が得られたとすると,
  $P\hconcat{A}{\bbb}=\hconcat{A'}{\bbb'}$.
  を満たす正則行列$P$が存在する.
  このとき,
  $\hconcat{PA}{P\bbb}=\hconcat{A'}{\bbb'}$.
  であるので,
  \begin{align*}
    PA&=A'\\
    P\bbb&=\bbb'
  \end{align*}
  である.
  よって, \Cref{thm:subsol}より,
  $\FFF=\FFF'$
\end{proofof}

\begin{proofof}{thm:ortho:eq}
  \paragraph{\Cref{thm:ortho:eq:item:sol}$\implies$\Cref{thm:ortho:eq:item:ortho}}
  $YA=O_{m,1}$, $A\xx=\bbb$
  とする.
  このとき, $Y\bbb=YA\xx=O_{m,1}\xx=O_{1,1}$.
  \paragraph{\Cref{thm:ortho:eq:item:ortho}$\implies$\Cref{thm:ortho:eq:item:sol}}
  対偶を示す.
  $A\xx=\bbb$が解を持たないとすると,
  \begin{align*}
    P\hconcat{A}{\bbb}=
    \left(
    \begin{array}{c|c}
      S&O_{r,1}\\\hline
      O_{1,n}&1\\\hline
      O_{m-r-1,n}&O_{m-r-1,1}
    \end{array}
    \right)
  \end{align*}
  とできる.
  つまり,
  \begin{align*}
    PA&=\vconcat{S}{O_{m-r,n}}&
    P\bbb&=
    \left(
    \begin{array}{c}
      O_{r,1}\\\hline
      1\\\hline
      O_{m-r-1,1}
    \end{array}
    \right).
  \end{align*}
  このとき,
  $Y'= \left(\begin{array}{c|c|c}O_{1,r}&1&O_{1,n-r-1}\end{array}\right)$
  とおき, $Y=Y'P$とおけば,
  \begin{align*}
    YA&=Y'PA=Y'\vconcat{S}{O_{m-r,n}}=O_{1,n}\\
    Y\bbb&=Y'P\bbb=(1).
  \end{align*}
  これは,
  \begin{align*}
    YA=O_{n,1}\implies
    Y\bbb=O_{1,1}
  \end{align*}
  の反例.
\end{proofof}
\begin{proofof*}{thm:solspace:vec}{sum}
  $\uu,\vv\in\KKK$とすると, $A\uu=\zzero_m$, $A\vv=\zzero_m$である.
  したがって$A(\uu+\vv)=A\uu+A\vv=\zzero_m+\zzero_m=\zzero_m$.
\end{proofof*}
\begin{proofof*}{thm:solspace:vec}{prod}
  $\uu\in\KKK$とすると, $A\uu=\zzero_m$である.
  数$a$に対し,
  $A(a\uu)=aA\uu=a\zzero_m=\zzero_m$.
\end{proofof*}
\begin{proofof}{thm:solspace:withkerspace}
$\FFF'=\Set{\uu+\vv|\uu\in\KKK}$
  とおき,
  $\FFF=\FFF'$を示す.
  \paragraph{$\FFF\subset\FFF'$}
  $\ww\in \FFF$とする.
  このとき,   $\ww\in\FFF'$を示す.
  $\uu=\ww-\vv$とおく.
  このとき,
  \begin{align*}
    A\uu=A(\ww-\vv)=A\ww-A\vv=\bbb-\bbb=\zzero_m/
  \end{align*}
  したがって, $\uu\in\KKK$.
  また, $w=\uu-(-\vv)=\uu+\vv$であるので,
  $\ww\in\FFF'$.

  \paragraph{$\FFF'\subset\FFF$}
  $\uu\in\KKK$とする.
  このとき,
  \begin{align*}
    A(\vv+\uu)=A\vv+A\uu=\bbb+\zzero_m=\bbb.
  \end{align*}
  したがって$\vv+\uu\in\FFF$.
\end{proofof}


\section{行列式に関する命題の証明}
\subsection{ラプラス展開に関すること}
\begin{proofof}{det:tiangular:lower}
  行列のサイズ$n$に関する数学的帰納法で証明する.
  \paragraph{Base case}
  $n=1$のときは,
  $\det(\begin{pmatrix}a_1\end{pmatrix})=a_1$.
    \paragraph{Induction step}
    $n>1$とし, $A$の$(i,j)$-成分を$a_{i,j}$とおく.
    $A$の$1$行目と$k$列目を忘れて得られる$(n-1)$次正方行列を$\pi_{1,k}(A)$とおく.
    $A$の$1$行目は, $1$列目の成分は$a_1$であるが, それ以外は$0$である.
    つまり,
    $a_{1,1}=a_1$,  $a_{1,2}=\cdots=a_{1,n}=0$であるので
    \begin{align*}
      \det(A)&=\sum_{j=1}^{n}
      (-1)^{j+1}a_{1,j}\det(\pi_{1,j}(A))\\
      &=a_{1}\det(\pi_{1,1}(A)).
    \end{align*}
    $\pi_{1,1}(A)$は下半三角行列であり,
    $\pi_{1,1}(A)$の対角成分は$a_2,\ldots,a_n$であるので,
    帰納法の仮定から$\det(\pi_{1,1}(A))=a_2\cdots a_n$である.
    よって, $\det(A)=a_1a_2\cdots a_n$.    
\end{proofof}

\begin{proofof}{det:tiangular:upper}
  行列のサイズ$n$に関する数学的帰納法で証明する.
  \paragraph{Base case}
  $n=1$のときは, $\det(\begin{pmatrix}a_1\end{pmatrix})=a_1$.
    \paragraph{Induction step}
    $n>1$とし, $A$の$(i,j)$-成分を$a_{i,j}$とおく.
    $A$の$1$行目と$k$列目を忘れて得られる$(n-1)$次正方行列を
    $\pi_{1,k}(A)$とおく.
    $A$が上半三角行列であるので,
    $\pi_{1,k}(A)$は上半三角行列である.
    また, $k>1$のとき,
    $\pi_{1,k}(A)$の$(1,1)$-成分は,
    $\pi_{1,1}(A)$の$(2,1)$-成分であり$0$となる.
    したがって,
    $k>1$ならば,
    対角成分に$0$が含まれるので,
    帰納法の仮定から,
    $\det(\pi_{1,k}(A))=0$となる.
    一方
    $\pi_{1,1}(A)$の対角成分は$a_2,\ldots,a_n$であるので,
    帰納法の仮定から,
    $\det(\pi_{1,1}(A))=a_2\cdots a_n$となる.
    よって, $a_{1,1}=a_1$であるから,
    \begin{align*}
      \det(A)&=\sum_{j=1}^{n}
      (-1)^{j+1}a_{1,j}\det(\pi_{1,j}(A))\\
      &=a_{1,1}\det(\pi_{1,1}(A))\\
      &=a_{1}\det(\pi_{1,1}(A))\\
      &=a_{1}a_2\cdots a_n.
    \end{align*}
\end{proofof}


%% \begin{lemma}
%%   \label{lem:det:zero:row:1}
%%   $A$を$n$次正方行列とし,
%%   $A$の$1$行目の成分が全て$0$であるとする.
%%   このとき$\det(A)=0$である.
%% \end{lemma}
%% \begin{proof}
%%   $n=1$のとき,
%%   $\det(A)=\det(\begin{pmatrix}0\end{pmatrix})=0$であるので,
%%     $n>1$のときについて考える.
%%     このとき,
%%     $a_{1,k}=0$であるので,
%%     \begin{align*}
%%       \det(A)&=\sum_{j=1}^{n}a_{1,j}\pi_{1,j}(A)\\
%%       &=\sum_{j=1}^{n}0\pi_{1,j}(A)\\
%%       &=0.
%%     \end{align*}
%% \end{proof}
%% \begin{proofof}{det:zero:row}
%%   行列のサイズ$n$に関する数学的帰納法で証明する.
%%   \paragraph{Base case}
%%   $n=1$のときは,
%%   $t=1$である. つまり$1$行目の成分が$0$であるということであるので,
%%   \cref{lem:det:zero:row:1}より,
%%   $\det(A)=0$.
%%   \paragraph{Induction step}
%%   $n>1$とし,
%%   $A$の$(i,j)$-成分を$a_{i,j}$とおく.

%%   $t=1$のときは, 
%%   \cref{lem:det:zero:row:1}より,
%%   $\det(A)=0$であるので,
%%   $t>1$のときについて考える.
%%   $A$の$1$行目と$k$列目を忘れて得られる$(n-1)$次正方行列を
%%   $\pi_{1,k}(A)$とおく.
%%   このとき,
%%   $A$の$t$行目の成分はすべて$0$であるから,
%%   $\pi_{1,k}(A)$の$t-1$行目の成分はすべて$0$である.
%%   したがって, 帰納法の仮定から,
%%   $\det(\pi_{1,k}(A))=0$である.
%%   よって,
%%   \begin{align*}
%%     \det(A)
%%     &=\sum_{j=1}^{n}
%%     (-1)^{j+1}a_{1,j}\det(\pi_{1,j}(A))\\
%%     &=\sum_{j=1}^{n}
%%     (-1)^{j+1}a_{1,j}0\\
%%     &=0.      
%%   \end{align*}
%% \end{proofof}

%% \begin{proofof}{det:zero:col}
%%   行列のサイズ$n$に関する数学的帰納法で証明する.
%%   \paragraph{Base case}
%%   $n=1$のときは,
%%   $t=1$である. つまり$1$列目の成分が$0$であるということであるので,
%%   $\det(A)=\det(\begin{pmatrix}0\end{pmatrix})=0$.
%%     \paragraph{Induction step}
%%     $n>1$とし,
%%     $A$の$(i,j)$-成分を$a_{i,j}$とおく.

%%     $A$の$1$行目と$k$列目を忘れて得られる$(n-1)$次正方行列を
%%     $\pi_{1,k}(A)$とおく.
%%     $A$の$t$列目の成分がすべて$0$であることから, 以下がわかる:
%%     $k<t$なら,
%%     $\pi_{1,k}(A)$の$t-1$列目の成分はすべて$0$であるので,
%%     帰納法の仮定から,
%%     $\det(\pi_{1,k}(A))=0$.
%%     $k>t$なら,
%%     $\pi_{1,k}(A)$の$t$列目の成分はすべて$0$であるので,
%%     帰納法の仮定から,
%%     $\det(\pi_{1,k}(A))=0$.
%%     $k=t$なら, $a_{1,k}=a_{1,t}=0$.
%%     よって,
%%     \begin{align*}
%%       \det(A)
%%       &=\sum_{j=1}^{n}
%%       (-1)^{j+1}a_{1,j}\det(\pi_{1,j}(A))\\
%%       &=
%%       \sum_{j=1}^{t-1}
%%       (-1)^{j+1}a_{1,j}\det(\pi_{1,j}(A))\\
%%       &\quad +a_{1,t}\det(\pi_{1,j}(A)\\
%%       &\quad +\sum_{j=t+1}^{n}
%%       (-1)^{j+1}a_{1,j}\det(\pi_{1,j}(A))\\
%%       &=
%%       \sum_{j=1}^{t-1}
%%       (-1)^{j+1}a_{1,j}0\\
%%       &\quad +0\det(\pi_{1,j}(A)\\
%%       &\quad +\sum_{j=t+1}^{n}
%%       (-1)^{j+1}a_{1,j}0\\
%%       &=0.      
%%     \end{align*}
%% \end{proofof}

\begin{proofof}{det:col:expansion:1}
  行列のサイズ$n$に関する数学的帰納法で証明する.
  \paragraph{Base case}
  $n=1$のときは$\det(A)=a_{1,1}$であるので成り立つ.
  $n=2$のときは$\det(A)=a_{1,1}a_{2,2}-a_{1,2}a_{2,1}=a_{1,1}a_{2,2}-a_{2,1}a_{1,2}$であるので成り立つ.

  \paragraph{Induction step}
  $n>2$とする.
  $\pi_{\Set{1,i},\Set{1,j}}(A)$で, $A$の1行目, $i$行目, $1$列目, $j$列目を
  取り除いて得られる$(n-2)$-次正方行列を表すことにする.
  \begin{align*}
    D&=\sum_{i=1}^{n}
    (-1)^{i+1}a_{i,1}\det(\pi_{i,1}(A))\\
    &=
    a_{1,1}\det(\pi_{1,1}(A))
    +
    \sum_{i=2}^{n}
    (-1)^{i+1}a_{i,1}\det(\pi_{i,1}(A))
  \end{align*}
  とおく.
  $i>1$のとき,
  $\pi_{i,1}(A)$が$A$の$i$行目と$1$列目を取り除いて得られる行列であるので,
  $\pi_{i,1}(A)$の$(1,j)$成分は$a_{1,j+1}$であり,
  $\pi_{1,j}(\pi_{i,1}(A))=\pi_{\Set{1,i},\Set{1,j+1}}(A)$である.
  したがって, $i>1$のときには,
  $\det(\pi_{i,1}(A))$を定義に従って展開すると,
  \begin{align*}
    \det(\pi_{i,1}(A)))
    &=\sum_{j=1}^{n-1}
    (-1)^{j+1}a_{1,j+1}\det(\pi_{1,j}(\pi_{i,1}(A)))\\
    &=\sum_{j=1}^{n-1}
    (-1)^{j+1}a_{1,j+1}\det(\pi_{\Set{1,i},\Set{1,j+1}}(A))\\
    &=\sum_{j=2}^{n}
    (-1)^{j}a_{1,j}\det(\pi_{\Set{1,i},\Set{1,j}}(A)).
  \end{align*}
  したがって,
  \begin{align*}
    D
    &=
    a_{1,1}\det(\pi_{1,1}(A))
    +
    \sum_{i=2}^{n}
    (-1)^{i+1}a_{i,1}\det(\pi_{i,1}(A))\\
    &=
    a_{1,1}\det(\pi_{1,1}(A))
    +
    \sum_{i=2}^{n}
    (-1)^{i+1}a_{i,1}\sum_{j=2}^{n}
    (-1)^{j}a_{1,j}\det(\pi_{\Set{1,i},\Set{1,j}}(A))
  \\
  &=
  a_{1,1}\det(\pi_{1,1}(A))
  +
    \sum_{i=2}^{n}\sum_{j=2}^{n}
    (-1)^{i+j+1}
    a_{i,1}a_{1,j}\det(\pi_{\Set{1,i},\Set{1,j}}(A)).
  \end{align*}
  
  一方$\det(A)$を定義に従って展開すると
  \begin{align*}
    \det(A)
    &=\sum_{j=1}^{n}
      (-1)^{j+1}a_{1,j}\det(\pi_{1,j}(A))\\
    &=a_{1,1}\det(\pi_{1,1}(A))
    +\sum_{j=2}^{n}(-1)^{j+1}a_{1,j}\det(\pi_{1,j}(A)).
  \end{align*}
  である.
  $j>1$のとき,
  $\pi_{1,j}(A)$が$A$の$1$行目と$j$列目を取り除いて得られる行列であるので,
  $\pi_{1,j}(A)$の$(i,1)$成分は$a_{i+1,1}$であり,
  $\pi_{i,1}(\pi_{1,j}(A))=\pi_{\Set{1,i+1},\Set{1,j}}(A)$である.
  したがって, $j>1$のときには,
  帰納法の仮定を使い$\det(\pi_{1,j}(A))$を展開すると,
  \begin{align*}
    \det(\pi_{1,j}(A))
    &=
    \sum_{i=1}^{n-1}
    (-1)^{i+1}a_{i+1,1}\det(\pi_{i,1}(\pi_{1,j}(A)))\\
    &=
    \sum_{i=1}^{n-1}
    (-1)^{i+1}a_{i+1,1}\det(\pi_{\Set{1,i+1},\Set{1,j}}(A))\\
    &=
    \sum_{i=2}^{n}
    (-1)^{i}a_{i,1}\det(\pi_{\Set{1,i},\Set{1,j}}(A)).
  \end{align*}
  したがって,
  \begin{align*}
    \det(A)
    &=
    a_{1,1}\det(\pi_{1,1}(A))
    +\sum_{j=2}^{n}(-1)^{j+1}a_{1,j}\det(\pi_{1,j}(A))\\
    &=
    a_{1,1}\det(\pi_{1,1}(A))
    +\sum_{j=2}^{n}(-1)^{j+1}a_{1,j}\sum_{i=2}^{n}
    (-1)^{i}a_{i,1}\det(\pi_{\Set{1,i},\Set{1,j}}(A))
  \\
    &=
    a_{1,1}\det(\pi_{1,1}(A))
    +\sum_{j=2}^{n}\sum_{i=2}^{n}
    (-1)^{i+j+1}
    a_{i,1}a_{1,j}\det(\pi_{\Set{1,i},\Set{1,j}}(A)).
  \end{align*}
  したがって, $\det(A)=D$である.
\end{proofof}

\begin{proofof}{det:col:expansion}
  行列のサイズ$n$に関する数学的帰納法で証明する.
  \paragraph{Base case}
  $n=1$のときは$\det(A)=a_{1,1}$であるので成り立つ.
  $n=2$のときは$\det(A)=a_{1,1}a_{2,2}-a_{1,2}a_{2,1}=a_{1,1}a_{2,2}-a_{2,1}a_{1,2}$であるので成り立つ.

 \paragraph{Induction step}
  $n>2$とする.
  $\pi_{\Set{1,i},\Set{k,j}}(A)$で, $A$の1行目, $i$行目, $k$列目, $j$列目を
  取り除いて得られる$(n-2)$-次正方行列を表すことにする.
  \begin{align*}
    D&=\sum_{i=1}^{n}
    (-1)^{i+t}a_{i,t}\det(\pi_{i,t}(A))\\
    &=
    (-1)^{1+t}a_{1,t}\det(\pi_{1,t}(A))
    +
    \sum_{i=2}^{n}
    (-1)^{i+t}a_{i,t}
    \det(\pi_{i,t}(A))
  \end{align*}
  とおく.
  $i>1$のとき,
  $\pi_{i,t}(A)$が$A$の$i$行目と$t$列目を取り除いて得られる行列である.
  したがって,
  $j<t$なら,
  $\pi_{i,t}(A)$の$(1,j)$成分は
  $a_{1,j}$であり,
  $\pi_{1,j}(\pi_{i,t}(A))=\pi_{\Set{1,i},\Set{t,j}}(A)$である.
  また,
  $j\geq t$なら,
  $\pi_{i,t}(A)$の$(1,j)$成分は
  $a_{1,j+1}$であり,
  $\pi_{1,j}(\pi_{i,t}(A))=\pi_{\Set{1,i},\Set{t,j+1}}(A)$である.
  したがって, 
  $\det(\pi_{i,t}(A))$を定義に従って展開すると,
  \begin{align*}
    \det(\pi_{i,t}(A)))
    &=
    \sum_{j=1}^{t-1}
    (-1)^{j+1}a_{1,j}\det(\pi_{1,j}(\pi_{i,t}(A)))
    +
    \sum_{j=t}^{n-1}
    (-1)^{j+1}a_{1,j+1}\det(\pi_{1,j}(\pi_{i,t}(A)))
    \\
    &=
    \sum_{j=1}^{t-1}
    (-1)^{j+1}a_{1,j}\det(\pi_{\Set{1,i},\Set{t,j}}(A))
    +
    \sum_{j=t}^{n-1}
    (-1)^{j+1}a_{1,j+1}\det(\pi_{\Set{1,i},\Set{t,j+1}}(A))
    \\
    &=
    \sum_{j=1}^{t-1}
    (-1)^{j+1}a_{1,j}\det(\pi_{\Set{1,i},\Set{t,j}}(A))
    +
    \sum_{j=t+1}^{n}
    (-1)^{j}a_{1,j}\det(\pi_{\Set{1,i},\Set{t,j}}(A)).
  \end{align*}
  したがって,
  \begin{align*}
    D
    &=
    (-1)^{1+t}a_{1,t}\det(\pi_{1,t}(A))
    +
    \sum_{i=2}^{n}
    (-1)^{i+t}a_{i,t}
    \sum_{j=1}^{t-1}
    (-1)^{j+1}a_{1,j}\det(\pi_{\Set{1,i},\Set{t,j}}(A))
    +
    \sum_{i=2}^{n}
    (-1)^{i+t}a_{i,t}
    \sum_{j=t+1}^{n}
    (-1)^{j}a_{1,j}\det(\pi_{\Set{1,i},\Set{t,j}}(A))\\
    &=
    (-1)^{1+t}a_{1,t}\det(\pi_{1,t}(A))
    +
    \sum_{i=2}^{n}
    \sum_{j=1}^{t-1}
    (-1)^{i+j+t+1}a_{i,t}a_{1,j}\det(\pi_{\Set{1,i},\Set{t,j}}(A))
    +
    \sum_{i=2}^{n}
    \sum_{j=t+1}^{n}
    (-1)^{i+j+t}a_{i,t}a_{1,j}\det(\pi_{\Set{1,i},\Set{t,j}}(A)).
  \end{align*}
  
  一方$\det(A)$を定義に従って展開すると
  \begin{align*}
    \det(A)
    &=\sum_{j=1}^{n}
      (-1)^{j+1}a_{1,j}\det(\pi_{1,j}(A))\\
    &=
    (-1)^{1+t}a_{1,t}\det(\pi_{1,t}(A))
    +\sum_{j=1}^{t-1}(-1)^{j+1}a_{1,j}\det(\pi_{1,j}(A))
    +\sum_{j=t+1}^{n}(-1)^{j+1}a_{1,j}\det(\pi_{1,j}(A)).
  \end{align*}
  である.
  $j<t$のとき,
  $\pi_{1,j}(A)$が$A$の$1$行目と$j$列目を取り除いて得られる行列であるので,
  $\pi_{1,j}(A)$の$(i,t-1)$成分は$a_{i+1,t}$であり,
  $\pi_{i,t-1}(\pi_{1,j}(A))=\pi_{\Set{1,i+1},\Set{j,t}}(A)$である.
  したがって, $j<t$のときには,
  帰納法の仮定を使い$\det(\pi_{1,j}(A))$を$t-1$列目で展開すると,
  \begin{align*}
    \det(\pi_{1,j}(A))
    &=
    \sum_{i=1}^{n-1}
    (-1)^{i+t-1}a_{i+1,t}\det(\pi_{i,t-1}(\pi_{1,j}(A)))\\
    &=
    \sum_{i=1}^{n-1}
    (-1)^{i+t-1}a_{i+1,t}\det(\pi_{\Set{1,i+1},\Set{j,t}}(A))\\
    &=
    \sum_{i=2}^{n}
    (-1)^{i+t-2}a_{i,t}\det(\pi_{\Set{1,i},\Set{j,t}}(A))\\
    &=
    \sum_{i=2}^{n}
    (-1)^{i+t}a_{i,t}\det(\pi_{\Set{1,i},\Set{j,t}}(A)).
  \end{align*}
  また,
  $j>t$のとき,
  $\pi_{1,j}(A)$が$A$の$1$行目と$j$列目を取り除いて得られる行列であるので,
  $\pi_{1,j}(A)$の$(i,t)$成分は$a_{i+1,t}$であり,
  $\pi_{i,t}(\pi_{1,j}(A))=\pi_{\Set{1,i+1},\Set{j,t}}(A)$である.
  したがって, $j<t$のときには,
  帰納法の仮定を使い$\det(\pi_{1,j}(A))$を$t$列目で展開すると,
  \begin{align*}
    \det(\pi_{1,j}(A))
    &=
    \sum_{i=1}^{n-1}
    (-1)^{i+t}a_{i+1,t}\det(\pi_{i,t}(\pi_{1,j}(A)))\\
    &=
    \sum_{i=1}^{n-1}
    (-1)^{i+t}a_{i+1,t}\det(\pi_{\Set{1,i+1},\Set{j,t}}(A))\\
    &=
    \sum_{i=2}^{n}
    (-1)^{i+t-1}a_{i,t}\det(\pi_{\Set{1,i},\Set{j,t}}(A)).
  \end{align*}  
  したがって,
  \begin{align*}
    \det(A)
    &=
    (-1)^{1+t}a_{1,t}\det(\pi_{1,t}(A))
    +\sum_{j=1}^{t-1}(-1)^{j+1}a_{1,j}
    \sum_{i=2}^{n}
    (-1)^{i+t}a_{i,t}\det(\pi_{\Set{1,i},\Set{j,t}}(A))
    +\sum_{j=t+1}^{n}(-1)^{j+1}a_{1,j}
    \sum_{i=2}^{n}
    (-1)^{i+t-1}a_{i,t}\det(\pi_{\Set{1,i},\Set{j,t}}(A))\\
    &=
    (-1)^{1+t}a_{1,t}\det(\pi_{1,t}(A))
    +
    \sum_{i=2}^{n}
    \sum_{j=1}^{t-1}
    (-1)^{i+t+j+1}a_{1,j}a_{i,t}\det(\pi_{\Set{1,i},\Set{j,t}}(A))    
    +
    \sum_{i=2}^{n}
    \sum_{j=t+1}^{n}
    (-1)^{i+j+t}a_{1,j}
    a_{i,t}\det(\pi_{\Set{1,i},\Set{j,t}}(A))\\
    &=
    (-1)^{1+t}a_{1,t}\det(\pi_{1,t}(A))
    +
    \sum_{j=1}^{t-1}
    \sum_{i=2}^{n}
    (-1)^{i+t+j+1}a_{1,j}a_{i,t}\det(\pi_{\Set{1,i},\Set{j,t}}(A))    
    +
    \sum_{j=t+1}^{n}
    \sum_{i=2}^{n}
    (-1)^{i+j+t}a_{1,j}
    a_{i,t}\det(\pi_{\Set{1,i},\Set{j,t}}(A)).
  \end{align*}
  したがって, $\det(A)=D$である.
\end{proofof}

\begin{proofof}{det:row:expansion}
\Cref{thm:det:col:expansion,thm:det:transpose}
から得られる.
\end{proofof}

\subsection{順列を用いた展開に関すること}
\begin{proofof}{det:closedformula}
  $\det(A)$を計算するため,
  定義に従って展開することを繰り返すと,
  各行各列から1つずつ成分を選んできてかけ合わせたものが項として現れる.
  つまり,
  \begin{align*}
    \det(A)&=
    \sum_{\sigma=[\sigma(1),\sigma(2),\ldots,\sigma(n)]\in S_n}
    c_{\sigma}
    a_{1,\sigma(1)}a_{2,\sigma(2)}\cdots a_{n,\sigma(n)}
  \end{align*}
  と書くことができる.
  $c_{\sigma}$を計算するには,
  $a_{i,\sigma(i)}$
  が選ばれたときの符号を考える必要がある.
  $A$の$1$行目, $2$行目,\ldots, $i-1$行目と,
  $\sigma(1)$列目, $\sigma(2)$列目,\ldots,$\sigma(i-1)$列目,
  を取り除いて得られる$(n-i+1)$-次正方行列
  を$A'$とすると,
  $a_{i,\sigma(i)}$は$A'$から選ばれている.
  $\sigma(1),\sigma(2),\ldots,(\sigma(i-1))$のうち$\sigma(i)$よりも
  小さいものの総数を$k_i$とすると,
  $a_{i,\sigma(i)}$は$A'$の$\sigma(i)-k_i$列目であるから,
  このときの符号は$(-1)^{\sigma(i)-k_i+1}$である.
  したがって, $c_\sigma$は
  \begin{align*}
    c_\sigma &= (-1)^{\sum_{i=1}^n(\sigma(i)+1-k_i)}\\
    &= (-1)^{\frac{n(n+1)}{2}+n-\sum_{i=1}^n k_i}\\
    &= (-1)^{2n}(-1)^{\frac{n(n+1)}{2}-n-\sum_{i=1}^n k_i}\\
    &= (-1)^{\frac{n(n+1)}{2}-n-\sum_{i=1}^n k_i}\\
    &= (-1)^{\frac{n(n-1)}{2}-\sum_{i=1}^n k_i}
  \end{align*}
  である. 
  \begin{align*}
    \#\Set{(i,j)|1\leq i<j\leq n}&=\binom{n}{2}=\frac{n(n-1)}{2},\\
    \#\Set{(i,j)|1\leq i<j\leq n, \sigma(i)<\sigma(j)}&=\sum_{i=1}^n k_i
  \end{align*}
  であるので,
  $\#I(\sigma)=\frac{n(n-1)}{2}-\sum_{i=1}^n k_i$であるから,
  $c_\sigma=\sgn(\sigma)$.


  記号を用意し詳細に書くなら以下のようになる:
  

  定義から,
  \begin{align*}
    \det(A)
    &=\sum_{j=1}^{n}
    (-1)^{j+1}a_{1,j}\det(\pi_{1,j}(A))\\
    &=\sum_{j\in\Set{1,2,\ldots,n}}
    (-1)^{j+1}a_{1,j}\det(\pi_{1,j}(A))
  \end{align*}
  である. 
  
  $\sum_{j=1}^{n}(-1)^{j+1}a_{1,j}\det(\pi_{1,j}(A))$ 
  という和の第$j$項目$(-1)^{j+1}a_{1,j}\det(\pi_{1,j}(A))$に注目する.
  これを計算するには,
  $\det(\pi_{1,j}(A))$を定義に従い展開することになる.
  $\pi_{1,j}(A)$は$A$の$1$行目と$j$列目を取り除いた行列であるので,
  $\pi_{1,j}(A)$の$1$行目と$j'$列目を取り除いた行列は,
  $j'<j$なら$\pi_{\Set{1,2},\Set{j,j'}}(A)$であり,
  $j'\geq j$なら$\pi_{\Set{1,2},\Set{j,j'+1}}(A)$である.
  また,
  $\pi_{1,j}(A)$の$1$行目は$a_{2,1},a_{2,2},\ldots,a_{2,j-1},a_{2,j+1},a_{2,j+12},\ldots,a_{2,n}$という$A$の2行目から$j$列目を除いたものである.
  したがって,
  \begin{align*}
    \det(\pi_{1,j}(A))
    &=
    \sum_{j'=1}^{j-1}(-1)^{j'+1}a_{2,j'}\det(\pi_{\Set{1,2},\Set{j,j'}}(A))
    \\&\quad+\sum_{j'=j}^{n-1}(-1)^{j'+1}a_{2,j'+1}\det(\pi_{\Set{1,2},\Set{j,j'+1}}(A))\\
    &=
    \sum_{j'=1}^{j-1}(-1)^{j'+1}a_{2,j'}\det(\pi_{\Set{1,2},\Set{j,j'}}(A))
    \\&\quad+\sum_{j'=j+1}^{n}(-1)^{j'}a_{2,j'}\det(\pi_{\Set{1,2},\Set{j,j'}}(A))\\
    &=
    \sum_{j'=1}^{j-1}(-1)^{j'+1}a_{2,j'}\det(\pi_{\Set{1,2},\Set{j,j'}}(A))
    \\&\quad+\sum_{j'=j+1}^{n}(-1)^{j'}a_{2,j'}\det(\pi_{\Set{1,2},\Set{j,j'}}(A))
  \end{align*}
  とかける. これは,
  \begin{align*}
    &\det(\pi_{1,j}(A))\\
    &=
    \sum_{j'=1}^{j-1}(-1)^{j'+1}a_{2,j'}\det(\pi_{\Set{1,2},\Set{j,j'}}(A))
    +\sum_{j'=j+1}^{n}(-1)^{j'}a_{2,j'}\det(\pi_{\Set{1,2},\Set{j,j'}}(A))\\
    &=
    \sum_{j'\in \Set{1,\ldots,n}\setminus\Set{j}} c_{j'} a_{2,j'}\det(\pi_{\Set{1,2},\Set{j,j'}}(A))
  \end{align*}
  と整理できる.\footnote{集合$X$と$Y$に対し,
  $X\setminus Y$で$X$の要素で$Y$には含まれないものを集めた集合を表している.}
  つまり, $I=\Set{1,2,\ldots,n}$とおくと,
  \begin{align*}
    &\det(\pi_{1,j}(A))\\
    &=
    \sum_{j'\in I\setminus\Set{j}} c_{j'} a_{2,j'}\det(\pi_{\Set{1,2},\Set{j,j'}}(A))
  \end{align*}
  とかける.
  

  
  このことから, $\det(A)$は, 
  \begin{align*}
    \det(A)&=
    \sum_{j\in I}
    \sum_{j'\in I\setminus\Set{j}}
    c_{j,j'} a_{1,j}a_{2,j'}\det(\pi_{\Set{1,2},\Set{j,j'}}(A))
  \end{align*}
  と書くことができることがわかるが,
  これは
  \begin{align*}
    \det(A)&=
    \sum_{j,j'\in\Set{1,2,\ldots,n}\colon \text{$j,j'$は異なる}}
    c_{j,j'} a_{1,j}a_{2,j'}\det(\pi_{\Set{1,2},\Set{j,j'}}(A))
  \end{align*}
  と書いてもよい.
  $j$, $j'$を, それぞれ$\sigma(1)$, $\sigma(2)$と書き直して,
  \begin{align*}
    \det(A)&=
    \sum_{\sigma(1),\sigma(2)\in I\colon \text{$\sigma(1),\sigma(2)$は異なる}}
    c_{\sigma(1),\sigma(2)} a_{1,\sigma(1)}a_{2,\sigma(2)}\det(\pi_{\Set{1,2},\Set{\sigma(1),\sigma(2)}}(A))
  \end{align*}
  と書ける.
  $\det(\pi_{\Set{1,2},\Set{\sigma(1),\sigma(2)}}(A))$にも同様の考察をすることで,
  \begin{align*}
    &\det(A)=\\
    &\sum_{\sigma(1),\sigma(2), \sigma(3) \in I \colon\text{相異なる}}
    c_{\sigma(1),\sigma(2),\sigma(3)} a_{1,\sigma(1)}a_{2,\sigma(2)}a_{2,\sigma(3)}\det(\pi_{\Set{1,2,3},\Set{\sigma(1),\sigma(2),\sigma(3)}}(A))
  \end{align*}
  と書ける.  以下同様に繰り返すと,
  \begin{align*}
    \det(A)&=
    \sum_{\sigma(1),\sigma(2), \ldots,\sigma(n) \in I \colon\text{相異なる}}
    c_{\sigma}
    a_{1,\sigma(1)}a_{2,\sigma(2)}\cdots a_{n,\sigma(n)}\\
    &=
    \sum_{\sigma=[\sigma(1),\sigma(2),\ldots,\sigma(n)]\in S_n}
    c_{\sigma}
    a_{1,\sigma(1)}a_{2,\sigma(2)}\cdots a_{n,\sigma(n)}
  \end{align*}
  と書くことができることがわかる.
  
  
  $c_\sigma$が$\sgn(\sigma)$と等しいことを示す.
  $c_\sigma$, つまり
  $a_{1,\sigma(1)}a_{2,\sigma(2)}\cdots a_{n,\sigma(n)}$
  の係数について考える.

  $I(\sigma,i)=\Set{(t,i)| \text{$t<i$かつ$\sigma(t)>\sigma(i)$}}$
  とおくことにする.
  $I(\sigma,i)$に属する要素の総数$\#I(\sigma,i)$は,
  $\sigma(1),\sigma(2),\ldots,\sigma(i-1)$のうち
  $\sigma(i)$よりも大きいものの総数である.
  したがって,
  $\sigma(1),\sigma(2),\ldots,\sigma(i-1)$のうち
  $\sigma(i)$より小さいものの総数は,
  $i-\#I(\sigma,i)-1$である.
  また, 定義から, $\#I(\sigma)=\sum_{i=1}^n\#I(\sigma,i)$
  である.

  行列式の
  最初の展開で
  $a_{1,\sigma(1)}$が選ばれたときの係数は,
  $(-1)^{\sigma(1)+1}$である.
  
  次の展開で$a_{2,\sigma(2)}$で選ばれたときの係数は,
  $\pi_{\Set{1},\Set{\sigma(1)}}$を展開しているので,
  $\sigma(2)<\sigma(1)$ならば$(-1)^{\sigma(2)+1}$であるが,
  $\sigma(1)<\sigma(2)$ならば$(-1)^{\sigma(2)}$である.

  さらに次の展開で$a_{3,\sigma(3)}$で選ばれたときの係数は,
  $\pi_{\Set{1,2},\Set{\sigma(1),\sigma(2)}}$を展開しているので,
  $\sigma(1),\sigma(2)$のうち$\sigma(3)$よりも小さいものの総数を$l$とすると,
  $l=0$のとき, $(-1)^{\sigma(3)+1}$であり,
  $l=1$のとき, $(-1)^{\sigma(3)}$であり,
  $l=2$のとき, $(-1)^{\sigma(3)-1}$である.

  一般に, $a_{i,\sigma(i)}$が選ばれたときの係数は,
  $(-1)^{\sigma(i)+1-(i-\#I(\sigma,i)-1)}$
  であることがわかる.
  \begin{align*}
    (-1)^{\sigma(i)+1-(i-\#I(\sigma,i)-1)}
    =(-1)^{2+\sigma(i)-i+\#I(\sigma,i)}
    =(-1)^{\sigma(i)-i+\#I(\sigma,i)}
  \end{align*}
  である.
  これらをかけ合わせたものが,
  $a_{1,\sigma(1)}a_{2,\sigma(2)}\cdots a_{n,\sigma(n)}$
  の係数であるから,
  \begin{align*}
    c_\sigma
    &=\prod_{i=1}^n
    (-1)^{\sigma(i)-i+\#I(\sigma,i)}\\
    &=
    (-1)^{\sum_{i=1}^n(\sigma(i)-i+\#I(\sigma,i))}\\
    &=
    (-1)^{\sum_{i=1}^n\sigma(i)-\sum_{i=1}^n i+\sum_{i=1}^n \#I(\sigma,i))}\\
    &=
    (-1)^{\sum_{i=1}^n\sigma(i)-\sum_{i=1}^n i+\sum_{i=1}^n \#I(\sigma,i))}.
  \end{align*}
  $\sigma(1),\ldots,\sigma(n)$は$1,\ldots,n$を並び替えたものなので,
  総和は等しい. したがって,
  \begin{align*}
    c_\sigma&=(-1)^{\sum_{i=1}^n\sigma(i)-\sum_{i=1}^n i+\sum_{i=1}^n \#I(\sigma,i))}\\
    &=
    (-1)^{\sum_{i=1}^n \#I(\sigma,i))}\\
    &=
    (-1)^{ \#I(\sigma))}\\
    &=\sgn(\sigma).
  \end{align*}
\end{proofof}

\subsection{行基本変形に関すること}
\begin{proofof*}{det:rowtrans}{F}
  行列のサイズ$n$に関する帰納法で示す.
  \paragraph{Base case}
  $n=1$のときは,
  $t=1$である.
  したがって,
  $\det(\begin{pmatrix}\alpha a\end{pmatrix})=\alpha a=\alpha \det(\begin{pmatrix}a\end{pmatrix})$.
  \paragraph{Induction step}
  $n>1$のときについて考える.
  $A$の$(i,j)$-成分を$a_{i,j}$,
  $A'$の$(i,j)$-成分を$a'_{i,j}$
  とする.
  $k\neq t$とする.
  このとき,
  $\pi_{k,j}(A')$には$A'$の$t$行目が含まれるので,
  帰納法の仮定から,
  $\det(\pi_{k,j}(A'))=\alpha\det(\pi_{k,j}(A))$
  である.
  また, $a'_{k,j}=a_{k,j}$であるから,
  $\det(A')$を$k$行目で展開すると,
  \begin{align*}
    \det(A')
    &=\sum_{j=1}^{n} (-1)^{k+j} a'_{k,j}\det(\pi_{k,j}(A'))\\
    &=\sum_{j=1}^{n} (-1)^{k+j} \alpha a_{k,j}\det(\pi_{k,j}(A)).
  \end{align*}
  一方, $\det(A)$を$k$行目で展開すると,
  \begin{align*}
    \alpha \det(A)
    &=\alpha \sum_{j=1}^{n} (-1)^{k+j} a_{k,j}\det(\pi_{k,j}(A))\\
    &=\sum_{j=1}^{n} (-1)^{k+j} \alpha a_{k,j}\det(\pi_{k,j}(A)).
  \end{align*}  
    
%%   $A$の$(i,j)$-成分を$a_{i,j}$,
%%   $A$の$(i,j)$-成分を$a'_{i,j}$
%%   とする.
%%   このとき,
%%   $i\neq t$なら$a_{i,j}=a'_{i,j}$であり,
%%   $\pi_{t,j}(A')=\pi_{t,j}(A)$である.
%%   また$\alpha a_{i,j}=a'_{i,j}$であるので,
%%   $\det(A')$を$t$行目で展開すると,
%%   \begin{align*}
%%     \det(A')
%%     &=\sum_{j=1}^{n} (-1)^{t+j} a'_{t,j}\det(\pi_{t,j}(A'))\\
%%     &=\sum_{j=1}^{n} (-1)^{t+j} \alpha a_{t,j}\det(\pi_{t,j}(A)).
%%   \end{align*}
%%   一方, $\det(A)$を$t$行目で展開すると,
%%   \begin{align*}
%%     \alpha \det(A)
%%     &=\alpha \sum_{j=1}^{n} (-1)^{t+j} a_{t,j}\det(\pi_{t,j}(A))\\
%%     &=\sum_{j=1}^{n} (-1)^{t+j} \alpha a_{t,j}\det(\pi_{t,j}(A)).
%%   \end{align*}  
\end{proofof*}

\begin{proofof*}{det:rowtrans}{G}
  行列のサイズ$n$に関する帰納法で示す.
  \paragraph{Base case}
  $n=2$のときについて考える.
  \begin{align*}
    \det(
    \begin{pmatrix}
      a&b\\
      c&d
    \end{pmatrix}
    )==ad-bc\\
    \det(
    \begin{pmatrix}
      a+\alpha c& b+\alpha d\\
      c&d
    \end{pmatrix}
    )=(a+\alpha c)d-(b+\alpha d)c=ad-bc\\
    \det(
    \begin{pmatrix}
      a& b\\
      c+\alpha a&d+\alpha b
    \end{pmatrix}
    )=a(d+\alpha b)-b(c+\alpha a)=ad-bc.
  \end{align*}
  \paragraph{Induction step}
  $n>2$のときについて考える.
  $A$の$(i,j)$-成分を$a_{i,j}$,
  $A'$の$(i,j)$-成分を$a'_{i,j}$
  とする.
  $k$は$t$でも$s$でもないとする.
  このとき,
  $\pi_{k,j}(A')$には$A'$の$t$行目と$s$行目が含まれるので,
  帰納法の仮定から,
  $\det(\pi_{k,j}(A'))=\det(\pi_{k,j}(A))$
  である.
  また, $a'_{k,j}=a_{k,j}$であるから,
  $\det(A')$を$k$行目で展開すると,
  \begin{align*}
    \det(A')
    &=\sum_{j=1}^{n} (-1)^{k+j} a'_{k,j}\det(\pi_{k,j}(A'))\\
    &=\sum_{j=1}^{n} (-1)^{k+j} a_{k,j}\det(\pi_{k,j}(A)).
  \end{align*}
  一方, $\det(A)$を$k$行目で展開すると,
  \begin{align*}
    \det(A)
    &=\sum_{j=1}^{n} (-1)^{k+j} a_{k,j}\det(\pi_{k,j}(A)).
  \end{align*}  
\end{proofof*}


\begin{proofof*}{det:rowtrans}{H}
  行列のサイズ$n$に関する帰納法で示す.
  \paragraph{Base case}
  $n=2$のときについて考える.
  \begin{align*}
    \det(
    \begin{pmatrix}
      a&b\\
      c&d
    \end{pmatrix}
    )=ad-bc\\
    -\det(
    \begin{pmatrix}
      c&d\\
      a&b
    \end{pmatrix}
    )=-(cb-da)=ad-bc.
  \end{align*}
  \paragraph{Induction step}
  $n>2$のときについて考える.
  $A$の$(i,j)$-成分を$a_{i,j}$,
  $A'$の$(i,j)$-成分を$a'_{i,j}$
  とする.
  $k$は$t$でも$s$でもないとする.
  このとき,
  $\pi_{k,j}(A')$には$A'$の$t$行目と$s$行目が含まれるので,
  帰納法の仮定から,
  $\det(\pi_{k,j}(A'))=-\det(\pi_{k,j}(A))$
  である.
  また, $a'_{k,j}=a_{k,j}$であるから,
  $\det(A')$を$k$行目で展開すると,
  \begin{align*}
    \det(A')
    &=\sum_{j=1}^{n} (-1)^{k+j} a'_{k,j}\det(\pi_{k,j}(A'))\\
    &=\sum_{j=1}^{n} -(-1)^{k+j} a_{t,j}\det(\pi_{k,j}(A)).
  \end{align*}
  一方, $-\det(A)$を$k$行目で展開すると,
  \begin{align*}
    -\det(A)
    &=-\sum_{j=1}^{n} (-1)^{k+j} a_{k,j}\det(\pi_{k,j}(A))\\
    &=\sum_{j=1}^{n} -(-1)^{k+j} a_{k,j}\det(\pi_{k,j}(A)).
  \end{align*}  
\end{proofof*}


\begin{proofof*}{lem:det:hom:reg}{nonzero}
  $P$は正則行列なので, $P=P_1P_2P_3\cdot P_l$を満たす基本行列$P_i$が取れる.
  このとき
  \begin{align*}
    \det(P)
    &=\det(P_1P_2P_3\cdot P_l)\\
    &=\det(P_1)\det(P_2P_3\cdot P_l)\\
    &=\det(P_1)\det(P_2)\det(P_3\cdot P_l)\\
    &=\det(P_1)\det(P_2)\det(P_3)\cdot \det(P_l)
  \end{align*}
  となるが, 基本行列行列式は$0$ではないので, その積も$0$ではない.
\end{proofof*}

\begin{proofof*}{lem:det:hom:reg}{prod}
  $P$は正則行列なので, $P=P_1P_2P_3\cdot P_l$を満たす基本行列$P_i$が取れる.
  このとき
  \begin{align*}
    \det(P)
    &=\det(P_1P_2P_3\cdot P_l)\\
    &=\det(P_1)\det(P_2P_3\cdot P_l)\\
    &=\det(P_1)\det(P_2)\det(P_3\cdot P_l)\\
    &=\det(P_1)\det(P_2)\det(P_3)\cdot \det(P_l)
  \end{align*}
  である.  したがって
  \begin{align*}
    \det(PA)
    &=\det(P_1P_2P_3\cdot P_l A)\\
    &=\det(P_1)\det(P_2P_3\cdot P_lA)\\
    &=\det(P_1)\det(P_2)\det(P_3\cdot P_l A)\\
    &=\det(P_1)\det(P_2)\det(P_3)\cdot \det(P_l A)\\
    &=\det(P_1)\det(P_2)\det(P_3)\cdot \det(P_l)\det(A)\\
    &=\det(P)\det(A).
  \end{align*}
\end{proofof*}

\subsection{多重線形性, 交代性に関すること}

\begin{proofof*}{det:multilinear:row}{scalar}
  行列のサイズ$n$に関する帰納法で示す.
  \paragraph{Base case}
  $n=1$のときは,
  $t=1$である.
  したがって,
  $\det(\begin{pmatrix}\alpha a\end{pmatrix})=\alpha a=\alpha \det(\begin{pmatrix}a\end{pmatrix})$.
  \paragraph{Induction step}
  $n>1$のときについて考える.
  $A$の$(i,j)$-成分を$a_{i,j}$,
  $A'$の$(i,j)$-成分を$a'_{i,j}$
  とする.
  $k\neq t$とする.
  このとき,
  $\pi_{k,j}(A')$には$A'$の$t$行目が含まれるので,
  帰納法の仮定から,
  $\alpha\det(\pi_{k,j}(A'))=\det(\pi_{k,j}(A))$
  である.
  また, $a'_{k,j}=a_{k,j}$であるから,
  $\det(A')$を$k$行目で展開すると,
  \begin{align*}
    \alpha\det(A')
    &=\alpha\sum_{j=1}^{n} (-1)^{k+j} a'_{k,j}\det(\pi_{k,j}(A'))\\
    &=\alpha\sum_{j=1}^{n} (-1)^{k+j} a_{k,j}\det(\pi_{k,j}(A'))\\
    &=\sum_{j=1}^{n} (-1)^{k+j} \alpha a_{k,j}\det(\pi_{k,j}(A'))\\
    &=\sum_{j=1}^{n} (-1)^{t+j} a_{k,j}\det(\pi_{k,j}(A))\\
    &=\det(A).
  \end{align*}
\end{proofof*}

\begin{proofof*}{det:multilinear:row}{sum}
  行列のサイズ$n$に関する帰納法で示す.
  \paragraph{Base case}
  $n=1$とする.
  このとき,
  $\det(\begin{pmatrix}a_{1,1}\end{pmatrix})=a_{1,1}=a'_{1,1}+a''_{1,1}=\det(\begin{pmatrix}a'_{1,1}\end{pmatrix})+\det(\begin{pmatrix}a''_{1,1}\end{pmatrix})$.
  \paragraph{Induction step}
  $n>1$とする.
  $A$の$(i,j)$-成分を$a_{i,j}$,
  $A'$の$(i,j)$-成分を$a'_{i,j}$
  とする.
  $k\neq t$とする.
  このとき,
  $\pi_{k,j}(A')$には$A'$の$t$行目が含まれるので,
  帰納法の仮定から,
  $\det(\pi_{k,j}(A'))+\det(\pi_{k,j}(A''))=\det(\pi_{k,j}(A))$
  である.
  また, $a'_{k,j}=a''_{k,j}=a_{k,j}$であるから,
  $\det(A')$, $\det(A'')$を$k$行目で展開すると,
  \begin{align*}
    \det(A')+\det(A'')
    &=\sum_{j=1}^{n} (-1)^{k+j} a'_{k,j}\det(\pi_{k,j}(A'))
    +\sum_{j=1}^{n} (-1)^{k+j} a''_{k,j}\det(\pi_{k,j}(A''))\\
    &=\sum_{j=1}^{n} (-1)^{k+j} a_{k,j}\det(\pi_{k,j}(A'))
    +\sum_{j=1}^{n} (-1)^{k+j} a_{k,j}\det(\pi_{k,j}(A''))\\
    &=\sum_{j=1}^{n} (-1)^{k+j} a_{k,j}(\det(\pi_{k,j}(A'))+\det(\pi_{k,j}(A'')))\\
    &=\sum_{j=1}^{n} (-1)^{k+j} a_{k,j}\det(\pi_{k,j}(A))\\
    &=\det(A).
  \end{align*}
\end{proofof*}

\begin{proofof}{det:alt:row:neg}
  行列のサイズ$n$に関する帰納法で示す.
  \paragraph{Base case}
  $n=2$のときは,
  \begin{align*}
    \det(
    \begin{pmatrix}
      a&b\\
      c&d
    \end{pmatrix}
    )=ad-bc\\
    -\det(
    \begin{pmatrix}
      c&d\\
      a&b
    \end{pmatrix}
    )=-(cb-ad)=ad-bc.
  \end{align*}

    \paragraph{Induction step}
  $n>2$のときについて考える.
  $A$の$(i,j)$-成分を$a_{i,j}$,
  $A'$の$(i,j)$-成分を$a'_{i,j}$
  とする.
  $k$は$t$でも$s$でもないとする.
  このとき,
  $\pi_{k,j}(A')$には$A'$の$t$行目, $s$行目がどちらも含まれるので,
  帰納法の仮定から,
  $-\det(\pi_{k,j}(A'))=\det(\pi_{k,j}(A))$
  である.
  また, $a'_{k,j}=a_{k,j}$であるから,
  $\det(A')$を$k$行目で展開すると,
  \begin{align*}
    -\det(A')
    &=-\sum_{j=1}^{n} (-1)^{k+j} a'_{k,j}\det(\pi_{k,j}(A'))\\
    &=-\sum_{j=1}^{n} (-1)^{k+j} a_{k,j}\det(\pi_{k,j}(A'))\\
    &=\sum_{j=1}^{n} (-1)^{k+j} (-1) a_{k,j}\det(\pi_{k,j}(A'))\\
    &=\sum_{j=1}^{n} (-1)^{t+j}a_{k,j}\det(\pi_{k,j}(A))\\
    &=\det(A).
  \end{align*}
\end{proofof}

\begin{proofof}{det:alt:row:zero}
  行列のサイズ$n$に関する帰納法で示す.
  \paragraph{Base case}
  $n=2$のときは,
  \begin{align*}
    \det(
    \begin{pmatrix}
      a&b\\
      a&b
    \end{pmatrix}
    )=ab-ba=.
  \end{align*}

  \paragraph{Induction step}
  $n>2$のときについて考える.
  $A$の$(i,j)$-成分を$a_{i,j}$,
  とする.
  $k$は$t$でも$s$でもないとする.
  このとき,
  $\pi_{k,j}(A)$には$A$の$t$行目, $s$行目がどちらも含まれるので,
  帰納法の仮定から,
  $-\det(\pi_{k,j}(A))=0$
  である.
  $\det(A)$を$k$行目で展開すると,
  \begin{align*}
    \det(A)
    &=\sum_{j=1}^{n} (-1)^{k+j} a_{k,j}\det(\pi_{k,j}(A))\\
    &=\sum_{j=1}^{n} (-1)^{k+j} a_{k,j}0\\
    &=0.
  \end{align*}
\end{proofof}
\begin{remark}
  $1+1\neq 0$のときには,
  以下のように証明することもできる:
  $A$の$t$行目と$s$行目を入れ替えた行列を$A'$とすると,
  \Cref{thm:det:alt:row:neg}から, $\det(A')=-\det(A)$
  である.
  また$A$の$t$行目と$s$行目が等しいので, $A=A'$である.
  従って, $\det(A)=-\det(A)$である.
  両辺に$\det(A)$を足すことで$2\det(A)=0$である.
  従って$\det(A)=0$.  
\end{remark}



\begin{proofof*}{lem:mul:alt:rowfun}{mul}
  多重線型性から明らかである.
\end{proofof*}
\begin{proofof*}{lem:mul:alt:rowfun}{sum}
  $A'$の$t$行目に$s$行目の$\alpha$倍を加えた行列を$A$とする.
  このとき, 
  $D(A)=D(A')$を示す.
  
  $t$行目以外は
  $A'$と等しく,
  $t$行目は
  $A'$の
  $s$行目と等しいような
  $n$次正方行列$A''$を考える.
  このとき, 交代性から$D(A'')=0$である.
  また, $A''$の$t$行目を$\alpha$倍をした行列を$A'''$とすると,
  $t$行目に関する線型性から
  $D(A''')=\alpha D(A'')=\alpha 0=0$
  である.
  一方$A'$の$t$行目に$s$行目の$\alpha$倍を加えたものを$A$とすると,
  $t$行目に関する線型性から,
  $D(A)=D(A')+D(A''')$である.
  したがって$D(A)=D(A')+D(A''')=D(A')+0=D(A')$.
\end{proofof*}
\begin{proofof*}{lem:mul:alt:rowfun}{alt}
  $A'$の$t$行目と$s$行目を入れ替えた行列を$A$とする.
  このとき, 
  $D(A)=-D(A')$を示す.

  $A'$の$t$行目に$s$行目の$-1$倍を加えるたものを$A''$とおく.
  $A''$の$s$行目に$t$行目の$1$倍を加えたものを$A'''$とする.
  $A'''$の$s$行目は$A'$の$t$行目と等しい.
  $A'''$の$t$行目に$s$行目の$-1$倍を加えたものを$A''''$とする.
  $A''''$の$t$行目は$A'$の$s$行目の$-1$倍に等しく,
  $A''''$の$s$行目は$A'$の$t$行目に等しい.
  また$D(A')=D(A'''')$である.
  $A''''$の$t$行目を$-1$倍したものは$A$に等しいので,
  $t$行目に関する線型性から,
  $D(A)=-D(A''')=-D(A')$.
\end{proofof*}

\begin{proofof}{lem:mul:alt:nonreg}
  $A$の$t$行目をすべて$0$倍した行列を$A'$とおくと,
  $\det(A')=0\det(A)=0$
  であるが,
  $A$の$t$行目の成分はもともと$0$であったので, $A=A'$である.
  よって$\det(A)=\det(A')=0$.
\end{proofof}


\subsection{余因子に関すること}
\begin{proofof*}{lem:cofac:expand}{row}
  $s=t$のときには, $\det(A)$を$t$行目で展開したものが左辺である.

  $s\neq t$のときには,
  次のような$n$次正方行列$A'$を考える:
  $A'$は$s$行目を除き$A$と等しい.
  $A'$の$s$行目は$A$の$t$行目と等しい.
  このとき, $A'$の$t$行目と$s$行目は等しいので,
  $\det(A')=0$である.
  一方$\det(A')$を展開したものが,
  左辺である.
\end{proofof*}
\begin{proofof*}{lem:cofac:expand}{col}
  $\det(A)=\det(\transposed{A})$であるので,
  行の場合に帰着できる.
\end{proofof*}

\begin{proofof}{adj:prod}
  $A\tilde A$を直接計算すると,
  $(s,t)$成分は,
  \Cref{lem:cofac:expand}の
  \cref{lem:cofac:expand:item:row}の
  左辺である.
  よって, $\det(A)E_n$に等しいことがわかる.

  また, $\tilde A A$は
  \Cref{lem:cofac:expand}の
  \cref{lem:cofac:expand:item:col}に帰着できる.
  左辺である.
\end{proofof}

\begin{proofof*}{det:adj}{zero}
  \Cref{lem:adj:prod}より,
  \begin{align*}
    A \tilde A =\det(A)E_n
  \end{align*}
  である.

  $\det(A)=0$とすると,
  $A \tilde A =O_{n,n}$
  となる.
  $A=O_{n,n}$のとき, $\tilde A=O_{n,n}$であるから,
  $\det(\tilde A)=0$.
  $A\neq O_{n,n}$のとき,
  零行列ではない行列をかけて零行列となったので,
  $\tilde A$は正則ではない.
  よって, $\det(\tilde A)=0$.
  
  一方$\det(A)\neq 0$とすると,
  $A$は正則で,
  $\tilde A=\det(A)A^{-1}$である.
  したがって$\tilde A$も正則であり,
  $\det(\tilde A)\neq 0$.


\end{proofof*}
\begin{proofof*}{det:adj}{pow}
\Cref{lem:adj:prod}より,
\begin{align*}
  A \tilde A =\det(A)E_n.
\end{align*}
行列式を考えると,
$\det(\alpha E_n)=\alpha^n$であるので,
\begin{align*}
  \det(A) \det(\tilde A) =\det(A)^n
\end{align*}
となる.
$\det(A)\neq 0$ならば,
$\det(A)$でわることで,
\begin{align*}
  \det(\tilde A) =\det(A)^{n-1}
\end{align*}

$\det(A)=0$ならば, 
$\det(\tilde A)=0$なので,
\begin{align*}
  \det(\tilde A)=0 =\det(A)^{n-1}.
\end{align*}

\end{proofof*}


\begin{proofof}{cramer}
  \label{thm:cramer}
  $A$が正則なので,
  $\xx=A^{-1}\bbb$である.
  $A^{-1}=\frac{1}{\det(A)}\adj(A)$であるので,
  $\Delta_{i,j}$を$A$の$(i,j)$-余因子とすると,
  \begin{align*}
    A^{-1}\bbb
    &=\frac{1}{\det(A)}\adj(A)\bbb\\
    &=\frac{1}{\det(A)}\transposed{
      \begin{pmatrix}
        \Delta_{1,1}&\Delta_{1,2}&\cdots&\Delta_{1,n}\\
        \Delta_{2,1}&\Delta_{2,2}&\cdots&\Delta_{2,n}\\
        \vdots\\
        \Delta_{n,1}&\Delta_{n,2}&\cdots&\Delta_{n,n}
      \end{pmatrix}
      }\bbb\\
    &=\frac{1}{\det(A)}
      \begin{pmatrix}
        \Delta_{1,1}&\Delta_{2,1}&\cdots&\Delta_{n,1}\\
        \Delta_{1,2}&\Delta_{2,2}&\cdots&\Delta_{n,2}\\
        \vdots\\
        \Delta_{1,n}&\Delta_{2,n}&\cdots&\Delta_{n,n}
      \end{pmatrix}
      \bbb\\
    &=\frac{1}{\det(A)}
      \begin{pmatrix}
        \Delta_{1,1}b_1+\Delta_{2,1}b_2+\cdots+\Delta_{n,1}b_n\\
        \Delta_{1,2}b_1+\Delta_{2,2}b_2+\cdots+\Delta_{n,2}b_n\\
        \vdots\\
        \Delta_{1,n}b_1+\Delta_{2,n}b_2+\cdots+\Delta_{n,n}b_n
      \end{pmatrix}.
  \end{align*}
  よって,
  \begin{align*}
    x_j=
    \frac{1}{\det(A)}
    \Delta_{1,j}b_j+\Delta_{2,j}b_2+\cdots+\Delta_{n,j}b_n
  \end{align*}
  一方,
  \begin{align*}
    A'&=
    \left(\begin{array}{c|c|c|c|c|c|c}
      \aaa_1&\cdots&\aaa_{j-1}&\bbb&\aaa_{j-1}&\cdots&\aaa_n
    \end{array}\right)
  \end{align*}
  とおき, $\det(A')$を$j$列目で展開すると,
  \begin{align*}
    \det(A')&=
    \Delta_{1,j}b_j+\Delta_{2,j}b_2+\cdots+\Delta_{n,j}b_n.
  \end{align*}
  よって,
  \begin{align*}
    x_j &= \frac{1}{\det(A)}\det(A').
  \end{align*}
\end{proofof}

\endinput

\section{逆行列や正則に関する命題の証明}
ここでは, \Cref{chap:inverse}に現れる命題の証明を行う.
逆行列や正則行列に関する命題については,
一般の正方行列の場合について証明をしているので,
煩雑になっている部分がある.
$2$-次正方行列の場合のみについて証明をするのであれば,
直接計算をすることで示すほうが手っ取り早いものが多い.
行列式に関する命題は,
行列式を$2$次正方行列にしかここでは定義していないこともあり,
$2$-次正方行列の場合のみについて証明をしている.





\begin{proofof}{det:hom}
  \begin{align*}
    A&=
    \begin{pmatrix}
      a&b\\c&d
    \end{pmatrix},&
    B&=
    \begin{pmatrix}
      a'&b'\\c'&d'
    \end{pmatrix}
  \end{align*}
  とする.
  \begin{align*}
    AB=
    \begin{pmatrix}
      a&b\\c&d
    \end{pmatrix}
    \begin{pmatrix}
      a'&b'\\c'&d'
    \end{pmatrix}
    =
    \begin{pmatrix}
    aa'+bc'&ab'+bd'\\ca'+dc'&db'+dd'
    \end{pmatrix}
  \end{align*}
  であるので,
  \begin{align*}
    \det(AB)&=(aa'+bc')(db'+dd')-(ab'+bd')(ca'+dc')\\
    &=aa'(db'+dd')+bc'(db'+dd')-ab'(ca'+dc')-bd'(ca'+dc')\\
    &=aa'db'+aa'dd'+bc'db'+bc'dd'-ab'ca'-ab'dc'-bd'ca'-bd'dc'\\
    &=ada'b'+ada'd'+bdb'c'+bdc'd'-aca'b'-adb'c'-bca'd'-bdc'd'\\
    &=ada'd'-adb'c'-bca'd'+bcb'c'
  \end{align*}
  である.
  一方,
  $\det(A)=ad-bc$,
  $\det(B)=a'd'-b'c'$
  であるので,
  \begin{align*}
    \det(A)\det(B)&=(ad-bc)(a'd'-b'c')\\
    &=ad(a'd'-b'c')-bc(a'd'-b'c')\\
    &=ada'd'-adb'c'-bca'd'+bcb'c'
  \end{align*}
  である.
\end{proofof}

\begin{proofof}{det:transpose}
  \begin{align*}
    A&=
    \begin{pmatrix}
      a&b\\c&d
    \end{pmatrix}
  \end{align*}
  とする.
  このとき, $\det(A)=ad-bc$である.
  一方
  \begin{align*}
    \transposed{A}&=
    \begin{pmatrix}
      a&c\\b&d
    \end{pmatrix}
  \end{align*}
  であるので,
  $\det(A)=ad-cb=ad-bc$である.
\end{proofof}

\begin{proofof}{det:alt:row}
  \begin{align*}
    A&=\begin{pmatrix}a&b\\c&d\end{pmatrix},&
    A'&=\begin{pmatrix}c&d\\a&b\end{pmatrix}
  \end{align*}
  とする.
  このとき, $\det(A)=ad-bc$である.
  一方$-\det(A')=-(cb-da)=-cb+da=ad-bc$である.
\end{proofof}
  
  
\begin{proofof}{multlin:row:add}
      \begin{align*}
        \det(\begin{pmatrix}a'+a''&b'+b''\\c&d\end{pmatrix})
          &=(a'+a'')d-(b'+b'')c=a'd+a''d-b'c-b''c\\
        \det(\begin{pmatrix}a'&b'\\c&d\end{pmatrix})
          +
          \det(\begin{pmatrix}a''&b''\\c&d\end{pmatrix})
            &=(a'd-b'c)+(a''d-b''c)=a'd+a''d-b'c-b''c
      \end{align*}
      である.
      また
      \begin{align*}
        \det(\begin{pmatrix}a&b\\c'+c''&d'+d''\end{pmatrix})&=
          a(d'+d'')-b(c'+c'')=ad'+ad''-bc'-bc''\\
        \det(\begin{pmatrix}a&b\\c'&d'\end{pmatrix})+
          \det(\begin{pmatrix}a&b\\c''&d''\end{pmatrix})
            &=(ad'-bc')+(ad''-bc'')=ad'+ad''-bc'-bc''
      \end{align*}
      である.
\end{proofof}

\begin{proofof}{multlin:row:prod}
  \begin{align*}
    \alpha\det(\begin{pmatrix}a&b\\c&d\end{pmatrix})
      &=\alpha(ad-bc)=\alpha ad-\alpha bc\\
      \det(\begin{pmatrix}\alpha  a&\alpha b\\c&d\end{pmatrix})
        &=\alpha a d-\alpha bc
  \end{align*}
  である.
  また
  \begin{align*}
    \alpha\det(\begin{pmatrix}a&b\\c&d\end{pmatrix})
      &=\alpha(ad-bc)=\alpha ad-\alpha bc\\
      \det(\begin{pmatrix}a&b\\\alpha   c&\alpha  d\end{pmatrix}).
        &=a\alpha d-b\alpha c=\alpha ad-\alpha bc
  \end{align*}
  である.
\end{proofof}

\section{階数に関する命題の証明}
ここでは, \Cref{chap:systemoflineq}に現れる命題の証明を行う.


\begin{proofof*}{thm:ker:vec}{sum}
  $\vv,\ww$を$A\xx=\zzero$の解とする.
  このとき$A\vv=\zzero$, $A\ww=\zzero$が成り立つ.
  したがって,
  $A(\vv+\ww)=A\vv+A\ww=\zzero+\zzero=\zzero$である.
\end{proofof*}

\begin{proofof*}{thm:ker:vec}{scalar}
  $\vv$を$A\xx=\zzero$の解とする.
  このとき$A\vv=\zzero$, $A\ww=\zzero$が成り立つ.
  $\alpha$を数とする.
  このとき,
  $A(\alpha\vv)=\alpha A\vv=\alpha\zzero=\zzero$である.
\end{proofof*}

\begin{lemma}
  \label{lem:fundamental}
  $A$を$(m,n)$-行列とする.
  $E_{i,j}$を$(i,j)$-成分のみ$1$で他は$0$の$n$次正方行列とする.
  \begin{enumerate}
  \item\label{lem:fundamental:item:sc}
    $F(i;c)=E_n-E_{i,i}+cE_{i,i}$とする.
    このとき, $F(i;c)A$は$A$の$i$行目を$c$倍した行列である.
    $c$が逆数を持つとき, $F(i;c)$は正則で, $F(i;c)^{-1}=F(i;\frac{1}{c})$である.
  \item\label{lem:fundamental:item:sum}
    $i\neq j$とする. $G(i,j;c)=E_n+cE_{i,j}$とする.
    このとき, $G(i,j;c)A$は$A$の$i$行目に$j$行目の$c$倍を加えた行列である.
    $G(i,j;c)$は正則で, $G(i,j;c)^{-1}=G(i,j;-c)$である.
  \item\label{lem:fundamental:item:trans}
    $H(i,j)=E_n-E_{i,i}-E_{j,j}+E_{i,j}+E_{j,i}$とする.
    このとき, $H(i,j)A$は$A$の$i$行目と$j$行目を入れ替えた行列である.
    $H(i,j)$は正則で, $H(i,j)^{-1}=H(i,j)$である.
  \end{enumerate}
\end{lemma}
\begin{proof}
  定義から
  \begin{align*}
    E_{i,j}E_{k,l}=
    \begin{cases}
      E_{i,l}&(j=k)\\
      O_{n,m}&(j\neq k)
    \end{cases}
  \end{align*}
  であることが計算できる.
  また,
  $E_{i,j}A$は, $i$行目が$A$の$j$行目と等しく, それ以外の成分は$0$であることも
  計算できる.
  \paragraph{\Cref{lem:fundamental:item:sc}}
  $F(i;c)$は$(i,i)$-成分のみ$c$で
  他の対角成分は$c$である対角行列である.

 
  $F(i;c)A=(E_n-E_{i,i}+cE_{i,i})A=A+(c-1)E_{i,i}A$である.
  したがって, $F(i;c)A$の$i$行目は$A$の$i$行目に$A$の$i$行目の$c-1$倍を加えたものであるので,
  $A$の$i$行目の$c$倍である.
  また$i$行目以外は$F(i;c)A$と$A$は等しい.
  つまり,
  $F(i;c)A$は$A$の$i$行目を$c$倍した行列である.

  また, $F(i;c)F(i;\frac{1}{c})$は$F(i;\frac{1}{c}$の$i$行目を$c$倍したものであるので,
  $E_n$である.
  $F(i;\frac{1}{c})F(i;c)$は$F(i;c$の$i$行目を$\frac{1}{c}$倍したものであるので,
  $E_n$である.
  よって$F(i;c)^{-1}=F(i;\frac{1}{c})$である.

  \paragraph{\Cref{lem:fundamental:item:sum}}
  $i\neq j$とする. $G(i,j;c)=E_n+cE_{i,j}$とする.
  $G(i,j;c)$は, 対角成分は全て$1$, $(i,j)$-成分は$c$, 他の成分は全て$0$である.

  $G(i,j;c)A=(E_n+cE_{i,j})A=A+cE_{i,j}A$である.
  したがって, $G(i,j;c)A$の$i$行目は$A$の$i$行目に$A$の$j$行目の$c$倍を加えたものである.
  また$i$行目以外は$G(i,j;c)A$と$A$は等しい.
  つまり, $G(i,j;c)A$は$A$の$i$行目に$j$行目の$c$倍を加えた行列である.

  また, $G(i,j;c)G(i,j;-c)$は, $G(i,j;-c)$の$i$行めに$G(i,j;-c)$の$j$行目の$c$倍を加えたものであるが, $G(i,j;-c)$の$j$行目は$(j,j)$成分が$1$で他は$0$であるので,
  $(i,j)$成分に$c$が加えられることになる.
  したがって, $G(i,j;c)G(i,j;-c)=E_n$である.
  一方$G(i,j;-c)G(i,j;c)$は, $G(i,j;c)$の$i$行めに$G(i,j;c)$の$j$行目の$-c$倍を加えたものであるが, $G(i,j;c)$の$j$行目は$(j,j)$成分が$1$で他は$0$であるので,
  $(i,j)$成分に$-c$が加えられることになる.
  したがって, $G(i,j;-c)G(i,j;c)=E_n$である.
  よって, $G(i,j;c)^{-1}=G(i,j;-c)$である.
  
  \paragraph{\Cref{lem:fundamental:item:trans}}
  $H(i,j)$は$(i,i)$-成分と$(j,j)$-成分は$0$, それ以外の対角成分は$1$,
  $(i,j)$-成分と$(j,i)$-成分は$1$, それ以外の成分は$0$である.

  $H(i,j)A=(E_n-E_{i,i}-E_{j,j}+E_{i,j}+E_{i,j})A=A-E_{i,i}A+E_{i,j}A-E_{j,j}A+E_{j,i}A$
  である.
  したがって,
  $H(i,j)A$の$i$行目は,
  $A$の$i$行めに$A$の$i$行目の$-1$倍と$A$の$j$行目の$1$倍を加えたものであるので,
  $A$の$j$行目である.
  同様に, $H(i,j)A$の$j$行目は,
  $A$の$j$行めに$A$の$j$行目の$-1$倍と$A$の$i$行目の$1$倍を加えたものであるので,
  $A$の$i$行目である.
  また$i$行目と$j$行目以外は$H(i,j)A$と$A$は等しい.
  つまり$H(i,j)A$は$A$の$i$行目と$j$行目を入れ替えた行列である.
  
  また, $H(i,j)H(i,j)$は$H(i,j)$の$i$行目と$j$行目を入れ替えたものであるので$E_n$である.
  したがって, $H(i,j)^{-1}=H(i,j)$である.
\end{proof}

%% \begin{remark}
%%   \label{lem:fund:is:invertible}
%%   \Cref{lem:fundamental}の$F(i;c)$, $G(i,j;c)$, $H(i,j)$
%%   を基本行列と呼ぶ.
%%   \Cref{lem:fundamental}から,
%%   $A$に行基本変形を行うことは,
%%   $A$に基本行列を左からかけることであり,
%%   逆に
%%   $A$に基本行列を左からかけることは,
%%   $A$に行基本変形を行うことであることがわかる.

%%   また, 基本行列は正則行列であり, その逆行列もまた基本行列である.
%%   このことは, $A$に行基本変形を行って$B$が得られたときには,
%%   $B$行基本変形を行って$A$にすることもできるということを意味する.
%% \end{remark}


\section{平面に関する命題の証明}
ここでは, \Cref{chap:lintrans}に現れる命題の証明を行う.

\begin{proofof}{thm:linindep:coef}
  $\aaa$, $\bbb$を$2$項実ベクトルとし,
  $(\aaa,\bbb)$は一次独立であるとする.
  $c\aaa + d\bbb = c'\aaa+d'\bbb$とする.
  このとき,
  \begin{align*}
    c\aaa + d\bbb = c'\aaa+d'\bbb\\
    c\aaa + d\bbb -c'\aaa-d'\bbb=\zzero \\
    (c-c')\aaa + (d -d')\bbb=\zzero 
  \end{align*}
  となるが, 
  $(\aaa,\bbb)$は一次独立であるので,
  $c-c'=$, $d-d'=0$である.
  したがって, $c=c'$, $d=d'$である.
\end{proofof}
\begin{proofof}{linindep:det}
  \begin{align*}
    \aaa&=\begin{pmatrix}a\\a'\end{pmatrix}\\
    \bbb&=\begin{pmatrix}b\\b'\end{pmatrix}\\
    A&=\begin{pmatrix}a&b\\a'&b'\end{pmatrix}
  \end{align*}
  とする. このとき,
  \begin{align*}
    x\aaa+y\bbb=\begin{pmatrix}ax+by\\a'x+b'y\end{pmatrix}=A\begin{pmatrix}x\\y\end{pmatrix}
  \end{align*}
  である.

  \paragraph{\Cref{thm:linindep:det:item:det}$\implies$\Cref{thm:linindep:det:item:linindep}}
  $\det(A)\neq 0$を仮定する.
  $x\aaa+y\bbb=\zzero$とする.
  このとき, $x=y=0$となることを示す.

  $x\aaa+y\bbb=\zzero$であるので,
  \begin{align*}
    A\begin{pmatrix}x\\y\end{pmatrix}=\zzero
  \end{align*}
  である.
  $\det(A)\neq 0$であるので, $A$は正則である.
  したがって,
  \begin{align*}
    \begin{pmatrix}x\\y\end{pmatrix}=A^{-1}\zzero=\zzero
  \end{align*}
  である.
  
  
  \paragraph{\Cref{thm:linindep:det:item:linindep}$\implies$\Cref{thm:linindep:det:item:det}}
  対偶を示す.
  $\det A=0$を仮定する.
  $x\aaa+y\bbb=\zzero$を満たす
  \begin{align*}
    \begin{pmatrix}x\\y\end{pmatrix}
  \end{align*}
  が$\zzero$以外にもあることを示す.
  
    
  $\det(A)\neq 0$であるので,
  \cref{thm:rank:regular}より,
  $\rank(A)<2$である.
  したがって, \Cref{thm:solution:homoge}より, 
  \begin{align*}
    A\begin{pmatrix}x\\y\end{pmatrix}&=\zzero,
    \begin{pmatrix}x\\y\end{pmatrix}&\neq\zzero
  \end{align*}
  を満たす
  \begin{align*}
    \begin{pmatrix}x\\y\end{pmatrix}
  \end{align*}
  がとれる. この$x,y$に対し,
  $x\aaa+y\bbb=\zzero$が成り立つ.
\end{proofof}
\begin{proofof*}{thm:innerprod:is:innerprod}{positive}
  \begin{align*}
        \aaa=\begin{pmatrix}a_1\\a_2\end{pmatrix}
  \end{align*}
  とし,   $a_i$は実数とする.
  $a_i$は実数であるから$a_i^2\geq 0$であるので,
  \begin{align*}
        \Braket{\aaa,\aaa}=\sum_{i=1}^2 a_i^2\geq 0
  \end{align*}
  となる. 
\end{proofof*}

\begin{proofof*}{thm:innerprod:is:innerprod}{nondeg}
  \begin{align*}
        \aaa=\begin{pmatrix}a_1\\a_2\end{pmatrix}
  \end{align*}
  とし,   $a_i$は実数とする.
  また$n=2$とする.

  \paragraph{$\langle{\aaa},{\aaa}\rangle=0 \implies \aaa=\zzero$.}
  対偶を示す.
  $\aaa\neq \zzero$とすると,
  $a_{i_0}\neq 0$となる$i_0$が存在する.
  $a_i$は実数であるから,
  この$i_0$に対して,
  $a_{i_0}>0$である.
  また, どの$i$に対して, $a_i^2\leq 0$である.
  したがって,
  \begin{align*}
    \Braket{\aaa,\aaa}=\sum_{i=1}^n a_i^2>0
  \end{align*}
  である.


  \paragraph{$\aaa=\zzero\implies \langle\aaa,\aaa\rangle=0$.}
  $\aaa=\zzero$とする. このとき, どの$i$に対しても$a_i=0$であるので,
  \begin{align*}
    \Braket{\aaa,\aaa}=\sum_{i=1}^n 0^2=0
  \end{align*}
  である.
  
\end{proofof*}

\begin{proofof*}{thm:innerprod:is:innerprod}{symmetric}
  \begin{align*}
    \aaa&=\begin{pmatrix}a_1\\a_2\end{pmatrix},&
    \bbb&=\begin{pmatrix}b_1\\b_2\end{pmatrix}
  \end{align*}
  とし,   $a_i$, $b_i$は実数とする.
  また$n=2$とする.
  このとき, 
  \begin{align*}
    \Braket{\aaa,\bbb}&=\sum_{i=1}^n a_ib_i\\
    \Braket{\bbb,\aaa}&=\sum_{i=1}^n b_ia_i=\sum_{i=1}^n a_ib_i
  \end{align*}
  である.
\end{proofof*}

\begin{proofof*}{thm:innerprod:is:innerprod}{additive}
  \begin{align*}
    \aaa&=\begin{pmatrix}a_1\\a_2\end{pmatrix},&
    \bbb&=\begin{pmatrix}b_1\\b_2\end{pmatrix},&
    \ccc&=\begin{pmatrix}c_1\\c_2\end{pmatrix},&
  \end{align*}
  とし,   $a_i$, $b_i$, $c_i$は実数とする.
  また$n=2$とする.
  このとき, 
  \begin{align*}
    \Braket{\aaa+\bbb,\ccc}&=\sum_{i=1}^n (a_i+b_i)c_i=\sum_{i=1}^n (a_ic_i+b_ic_i)=\sum_{i=1}^n a_ic_i+\sum_{i=1}^nb_ic_i\\
    \Braket{\aaa,\ccc}+\Braket{\bbb,\ccc}&=\sum_{i=1}^n a_ic_i+\sum_{i=1}^nb_ic_i
  \end{align*}
  である.
\end{proofof*}

\begin{proofof*}{thm:innerprod:is:innerprod}{mult}
  \begin{align*}
    \aaa&=\begin{pmatrix}a_1\\a_2\end{pmatrix},&
    \bbb&=\begin{pmatrix}b_1\\b_2\end{pmatrix},&
  \end{align*}
  とし,   $a_i$, $b_i$は実数とする.
  $r$を実数とする.
  また$n=2$とする.
  このとき, 
  \begin{align*}
    \Braket{r\aaa,\bbb}&=\sum_{i=1}^n ra_ib_i\\
    r\Braket{\aaa,\bbb}&=r\left(\sum_{i=1}^n a_ib_i\right)=\sum_{i=1}^n ra_ib_i
  \end{align*}
  である.
\end{proofof*}

\begin{proofof*}{norm:is:norm}{positive}
  \Cref{thm:innerprod:is:innerprod}より, $\Braket{\aaa,\aaa}\geq 0$である.
  したがって,
  $\|\aaa\|=\sqrt{\Braket{\aaa,\aaa}}\geq 0$である.
\end{proofof*}

\begin{proofof*}{norm:is:norm}{nondeg}
  \paragraph{$\|\aaa\|=0\implies\aaa=\zzero$.}
  $\|\aaa\|=0$とする.
  このとき, $\Braket{\aaa,\aaa}=\|\aaa\|^2=0$である.
  \Cref{thm:innerprod:is:innerprod}より, $\aaa=\zzero$である.
  \paragraph{$\aaa=\zzero\implies\|\aaa\|=0$.}
  $\aaa=\zzero$とする.
  \Cref{thm:innerprod:is:innerprod}より, $\Braket{\aaa,\aaa}=0$である.
  よって, $\|\aaa\|=\sqrt{\Braket{\aaa,\aaa}}=\sqrt0=0$である.
\end{proofof*}

\begin{lemma}[コーシー--シュワルツの不等式.]
  \label{lem:csi}
  $\aaa$, $\bbb$を$n$項実ベクトルとする.
  このとき, $|\Braket{\aaa,\bbb}|\leq \|\aaa\|\|\bbb\|$である.
\end{lemma}

\begin{proof}
  $\|\aaa\|\|\bbb\|\leq 0$であるので,
  $\Braket{\aaa,\bbb}^2\leq \|\aaa\|^2\|\bbb\|^2$
  を示す.

  $\bbb=\zzero$であれば,  $\Braket{\aaa,\bbb}^2=0=\|\aaa\|^2\|\bbb\|^2$である.

  $\bbb\neq \zzero$とする.
  $t$を実数とし,
  $\vv=\aaa+t\bbb$とおくと,
  \begin{align*}
    &\|\vv\|^2\\
    &=\|\aaa+t\bbb\|^2\\
    &=\Braket{\aaa+t\bbb,\aaa+t\bbb}\\
    &=\Braket{\aaa,\aaa+t\bbb}+\Braket{t\bbb,\aaa+t\bbb}\\
    &=\Braket{\aaa,\aaa+t\bbb}+t\Braket{\bbb,\aaa+t\bbb}\\
    &=\Braket{\aaa,\aaa}+\Braket{\aaa,t\bbb}+t\Braket{\bbb,\aaa}+t\Braket{\bbb,t\bbb}\\
    &=\Braket{\aaa,\aaa}+t\Braket{\aaa,\bbb}+t\Braket{\aaa,\bbb}+t^2\Braket{\bbb,\bbb}\\
    &=\Braket{\aaa,\aaa}+2t\Braket{\aaa,\bbb}+t^2\Braket{\bbb,\bbb}
  \end{align*}
  である.
  $a=\Braket{\bbb,\bbb}$,
  $b=\Braket{\aaa,\bbb}$, 
  $c=\Braket{\aaa,\aaa}$とおくと,
  $\|\vv\|^2=at^2+bt+c$であるが,
  $\|\vv\|^2\leq 0$であるので,
  全ての$t$で
  \begin{align*}
    at^2+bt+c \geq 0
  \end{align*}
  が成り立っている.
  $\bbb\neq\zzero$より$a\neq 0$であるので,
  \cref{lem:numbers:quad}より,
  判別式を考えるとその値は$0$以下である.
  よって,
  \begin{align*}
    0&\geq b^2-4ac\\
    &\Braket{\aaa,\bbb}^2-\Braket{\bbb,\bbb}\Braket{\aaa,\aaa}
  \end{align*}
  である. したがって,
  \begin{align*}
    \Braket{\aaa,\aaa}\Braket{\bbb,\bbb}&\geq\Braket{\aaa,\bbb}^2\\
    \|\aaa\|^2\|\bbb\|^2&\geq\Braket{\aaa,\bbb}^2
  \end{align*}
  である.
\end{proof}

\begin{proofof*}{norm:is:norm}{triangle}
  $\|\aaa\|+\|\bbb\|\leq 0$であるので,
  $\|\aaa+\bbb\|^2 \leq (\|\aaa\|+\|\bbb\|)^2$
  を示す.
  \begin{align*}
    \|\aaa+\bbb\|^2
    &=\Braket{\aaa+\bbb,\aaa+\bbb}\\
    &=\Braket{\aaa,\aaa+\bbb}+\Braket{\bbb,\aaa+\bbb}\\
    &=\Braket{\aaa,\aaa}+\Braket{\aaa,\bbb}+\Braket{\bbb,\aaa}+\Braket{\bbb,\bbb}\\
    &=\Braket{\aaa,\aaa}+\Braket{\aaa,\bbb}+\Braket{\aaa,\bbb}+\Braket{\bbb,\bbb}\\
    &=\Braket{\aaa,\aaa}+2\Braket{\aaa,\bbb}+\Braket{\bbb,\bbb}\\
    &=\|\aaa\|^2+2\Braket{\aaa,\bbb}+\|\bbb\|^2\\
    (\|\aaa\|+\|\bbb\|)^2
    &=(\|\aaa\|^2+2\|\aaa\|\|\bbb\|+\|\bbb\|^2)\\
  \end{align*}
  である.
  \Cref{lem:csi}より,
  $\Braket{\aaa,\bbb}\leq \|\aaa\|\|\bbb\| $
  であるので,
  $\|\aaa+\bbb\|^2 \leq (\|\aaa\|+\|\bbb\|)^2$
  である.
\end{proofof*}


\begin{proofof*}{norm:is:norm}{scalar}
  \Cref{thm:innerprod:is:innerprod}より,
  \begin{align*}
    \Braket{r\aaa,r\aaa}=r\Braket{\aaa,r\aaa}=r\Braket{r\aaa,\aaa}=r^2\Braket{\aaa,\aaa}
  \end{align*}
  である. したがって, 
  \begin{align*}
    \|r\aaa\|=\sqrt{\Braket{r\aaa,r\aaa}}=\sqrt{r^2\Braket{\aaa,\aaa}}
    =\sqrt{r^2}\sqrt{\Braket{\aaa,\aaa}}
    =|r|\sqrt{\Braket{\aaa,\aaa}}
    =|r|\|\aaa\|
  \end{align*}
  である.
\end{proofof*}

\begin{lemma}
  \label{lem:ortho:det}
  $A$を実数を成分とする2次正方行列であるとする.
  $A$が直交行列なら, $\det(A)\in \Set{1,-1}$である.
\end{lemma}
\begin{proof}
  $\transposed{A}A=E_n$であるとする.
  このとき, $\det(\transposed{A}A)=\det(E_n)=1$である.
  一方,  $\det(\transposed{A}A)=\det(\transposed{A})\det(A)=\det(A)\det(A)=\det(A)^2$である. よって$\det(A)\in\Set{1,-1}$である.
\end{proof}

\begin{proofof}{thm:ortho:char}
  $A$は実数を成分とする$2$次正方行列であるとし,
  \begin{align*}
    A&=
    \begin{pmatrix}
      a&b\\
      a'&b'
    \end{pmatrix}\\
    \aaa&=
    \begin{pmatrix}
      a\\
      a'
    \end{pmatrix}\\
    \bbb&=
    \begin{pmatrix}
      b\\
      b'
    \end{pmatrix}
  \end{align*}
  とする.
  また, $n=2$とする.

  このとき,
  \begin{align*}
    \transposed{A}A&=
    \begin{pmatrix}
      a&a'\\
      b&b'
    \end{pmatrix}
    \begin{pmatrix}
      a&b\\
      a'&b'
    \end{pmatrix}\\
    &=
    \begin{pmatrix}
      a^2+(a')^2&ab+a'b'\\
      ba+b'a'&b^2+(b')^2
    \end{pmatrix}\\
    &=
    \begin{pmatrix}
      \Braket{\aaa,\aaa}&\Braket{\aaa,\bbb}\\
      \Braket{\bbb,\aaa}&\Braket{\bbb,\bbb}
    \end{pmatrix}
  \end{align*}
  である.
  
  \paragraph{\Cref{thm:ortho:char:item:transposeisinverse}$\implies$\Cref{thm:ortho:char:item:ortho}}
  $A$が正則であり,
  $A^{-1}=\transposed{A}$とする.
  このとき,
  $\transposed{A}A=A^{-1}A=E_n$

  \paragraph{\Cref{thm:ortho:char:item:transposeisinverse}$\implies$\Cref{thm:ortho:char:item:ortho2}}
  $A$が正則であり,
  $A^{-1}=\transposed{A}$とする.
  このとき,
  $A\transposed{A}=AA^{-1}=E_n$

  \paragraph{\Cref{thm:ortho:char:item:ortho}$\implies$\Cref{thm:ortho:char:item:transposeisinverse}}
  $\transposed{A}A=E_n$とする.
  このとき, \cref{lem:ortho:det}より,
  $\det(A)\neq 0$である.
  したがって, $A$は正則である.
  $\transposed{A}A=E_n$の
  両辺に右から$A^{-1}$をかけることで
  $\transposed{A}=A^{-1}$が得られる.

  \paragraph{\Cref{thm:ortho:char:item:ortho2}$\implies$\Cref{thm:ortho:char:item:transposeisinverse}}
  $A\transposed{A}=E_n$とする.
  このとき, \cref{lem:ortho:det}より,
  $\det(A)\neq 0$である.
  したがって, $A$は正則である.
  $A\transposed{A}=E_n$の
  両辺に左から$A^{-1}$をかけることで
  $\transposed{A}=A^{-1}$が得られる.

  \paragraph{\Cref{thm:ortho:char:item:ortho}$\implies$\Cref{thm:ortho:char:item:orthonormalbasis}}
  $\transposed{A}A=E_n$とする.
  \begin{align*}
    \begin{pmatrix}
      \Braket{\aaa,\aaa}&\Braket{\aaa,\bbb}\\
      \Braket{\bbb,\aaa}&\Braket{\bbb,\bbb}
    \end{pmatrix}=
    \begin{pmatrix}
      1&0\\
      0&1
    \end{pmatrix}
  \end{align*}
  であるので, 
  $\Braket{\aaa,\bbb}=0$, $\|\aaa\|=\sqrt{\Braket{\aaa,\aaa}}=\sqrt{1}=1$,
  $\|\bbb\|=\sqrt{\Braket{\bbb,\bbb}}=\sqrt{1}=1$である.
  \paragraph{\Cref{thm:ortho:char:item:orthonormalbasis}$\implies$\Cref{thm:ortho:char:item:ortho}}
  $(\aaa,\bbb)$が正規直交基底であるとする.
  このとき,
  $\Braket{\aaa,\bbb}=0$,
  $\Braket{\aaa,\aaa}=\|\aaa\|^2=1^2=1$,
  $\Braket{\bbb,\bbb}=\|\bbb\|^2=1^2=1$
  である.
  


  \paragraph{\Cref{thm:ortho:char:item:preserve}$\implies$\Cref{thm:ortho:char:item:ortho}}
  $\ee_i$を基本ベクトルとする.
  $S$を正方行列に対し,
  $\transposed{\ee_{i}}S\ee_{j}$は,
  $(1,1)$-行列でその成分は
  $S$の$(i,j)$である.
  また, $\xx,\yy\in\RR^n$に対し,
  $\transposed{\xx}\yy$は, $(1,1)$-行列でその成分は
  $\Braket{\xx,\yy}$である.

  仮定から,
  $\transposed{\xx}\yy=\transposed{(A\xx)}(A\xx)=\transposed{\xx}\transposed{A}A\xx$
  である.
  $\xx=\ee_i$, $\yy=\ee_j$とすると,
  \begin{align*}
    \transposed{\ee_i}\transposed{A}A\ee_j=\transposed{\ee_i}\ee_j
  \end{align*}
  である. $\Braket{\ee_i,\ee_j}$は, $i=j$のとき$1$, $i\neq j$のとき$0$であるから,
  $\transposed{A}A$の$(i,j)$-成分も, $i=j$のとき$1$, $i\neq j$のとき$0$である.
  したがって,
  
  \paragraph{\Cref{thm:ortho:char:item:ortho}$\implies$\Cref{thm:ortho:char:item:preserve}}

  $\transposed{A}A=E_n$とする.
  $\xx,\yy\in\RR^n$とする.
  \begin{align*}
    \transposed{(A\xx)}(A\yy)=\transposed{\xx}\transposed{A}A\yy=\transposed{\xx}E_n\yy=\transposed{\xx}\yy
  \end{align*}
  であるので,
  $\braket{\xx,\yy}=\braket{A\xx,A\yy}$である.
\end{proofof}


\begin{proofof}{kernel:inj}
$f$が線形変換であるとする.
\paragraph{\Cref{thm:kernel:ing:item:inj}$\implies$\Cref{thm:kernel:ing:item:ker}}
対偶を示す.
$\ker(f)\neq \Set{\zzero}$とする.
\Cref{thm:linmap:zero}から, $f(\zzero)=\zzero$であるので
$\zzero\in \Ker(f)$である.
よって, $\zzero\neq \aaa$かつ$ \aaa\in\Ker(f)$
となる$\aaa$が存在する.
$\aaa\neq\zzero$かつ$f(\aaa)=f(\zzero)$であるので, $f$は単射ではない.

\paragraph{\Cref{thm:kernel:ing:item:ker}$\implies$\Cref{thm:kernel:ing:item:inj}}
対偶を示す.
$f$が単射ではないとする.
このとき,
$\aaa\neq\bbb$かつ$f(\aaa)=f(\bbb)$
を満たす$\aaa$, $\bbb$が存在する.
この$\aaa$, $\bbb$に対して
$\vv=\aaa-\bbb$とおくと,
$f(\vv)=\zzero$であるので,
$\vv\in \Ker(f)$である.
$\vv\neq \zzero$であるので,
$\ker(f)\neq \Set{\zzero}$である.

\end{proofof}


\begin{proofof}{thm:repmat:inverse}
  線形変換$f$は行列$A$を用いて
  $f(\xx)=A\xx$と表せているとする.
  線形変換$g$は行列$B$を用いて
  $f(\xx)=B\xx$と表せているとする.
  このとき
  $g\circ f(\xx)=BA\xx$,
  $f\circ g(\xx)=AB\xx$
  である.

  \paragraph{$f$が全単射$\implies$ $A$は正則.}
  $f$が全単射であるとする.
  $g=f^{-1}$とし, $g(\xx)=B\xx$とする.
  このとき,
  $g\circ f=\id_{\RR^2}$であるので,
  $BA=E_2$である.
  また,
  $f\circ g=\id_{\RR^2}$であるので,
  $AB=E_2$である.
  したがって$B$は$A$の逆行列である.
  
  \paragraph{$A$は正則$\implies$ $f$が全単射.}
  $A$が正則であるとする.
  $B=A^{-1}$とし, $g(\xx)=B\xx$とする.
  このとき,
  $BA=E_2$であるので$g\circ f=\id_{\RR^2}$である.
  また,
  $AB=E_2$であるので
  $f\circ g=\id_{\RR^2}$である.
  したがって$g$は$f$の逆変換である.
  
\end{proofof}
\section{固有値に関連する命題の証明}
\begin{proofof}{thm:diagonizablity}
  \begin{align*}
    A=
    \begin{pmatrix}
      a &b\\c&d
    \end{pmatrix}
  \end{align*}
  とする.
  \paragraph{\Cref{thm:diagonizablity:item:diagonalizable}$\implies$\Cref{thm:diagonizablity:item:eigen}}
  正則行列$P$で,
  \begin{align*}
    P^{-1}AP=
    \begin{pmatrix}
      \lambda&0\\0&\mu
    \end{pmatrix}
  \end{align*}
  とできるとする.
  このとき,
  \begin{align*}
    AP=
    P\begin{pmatrix}
      \lambda&0\\0&\mu
    \end{pmatrix}
  \end{align*}
  である.
  \begin{align*}
    P&=
    \begin{pmatrix}
      p&q\\p'&q'
    \end{pmatrix}\\
    \aaa&=
    \begin{pmatrix}
      p\\p'
    \end{pmatrix}\\
    \bbb&=
    \begin{pmatrix}
      q\\q'
    \end{pmatrix}
  \end{align*}
  とすると
  \begin{align*}
    P
    \begin{pmatrix}
      \lambda &0\\0&\mu
    \end{pmatrix}=
    \begin{pmatrix}
      \lambda p&\mu q\\ \lambda p'&\mu q'
    \end{pmatrix}
  \end{align*}
  である.
  また,
  \begin{align*}
    AP&=
    \begin{pmatrix}
      a &b\\c&d
    \end{pmatrix}
    \begin{pmatrix}
      p&q\\p'&q'
    \end{pmatrix}=
    \begin{pmatrix}
      ap+bp' &aq+bq'\\cp+dp'&cq+dq'
    \end{pmatrix}
  \end{align*}
  であるので,
  \begin{align*}
    \begin{pmatrix}
      ap+bp' &aq+bq'\\cp+dp'&cq+dq'
    \end{pmatrix}
    =
    \begin{pmatrix}
      \lambda p&\mu q\\ \lambda p'&\mu q'
    \end{pmatrix}
  \end{align*}
  である.
  したがって,
  \begin{align*}
    A\aaa&=
        \begin{pmatrix}
      ap+bp' \\cp+dp'
    \end{pmatrix}
    =
    \begin{pmatrix}
      \lambda p\\ \lambda p'
    \end{pmatrix}
    =
    \lambda \aaa\\
    A\bbb&=
    \begin{pmatrix}
      aq+bq'\\cq+dq'
    \end{pmatrix}
    =
    \begin{pmatrix}
      \mu q\\\mu q'
    \end{pmatrix}
    =
    \mu \bbb
  \end{align*}
  である.
  $P$が正則であるので\cref{thm:det:reg}より$(\aaa,\bbb)$は一次独立であり,
  $\aaa$は固有値$\lambda$に属する固有ベクトル,
  $\bbb$は固有値$\mu$に属する固有ベクトル,
  である.
  
  \paragraph{\Cref{thm:diagonizablity:item:eigen}$\implies$\Cref{thm:diagonizablity:item:diagonalizable}}
  $\aaa$は固有値$\lambda$に属する固有ベクトル,
  $\bbb$は固有値$\mu$に属する固有ベクトルとし,
  $(\aaa,\bbb)$は一次独立とする.
  \begin{align*}
    \aaa&=
    \begin{pmatrix}
      p\\p'
    \end{pmatrix}\\
    \bbb&=
    \begin{pmatrix}
      q\\q'
    \end{pmatrix}\\
    P&=
    \begin{pmatrix}
      p&q\\p'&q'
    \end{pmatrix}
  \end{align*}
  とすると
  \cref{thm:det:reg}より,
  $P$は正則である.

  また,
  \begin{align*}
    A\aaa&=
        \begin{pmatrix}
      ap+bp' \\cp+dp'
    \end{pmatrix}
    =
    \begin{pmatrix}
      \lambda p\\ \lambda p'
    \end{pmatrix}
    =
    \lambda \aaa\\
    A\bbb&=
    \begin{pmatrix}
      aq+bq'\\cq+dq'
    \end{pmatrix}
    =
    \begin{pmatrix}
      \mu q\\\mu q'
    \end{pmatrix}
    =
    \mu \bbb
  \end{align*}
  であるので,
  \begin{align*}
    AP&=
    \begin{pmatrix}
      a &b\\c&d
    \end{pmatrix}
    \begin{pmatrix}
      p&q\\p'&q'
    \end{pmatrix}=
    \begin{pmatrix}
      ap+bp' &aq+bq'\\cp+dp'&cq+dq'
    \end{pmatrix}
    =
    \begin{pmatrix}
      \lambda p &\mu q\\\lambda p'&\mu q'
    \end{pmatrix}
    =
    P
    \begin{pmatrix}
      \lambda  &0\\0&\mu 
    \end{pmatrix}.
  \end{align*}
  $P$は正則なので, $P^{-1}$を左からかけると,
  \begin{align*}
    AP&=
    P
    \begin{pmatrix}
      \lambda  &0\\0&\mu 
    \end{pmatrix}\\
    P^{-1}AP&=
    \begin{pmatrix}
      \lambda  &0\\0&\mu 
    \end{pmatrix}
  \end{align*}
  である.
\end{proofof}

\begin{proofof}{thm:diferenteigen:linindep}
  $A$を$2$次正方行列とする.
  $\vv$は固有値$\lambda$に属する$A$の固有ベクトル,
  $\ww$は固有値$\mu$に属する$A$の固有ベクトルとする.
  \begin{align*}
    \vv&=
    \begin{pmatrix}
      v\\v' 
    \end{pmatrix}\\
    \ww&=
    \begin{pmatrix}
      w\\w' 
    \end{pmatrix}
  \end{align*}
  とおく.
  $x\vv+y\ww=\zzero$とする.
  このとき, $x=y=0$を示す.

  $x\vv+y\ww=\zzero$であるので,
  \begin{align*}
    x\vv+y\ww&=\zzero\\
    A(x\vv+y\ww)&=A\zzero\\
    xA\vv+yA\ww&=\zzero\\
    x\lambda \vv+y\mu \ww&=\zzero
  \end{align*}
  である.
  $x\vv+y\ww=\zzero$, $x\lambda \vv+y\mu \ww=\zzero$
  であるので,
  \begin{align*}
    \begin{pmatrix}
      xv&yw\\xv'&yw' 
    \end{pmatrix}
    \begin{pmatrix}
      1&\lambda\\1&\mu 
    \end{pmatrix}
    =
    \begin{pmatrix}
      xv+yw&\lambda xv+\mu yw\\xv'+xv'&\lambda xv'+\mu yw' 
    \end{pmatrix}
    =
    \begin{pmatrix}
      0&0\\0&0
    \end{pmatrix}
  \end{align*}
  となる.
  \begin{align*}
    \det(
    \begin{pmatrix}
      1&\lambda\\1&\mu 
    \end{pmatrix})
    =\mu-\lambda\neq 0
  \end{align*}
  であるので,
  \begin{align*}
    \begin{pmatrix}
      1&\lambda\\1&\mu 
    \end{pmatrix}
  \end{align*}
  は正則である.
  \begin{align*}
    \begin{pmatrix}
      xv&yw\\xv'&yw' 
    \end{pmatrix}=
    \begin{pmatrix}
      xv&yw\\xv'&yw' 
    \end{pmatrix}
    \begin{pmatrix}
      1&\lambda\\1&\mu 
    \end{pmatrix}
    \begin{pmatrix}
      1&\lambda\\1&\mu 
    \end{pmatrix}^{-1}
    =
    \begin{pmatrix}
      0&0\\0&0
    \end{pmatrix}
    \begin{pmatrix}
      1&\lambda\\1&\mu 
    \end{pmatrix}^{-1}
    =
    \begin{pmatrix}
      0&0\\0&0
    \end{pmatrix}
  \end{align*}
  である.
  よって,
  \begin{align*}
    x\vv&=
    \begin{pmatrix}
      xv\\xv' 
    \end{pmatrix}=
    \begin{pmatrix}
      0\\0
    \end{pmatrix}
    =\zzero\\
    y\ww&=
    \begin{pmatrix}
      yw\\yw' 
    \end{pmatrix}=
    \begin{pmatrix}
      0\\0
    \end{pmatrix}
    =\zzero
  \end{align*}
  である. $\vv\neq\zzero$であるから, $x=0$である.
  また, $\ww\neq\zzero$であるから, $y=0$である.
\end{proofof}

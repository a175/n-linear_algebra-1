% !TeX root =./x2.tex
% !TeX program = pdfpLaTeX


\chapter{章末問題の略解}

ここでは, 章末問題に関し簡単な解説と略解を挙げる.
答案には必要なことであっても省略しているものある.
逆に答案には必要ではないことが書いてあったりする.
ここで述べられているものは,
解答例でも模範解答でもないことに注意すること.


\section{実数の絶対値や平方根に関する問題}
\begin{answerof}{quiz:0:1}
  たとえば, $a=-2$のときには,
  $\sqrt{(-2)^2}=\sqrt{4}=2\neq -2$である.
  この例のように$\sqrt{a^2}\neq a$となる場合もある.  

  一般には, $\sqrt{a^2}=|a|$である.
\end{answerof}

\begin{answerof}{quiz:0:2}
  $5>2^2$であるから$\sqrt{5}>2$である.
  したがって, $2-\sqrt{5}<0$である.
  よって,
  $|2-\sqrt{5}|=-2+\sqrt{5}$
  である.

  一般に実数$a$の絶対値$|a|$は,
  $a>0$なら$|a|=a$であり,
  $a<0$なら$|a|=-a$である.
\end{answerof}

\begin{answerof}{quiz:0:3}
  判別式$D$を考える.
  \begin{align*}
    D=(-(s+u))^2-4su=s^2+2su+u^2-4su=s^2-2su+u^2=(s-u)^2
  \end{align*}
  である.  $D\geq 0$なので実数解をもつ.
\end{answerof}

\section{行列の定義と演算に関する問題}

\begin{answerof}{quiz:1:1}
  サイズ(型)は$(2,3)$.
  $A$の$(1,2)$-成分は$2$.
  $A$の$2$-行目は
  \begin{align*}\begin{pmatrix}1&2&3\end{pmatrix}.\end{align*}
  $A$の$2$-列目は
  \begin{align*}\begin{pmatrix}2\\7\end{pmatrix}.\end{align*}
  $A$の転置は
  \begin{align*}\begin{pmatrix}1&6\\2&7\\3&8\end{pmatrix}.\end{align*}
\end{answerof}

\begin{answerof}{quiz:1:2}
  具体的に成分を書き下せばわかる.
  \begin{enumerate}
  \item
    $(i,j)=(1,1)$のとき, $5i+j-5=1$.
    $(i,j)=(1,2)$のとき, $5i+j-5=2$.
    $(i,j)=(2,1)$のとき, $5i+j-5=6$.
    $(i,j)=(2,2)$のとき, $5i+j-5=7$.
    したがって,
    \begin{align*}A=\begin{pmatrix}1&2\\6&7\end{pmatrix}.\end{align*}
  \item
    $i\neq j$のとき, $2^i\delta_{i,j}=2^i\cdot 0=0$.
    $i=j=1$のとき, $2^i\delta_{i,j}=2^1\cdot 1=2$.
    $i=j=2$のとき, $2^i\delta_{i,j}=2^2\cdot 1=4$.
    したがって,
    \begin{align*}A=\begin{pmatrix}2&0\\0&4\end{pmatrix}.\end{align*}
  \item
    $i\neq 2$のとき, $\delta_{i,2}\delta_{j,1}=0\cdot \delta_{j,1}=0$.
    $j\neq 1$のとき, $\delta_{i,2}\delta_{j,1}=\delta_{i,2}\cdot 0=0$.
    $i=2, j=1$のとき, $\delta_{i,2}\delta_{j,1}=1\cdot 1=1$.
    したがって,
    \begin{align*}A=\begin{pmatrix}0&0\\1&0\end{pmatrix}.\end{align*}
  \end{enumerate}
\end{answerof}
\begin{answerof}{quiz:1:3}
  \begin{align*}A&=
    \begin{pmatrix}
      a+1& 3\\
      4+c &5d-10
    \end{pmatrix}\\
    B&=
    \begin{pmatrix}
      7& 2b+5\\
      6 &-4d+8
    \end{pmatrix}
  \end{align*}
  とする.
  行列が等しいということは, 対応する各成分が等しいということなので,
  $A=B$とすると
  \begin{align*}
    \begin{cases}
      a+1=7\\
      3=2b+5\\
      4+c=6\\
      5d-10=-4d+8
    \end{cases}
  \end{align*}
  となる.
  したがって,
  \begin{align*}
    \begin{cases}
      a=6\\
      b=-1\\
      c=2\\
      d=2
    \end{cases}
  \end{align*}
  を得る.
  実際, $(a,b,c,d)=(6,-1,2,2)$とすると,
  \begin{align*}A&=
    \begin{pmatrix}
      a+1& 3\\
      4+c &5d-10
    \end{pmatrix}
    =
    \begin{pmatrix}
      6+1& 3\\
      4+2 &5\cdot 2-10
    \end{pmatrix}
    =
    \begin{pmatrix}
      7& 3\\
      6&0
    \end{pmatrix}
    \\
    B&=
    \begin{pmatrix}
      7& 2b+5\\
      6 &-4d+8
    \end{pmatrix}
=    \begin{pmatrix}
      7& 2(-1)+5\\
      6 &-4\cdot 2+8
    \end{pmatrix}
=    \begin{pmatrix}
      7& 3\\
      6 &0
    \end{pmatrix}
  \end{align*}
  である.
\end{answerof}

\begin{answerof}{quiz:1:4}
  \begin{align*}
    A=
    \begin{pmatrix}
      1& 3\\
      4+a &5
    \end{pmatrix}
  \end{align*}
  とする.  $A$が対称行列であるということは, $A=\transposed{A}$ということである.
  \begin{align*}
    \transposed{A}=
    \begin{pmatrix}
      1& 4+a\\
      3 &5
    \end{pmatrix}
  \end{align*}
  であるので, $A=\transposed{A}$とすると,
  \begin{align*}
    \begin{pmatrix}
      1& 3\\
      4+a &5
    \end{pmatrix}
=
    \begin{pmatrix}
      1& 4+a\\
      3 &5
    \end{pmatrix}
  \end{align*}
  である.
  したがって, 
  \begin{align*}
    \begin{cases}
      1=1\\
      3=4+a\\
      4+a=3\\
      5=5
    \end{cases}
  \end{align*}
  となり, $a=-1$を得る.
  実際$a=-1$のとき,
  \begin{align*}
    A&=
    \begin{pmatrix}
      1& 3\\
      4+a &5
    \end{pmatrix}=
    \begin{pmatrix}
      1& 3\\
      3 &5
    \end{pmatrix}\\
    \transposed{A}&=
    \begin{pmatrix}
      1& 3\\
      3 &5
    \end{pmatrix}
  \end{align*}
  であるので対称行列である.
\end{answerof}


\begin{answerof}{quiz:1:5}
  \begin{align*}
    A=
    \begin{pmatrix}
      1-a& 3+b\\
      4  &0
    \end{pmatrix}
  \end{align*}
  とする.
  $A$が交代行列であるということは, $-A=\transposed{A}$ということである.
  \begin{align*}
    -A&=
    -\begin{pmatrix}
      1-a& 3+b\\
      4  &0
    \end{pmatrix}
=    \begin{pmatrix}
      -1+a& -3-b\\
      -4  &0
    \end{pmatrix}\\    
    \transposed{A}&=
    \begin{pmatrix}
      1-a& 4\\
      3+b &0
    \end{pmatrix}
  \end{align*}
  であるので, $-A=\transposed{A}$とすると,
  \begin{align*}
    \begin{pmatrix}
      -1+a& -3-b\\
      -4  &0
    \end{pmatrix}&=
    \begin{pmatrix}
      1-a& 4\\
      3+b &0
    \end{pmatrix}
  \end{align*}
  である.
  したがって, 
  \begin{align*}
    \begin{cases}
      -1+a=1-a\\
      -3-b=4\\
      -4=3+b\\
      0=0
    \end{cases}
  \end{align*}
  となり, $a=1$, $b=-7$を得る.
  実際$(a,b)=(1,-7)$のとき,
  \begin{align*}
    -A&=
    \begin{pmatrix}
      -1+a& -3-b\\
      -4  &0
    \end{pmatrix}
    =
    \begin{pmatrix}
      -1+1& -3-(-7)\\
      -4  &0
    \end{pmatrix}
    =
    \begin{pmatrix}
      0& 4\\
      -4  &0
    \end{pmatrix}\\    
    \transposed{A}&=
    \begin{pmatrix}
      1-a& 4\\
      3+b &0
    \end{pmatrix}
    =
      \begin{pmatrix}
      1-1& 4\\
      3+-7 &0
    \end{pmatrix}
    =
      \begin{pmatrix}
      0& 4\\
      -4 &0
    \end{pmatrix}
\end{align*}
となり$A$は交代行列である.  
\end{answerof}

\begin{answerof}{quiz:1:6}
  \begin{align*}
    \begin{pmatrix}
      10\\4
    \end{pmatrix}
    +
    \begin{pmatrix}
      -2\\-3
    \end{pmatrix}
    +
    \begin{pmatrix}
      -3\\6
    \end{pmatrix}
    =
    \begin{pmatrix}
      10-2-3\\4-3+6
    \end{pmatrix}
    =
    \begin{pmatrix}
      5\\7
    \end{pmatrix}.
  \end{align*}

  \begin{align*}
    \begin{pmatrix}
      6&0\\0&4
    \end{pmatrix}+
    \begin{pmatrix}
      0&3\\-7&0
    \end{pmatrix}
    =
    \begin{pmatrix}
      6+0&0+3\\0-7&4+0
    \end{pmatrix}
    =
    \begin{pmatrix}
      6&3\\-7&4
    \end{pmatrix}.
  \end{align*}

  \begin{align*}
    \begin{pmatrix}
      -9&2\\3&7
    \end{pmatrix}
    \begin{pmatrix}
      1&0\\-5&4
    \end{pmatrix}=
    \begin{pmatrix}
     -9\cdot 1+2\cdot (-5)&-9\cdot 0+2\cdot 4\\3\cdot 1+7\cdot (-5)&3\cdot 0 +7\cdot 4
    \end{pmatrix}=
    \begin{pmatrix}
     -19&8\\-32&28
    \end{pmatrix}.
  \end{align*}

  \begin{align*}
    \begin{pmatrix}
      -9&2\\3&7
    \end{pmatrix}
    \begin{pmatrix}
      1\\-5
    \end{pmatrix}=
    \begin{pmatrix}
     -9\cdot 1+2\cdot (-5)\\3\cdot 1+7\cdot (-5)
    \end{pmatrix}=
    \begin{pmatrix}
     -19\\-32
    \end{pmatrix}.
  \end{align*}


  
  \begin{align*}
    \begin{pmatrix}
      -9&2
    \end{pmatrix}
    \begin{pmatrix}
      1\\-5
    \end{pmatrix}
    =
    \begin{pmatrix}
      (-9)\cdot 1 + 2\cdot (-5)
    \end{pmatrix}
    =
    \begin{pmatrix}
      -19
    \end{pmatrix}.
  \end{align*}

  \begin{align*}
    \begin{pmatrix}
      1\\-5
    \end{pmatrix}
    \begin{pmatrix}
      -9&2
    \end{pmatrix}
    =
    \begin{pmatrix}
      1\cdot (-9)& 1\cdot 2\\-5\cdot 1&-5\cdot 2
    \end{pmatrix}
    =
    \begin{pmatrix}
      -9& 2\\-5&-10
    \end{pmatrix}.
  \end{align*}

  \begin{align*}
    \begin{pmatrix}
      1&-3\\2&5
    \end{pmatrix}^2
    &=
    \begin{pmatrix}
      1&-3\\2&5
    \end{pmatrix}
    \begin{pmatrix}
      1&-3\\2&5
    \end{pmatrix}\\
   & =
    \begin{pmatrix}
      1\cdot 1+(-3)\cdot 2&1\cdot(-3)+(-3)\cdot 5\\2\cdot1 +5\cdot2&2\cdot(-3) +5\cdot5
    \end{pmatrix}
    =
    \begin{pmatrix}
      -5&-18\\12&19
    \end{pmatrix}.
  \end{align*}

%%   $A^2$に関する結果は次の様に検算をすることができる:
%%     \begin{align*}
%%     A=
%%     \begin{pmatrix}
%%       1&-3\\2&5
%%     \end{pmatrix}
%%   \end{align*}
%%     とおくと,
%%   \cref{thm:cht:2dim}から,
%%   \begin{align*}
%%     A^2-(1+5)A+(1\cdot 5-(-3)\cdot2)E_2&=O_{2,2}\\
%%     A^2-6A+11E_2&=O_{2,2}\\
%%     A^2&=6A-11E_2.
%%   \end{align*}
%%   となる.
%%   したがって,
%%   \begin{align*}
%%     A^2&=6A-11E_2=
%%    6\begin{pmatrix}
%%       1&-3\\2&5
%%    \end{pmatrix}
%%    -11
%%   \begin{pmatrix}
%%       1&0\\0&1
%%   \end{pmatrix}
%%   =
%%   \begin{pmatrix}
%%       6-11&-18\\12&30-11
%%    \end{pmatrix}
%%   =
%%   \begin{pmatrix}
%%       -5&-18\\12&19
%%    \end{pmatrix}.
%%   \end{align*}
\end{answerof}

\begin{answerof}{quiz:1:7}
  \begin{align*}
    A=\begin{pmatrix}
      9&0\\0&5
    \end{pmatrix}
  \end{align*}
  とし,
  \begin{align*}
    \begin{pmatrix}
      a_n&b_n\\c_n&d_n
    \end{pmatrix}
    =A^n
  \end{align*}
  とする.
  このとき,
  \begin{align*}
    \begin{pmatrix}
      a_{n+1}&b_{n+1}\\c_{n+1}&d_{n+1}
    \end{pmatrix}
    &=A^{n+1}\\
    &=AA^{n}\\
    &=
    \begin{pmatrix}
      9&0\\0&5
    \end{pmatrix}
    \begin{pmatrix}
      a_n&b_n\\c_n&d_n
    \end{pmatrix}\\
    &=
    \begin{pmatrix}
      9a_n+0c_n&9b_n+0d_n\\0a_n+5c_n&0b_n+5d_n
    \end{pmatrix}
    =
    \begin{pmatrix}
      9a_n&9b_n\\5c_n&5d_n
    \end{pmatrix}
  \end{align*}
  となるので,
  \begin{align*}
    \begin{pmatrix}
      a_{n+1}&b_{n+1}\\c_{n+1}&d_{n+1}
    \end{pmatrix}
    =
    \begin{pmatrix}
      9a_n&9b_n\\5c_n&5d_n
    \end{pmatrix}.
  \end{align*}
  よって,
  \begin{align*}
    a_n&=
    \begin{cases}
      9a_{n-1} &(n>1)\\
      9&(n=1)
    \end{cases}\\
    b_n&=
    \begin{cases}
      9b_{n-1}&(n>1)\\
      0&(n=1)
    \end{cases}\\
    c_n&=
    \begin{cases}
      5c_{n-1}&(n>1)\\
      0&(n=1)
    \end{cases}\\
    d_n&=
    \begin{cases}
      5d_{n-1}&(n>1)\\
      5&(n=1)
    \end{cases}
  \end{align*}
  となるので,
  \begin{align*}
    a_n&=9^{n-1}\cdot 9=9^n,\\
    b_n&=9^{n-1}\cdot 0=0,\\
    c_n&=5^{n-1}\cdot 0=0,\\
    d_n&=5^{n-1}\cdot 5=5^n.
  \end{align*}
  したがって,
  \begin{align*}
    A^n=\begin{pmatrix}9^n&0\\0&5^n\end{pmatrix}.
  \end{align*}

  この結果は次のように検算できる:
  $n=1$のとき
\begin{align*}
  A&=\begin{pmatrix}9&0\\0&5\end{pmatrix}\\
  \begin{pmatrix}9^n&0\\0&5^n\end{pmatrix}&=\begin{pmatrix}9&0\\0&5\end{pmatrix}.
\end{align*}
となり正しい.
  $n=2$のとき
\begin{align*}
  A^2&=\begin{pmatrix}9&0\\0&5\end{pmatrix}\begin{pmatrix}9&0\\0&5\end{pmatrix}
    =\begin{pmatrix}9\cdot 9+0\cdot 0&9\cdot 0+0\cdot5 \\0\cdot 9+5\cdot 0&0\cdot 0+5\cdot 5\end{pmatrix}
    =\begin{pmatrix}81&0\\0&25\end{pmatrix}
  \\
  \begin{pmatrix}9^2&0\\0&5^2\end{pmatrix}&=\begin{pmatrix}81&0\\0&25\end{pmatrix}.
\end{align*}
となり正しい.  
\end{answerof}


\begin{answerof}{quiz:1:8}
  \begin{align*}
    A=
    \begin{pmatrix}
      1&2\\0&3
    \end{pmatrix}
  \end{align*}
  とし,
  \begin{align*}
    \begin{pmatrix}
      a_n&b_n\\c_n&d_n
    \end{pmatrix}
    =A^n
  \end{align*}
  とする.
  このとき,
  \begin{align*}
    \begin{pmatrix}
      a_{n+1}&b_{n+1}\\c_{n+1}&d_{n+1}
    \end{pmatrix}
    &=A^{n+1}\\
    &=AA^{n}\\
    &=
    \begin{pmatrix}
      1&2\\0&3
    \end{pmatrix}
    \begin{pmatrix}
      a_n&b_n\\c_n&d_n
    \end{pmatrix}\\
    &=
    \begin{pmatrix}
      1a_n+2c_n&1b_n+2d_n\\0a_n+3c_n&0b_n+3d_n
    \end{pmatrix}
    =
    \begin{pmatrix}
      a_n+2c_n&b_n+2d_n\\3c_n&3d_n
    \end{pmatrix}
  \end{align*}
  となるので,
  \begin{align*}
    \begin{pmatrix}
      a_{n+1}&b_{n+1}\\c_{n+1}&d_{n+1}
    \end{pmatrix}
    =
    \begin{pmatrix}
      a_n+2c_n&b_n+2d_n\\3c_n&3d_n
    \end{pmatrix}.
  \end{align*}
  よって,
  \begin{align*}
    a_n&=
    \begin{cases}
      a_{n-1}+2c_{n-1} &(n>1)\\
      1&(n=1)
    \end{cases}\\
    b_n&=
    \begin{cases}
      b_{n-1}+2d_{n-1}&(n>1)\\
      2&(n=1)
    \end{cases}\\
    c_n&=
    \begin{cases}
      3c_{n-1}&(n>1)\\
      0&(n=1)
    \end{cases}\\
    d_n&=
    \begin{cases}
      3d_{n-1}&(n>1)\\
      3&(n=1)
    \end{cases}
  \end{align*}
  となる. したがって,
  \begin{align*}
    c_n&=3^{n-1}\cdot 0\\
    d_n&=3^{n-1}\cdot 3=3^n
  \end{align*}
  である.
  したがって,
  \begin{align*}
    a_n&=
    \begin{cases}
      a_{n-1}+2c_{n-1} &(n>1)\\
      1&(n=1)
    \end{cases}\\
    &=
    \begin{cases}
      a_{n-1} &(n>1)\\
      1&(n=1)
    \end{cases}\\
    b_n&=
    \begin{cases}
      b_{n-1}+2d_{n-1}&(n>1)\\
      2&(n=1)
    \end{cases}\\
  &=
    \begin{cases}
      b_{n-1}+2\cdot 3^{n-1}&(n>1)\\
      2&(n=1)
    \end{cases}
  \end{align*}
  となる.
  したがって,
  \begin{align*}
    a_n&=1^{n-1}\cdot 1=1
  \end{align*}
  となる. また,
  \begin{align*}
    b_n
    &=2\cdot 3^{n-1}+b_{n-1}\\
    &=2\cdot 3^{n-1}+2\cdot 3^{n-2}+b_{n-2}\\
    &=\cdots\\
    &=2\cdot 3^{n-1}+2\cdot 3^{n-2}+b_{n-2}+\cdots +2\cdot 3^{2}+b_2\\
    &=2\cdot 3^{n-1}+2\cdot 3^{n-2}+b_{n-2}+\cdots +2\cdot 3^{2}+2\cdot 3^1 +b_1\\
    &=2\cdot 3^{n-1}+2\cdot 3^{n-2}+b_{n-2}+\cdots +2\cdot 3^{2}+2\cdot 3^1 +2\\
    &=2(3^{n-1}+3^{n-2}+\cdots+3+1)\\
    &=2\frac{3^{n}-1}{3-1}\\
    &=2\frac{3^{n}-1}{2}\\
    &=3^{n}-1.
  \end{align*}
  したがって,
  \begin{align*}
    A^n=\begin{pmatrix}1&3^{n}-1\\0&3^n\end{pmatrix}.
  \end{align*}


  この結果は次のように検算できる:
  $n=1$のとき,
  \begin{align*}
    A^n&=A^1=
    \begin{pmatrix}
      1&2\\0&3
    \end{pmatrix}\\  
    \begin{pmatrix}1&2\cdot 3^{n}-1\cdot 2^{n+1}\\0&3^n\end{pmatrix}
      &=
      \begin{pmatrix}1&2\cdot 3^{1}-1\cdot 2^{2}\\0&3^1\end{pmatrix}
        =
    \begin{pmatrix}1&2\\0&3\end{pmatrix}      .
  \end{align*}
  となり正しい.
  \begin{align*}
    AA^n&=
    \begin{pmatrix}
      1&2\\0&3
    \end{pmatrix}
    \begin{pmatrix}1&3^{n}-1\\0&3^n\end{pmatrix}\\
      &=
    \begin{pmatrix}1\cdot 1+2\cdot 0&1\cdot(3^{n}-1)+2\cdot 3^n\\0\cdot 1+3\cdot 0&0\cdot (3^{n}-1)+3\cdot 3^n\end{pmatrix}\\
      &=
    \begin{pmatrix}1\cdot 1+2\cdot 0&1\cdot(3^{n}-1)+2\cdot 3^n\\0\cdot 1+3\cdot 0&0\cdot (3^{n}-1)+3\cdot 3^n\end{pmatrix}\\
      &=
    \begin{pmatrix}1&3^{n}-1+ 2\cdot 3^n\\3^{n+1}\end{pmatrix}\\
      &=
    \begin{pmatrix}1&3\cdot 3^{n}-1\\3^{n+1}\end{pmatrix}\\
      &=
    \begin{pmatrix}1&3^{n+1}-1\\3^{n+1}\end{pmatrix}\\
    A^{n+1}&=\begin{pmatrix}1&3^{n+1}-1\\0&3^{n+1}\end{pmatrix}.
  \end{align*}
  となり等しい.
\end{answerof}

%% \begin{answerof}{quiz:1:9}
%%  \Cref{thm:cht:2dim}から,
%%   \begin{align*}
%%     A^2-(1+5)A+(1\cdot 5-(-3)\cdot2)E_2&=O_{2,2}\\
%%     A^2-6A+11E_2&=O_{2,2}\\
%%     A^2&=6A-11E_2.
%%   \end{align*}
%%   したがって
%%   \begin{align*}
%%     A^3=A(6A-11E_2)=6A^2-11A=6(6A-11E_2)-11A=25A-66E_2.
%%   \end{align*}
%%   よって
%%   \begin{align*}
%%     A^3+A^2+A&=(25A-66E_2)+(6A-11E_2)+A\\
%%     &=32A-77E_2\\
%%     &=32\begin{pmatrix}1&-3\\2&5\end{pmatrix}-77\begin{pmatrix}1&0\\0&1\end{pmatrix}\\
%%     &=\begin{pmatrix}32-77&32\cdot(-3)\\32\cdot 2&32\cdot 5-77\end{pmatrix}\\
%%     &=\begin{pmatrix}-45&-96\\64&83\end{pmatrix}
%%   \end{align*}

%%   また次のようにも計算できる:
%%   \begin{align*}
%%     \begin{pmatrix}
%%       1&-3\\2&5
%%     \end{pmatrix}^2
%%     &=
%%     \begin{pmatrix}
%%       1&-3\\2&5
%%     \end{pmatrix}
%%     \begin{pmatrix}
%%       1&-3\\2&5
%%     \end{pmatrix}\\
%%     &=
%%     \begin{pmatrix}
%%       1\cdot 1+(-3)\cdot 2&1\cdot(-3)+(-3)\cdot 5\\2\cdot1 +5\cdot2&2\cdot(-3) +5\cdot5
%%     \end{pmatrix}
%%     =
%%     \begin{pmatrix}
%%       -5&-18\\12&19
%%     \end{pmatrix}.
%%   \\%\end{align*}
%%   %\begin{align*}
%%     \begin{pmatrix}
%%       1&-3\\2&5
%%     \end{pmatrix}^3
%%     &=
%%     \begin{pmatrix}
%%       1&-3\\2&5
%%     \end{pmatrix}
%%     \begin{pmatrix}
%%       1&-3\\2&5
%%     \end{pmatrix}^2\\
%%     &=
%%     \begin{pmatrix}
%%       1&-3\\2&5
%%     \end{pmatrix}
%%     \begin{pmatrix}
%%       -5&-18\\12&19
%%     \end{pmatrix}\\
%%     &=
%%     \begin{pmatrix}
%%       1\cdot(-5)+(-3)\cdot 12&1\cdot(-18)+(-3)\cdot 19\\2\cdot(-5) +5\cdot12&2\cdot(-18) +5\cdot19
%%     \end{pmatrix}
%%     =
%%     \begin{pmatrix}
%%       -41&-75\\50&59
%%     \end{pmatrix}.
%%   \\%\end{align*}
%%   %\begin{align*}
%%     A^3+A^2+A&=
%%     \begin{pmatrix}
%%       -41&-75\\50&59
%%     \end{pmatrix}
%%     +
%%     \begin{pmatrix}
%%       -5&-18\\12&19
%%     \end{pmatrix}
%%     +
%%     \begin{pmatrix}
%%       1&-3\\2&5
%%     \end{pmatrix}\\
%%     &=
%%     \begin{pmatrix}
%%       -41-5+1&-75-18-3\\50+12+2&59+19+5
%%     \end{pmatrix}
%%     =
%%     \begin{pmatrix}
%%       -45&-96\\64&83
%%     \end{pmatrix}.    
%%   \end{align*}
%% \end{answerof}


\begin{answerof}{quiz:1:10}
  \begin{align*}
    a\begin{pmatrix}
      1\\1
    \end{pmatrix}
    +b\begin{pmatrix}
      -1\\0
    \end{pmatrix}
    =
    \begin{pmatrix}
      a-b\\a
    \end{pmatrix}
  \end{align*}
  であるので,
  \begin{align*}
    \begin{pmatrix}
      -3\\1
    \end{pmatrix}
    =
    a\begin{pmatrix}
      1\\1
    \end{pmatrix}
    +b\begin{pmatrix}
      -1\\0
    \end{pmatrix}
  \end{align*}
  とすると,
  \begin{align*}
    \begin{cases}
      a-b=-3\\a=1
    \end{cases}
  \end{align*}
  を得る.
  したがって
  $a=1$, $b=4$を得る.
  実際,
  $(a,b)=(1,4)$とすると,
  \begin{align*}
    a\begin{pmatrix}
      1\\1
    \end{pmatrix}
    +b\begin{pmatrix}
      -1\\0
    \end{pmatrix}
    =
    \begin{pmatrix}
      1\\1
    \end{pmatrix}
    +4\begin{pmatrix}
      -1\\0
    \end{pmatrix}
    =
    \begin{pmatrix}
      -3\\1
    \end{pmatrix}
  \end{align*}
  となる.
\end{answerof}

\section{逆行列に関する問題}

\begin{answerof}{quiz:2:3}
  \begin{align*}
    \begin{pmatrix}
      1&0\\0&1
    \end{pmatrix}
    \begin{pmatrix}
      1&0\\0&1
    \end{pmatrix}=
    \begin{pmatrix}
      1&0\\0&1
    \end{pmatrix}
    =E_2
  \end{align*}
  であるので,
  \begin{align*}
      \begin{pmatrix}
      1&0\\0&1
    \end{pmatrix}
  \end{align*}
  の逆行列は
  \begin{align*}
      \begin{pmatrix}
      1&0\\0&1
    \end{pmatrix}
  \end{align*}
  である.
\end{answerof}

\begin{answerof}{quiz:2:11}
  $A^3+A^2+A+E_2=O_{2,2}$を満たすとすると,
  \begin{align*}
    -A^3-A^2-A=E_2
  \end{align*}
  を満たす.
  したがって, $B=-(A^2+A+E_2)$
  とおくと,
  \begin{align*}
  AB&=A(-A^2-A-E_2)=-A^3-A^2-A=E_2\\
  BA&=(-A^2-A-E_2)A=-A^3-A^2-A=E_2
  \end{align*}
  をみたすので,
  $B$は$A$の逆行列である.
  したがって, $A$は正則である.
\end{answerof}


\begin{answerof}{quiz:3:1}
  \begin{align*}
    \begin{pmatrix}
      1&3&1\\2&6&5
    \end{pmatrix}
  \end{align*}
  の2行目に1行目の$-2$倍を足すと,
  \begin{align*}
    \begin{pmatrix}
      1&3&1\\0&0&3
    \end{pmatrix}
  \end{align*}
  となる. さらに, 2行目を$\frac{1}{3}$倍すると,
  \begin{align*}
    \begin{pmatrix}
      1&3&1\\0&0&1
    \end{pmatrix}
  \end{align*}
  となる. さらに, 1行目に2行目の$-1$倍を足すと,
  \begin{align*}
    \begin{pmatrix}
      1&3&0\\0&0&1
    \end{pmatrix}
  \end{align*}
  となる.  これは, 階数$2$の被約階段行列である.
  したがって, 
  \begin{align*}
    \rank(
    \begin{pmatrix}
      1&3&1\\2&6&5
    \end{pmatrix}
    )=2
  \end{align*}
  である.


  \begin{align*}
    \begin{pmatrix}
      -5&1&2\\5&-1&-2
    \end{pmatrix}
  \end{align*}
  の$2$行目に$1$行目の$1$倍を加えると,
  \begin{align*}
    \begin{pmatrix}
      -5&1&2\\0&0&0
    \end{pmatrix}
  \end{align*}
  である. さらに$1$行目を$-\frac{1}{5}$倍すると
  \begin{align*}
    \begin{pmatrix}
      1&-\frac{1}{5}&-\frac{2}{5}\\0&0&0
    \end{pmatrix}
  \end{align*}
  である.
  これは, 階数$1$の被約階段行列である.
  よって
  \begin{align*}
    \rank(
    \begin{pmatrix}
      -5&1&2\\5&-1&-2
    \end{pmatrix}
    )=1
  \end{align*}
  である.

  \begin{align*}
    \begin{pmatrix}
      -5&1&2\\5&-1&2
    \end{pmatrix}
  \end{align*}
  の$2$行目に$1$行目の$1$倍を加えると,
  \begin{align*}
    \begin{pmatrix}
      -5&1&2\\0&0&4
    \end{pmatrix}
  \end{align*}
  である. さらに, 
  2行目を$\frac{1}{4}$倍すると,
  \begin{align*}
    \begin{pmatrix}
      -5&1&2\\0&0&1
    \end{pmatrix}
  \end{align*}
  である. さらに,
  1行目に2行目の$-2$倍を加えると,
  \begin{align*}
    \begin{pmatrix}
      -5&1&0\\0&0&1
    \end{pmatrix}
  \end{align*}
  である. さらに,
  1行目を$-\frac{1}{5}$倍すると
  \begin{align*}
    \begin{pmatrix}
      1&-\frac{1}{5}&0\\0&0&1
    \end{pmatrix}
  \end{align*}
  である.
  これは, 階数$2$の被約階段行列である.
  よって
  \begin{align*}
    \rank(
    \begin{pmatrix}
      -5&1&2\\5&-1&2
    \end{pmatrix}
    )=2
  \end{align*}
  である.


  \begin{align*}
    \begin{pmatrix}
      2&1&4\\5&6&8
    \end{pmatrix}
  \end{align*}
  の$2$行目に$1$行目の$-\frac{5}{2}$倍を加えると,
  \begin{align*}
    \begin{pmatrix}
      2&1&4\\0&\frac{7}{2}&-2
    \end{pmatrix}
  \end{align*}
  である. さらに,
  2行目を$\frac{7}{2}$倍すると
  \begin{align*}
    \begin{pmatrix}
      2&1&4\\0&1&-\frac{4}{7}
    \end{pmatrix}
  \end{align*}
  である. さらに,
  1行目に2行目の$-1$倍を加えると
  \begin{align*}
    \begin{pmatrix}
      2&0&\frac{32}{7}\\0&1&-\frac{4}{7}
    \end{pmatrix}
  \end{align*}
  である. さらに,
  2行目を$\frac{1}{2}$倍すると
  \begin{align*}
    \begin{pmatrix}
      1&0&\frac{16}{7}\\0&1&-\frac{4}{7}
    \end{pmatrix}
  \end{align*}
  である.
  これは, 階数$2$の被約階段行列である.
  よって
  \begin{align*}
    \rank(
    \begin{pmatrix}
      2&1&4\\5&6&8
    \end{pmatrix}
    )=2
  \end{align*}
  である.
\end{answerof}

%\endinput
\section{連立一次方程式に関する問題}
\begin{answerof}{quiz:3:2}
  \begin{align*}
    \begin{cases}
      2x+y=4\\
      5x+6y=8
      \end{cases}
  \end{align*}
  は,
  \begin{align*}
    \begin{pmatrix}
      2&1\\5&6
    \end{pmatrix}
    \begin{pmatrix}
      x\\y
    \end{pmatrix}
    =
    \begin{pmatrix}
      4\\8
    \end{pmatrix}
  \end{align*}
  とかける.
  この連立一次方程式の係数行列は,
  \begin{align*}
    \begin{pmatrix}
      2&1\\5&6
    \end{pmatrix}.
  \end{align*}
  この連立一次方程式の拡大係数行列は,
  \begin{align*}
    \begin{pmatrix}
      2&1&4\\5&6&8
    \end{pmatrix}.
  \end{align*}

  この連立方程式の拡大係数行列
  \begin{align*}
    \begin{pmatrix}
      2&1&4\\5&6&8
    \end{pmatrix}
  \end{align*}
  の$2$行目に$1$行目の$-\frac{5}{2}$倍を加えると,
  \begin{align*}
    \begin{pmatrix}
      2&1&4\\0&\frac{7}{2}&-2
    \end{pmatrix}
  \end{align*}
  である. さらに,
  2行目を$\frac{7}{2}$倍すると
  \begin{align*}
    \begin{pmatrix}
      2&1&4\\0&1&-\frac{4}{7}
    \end{pmatrix}
  \end{align*}
  である. さらに,
  1行目に2行目の$-1$倍を加えると
  \begin{align*}
    \begin{pmatrix}
      2&0&\frac{32}{7}\\0&1&-\frac{4}{7}
    \end{pmatrix}
  \end{align*}
  である. さらに,
  2行目を$\frac{1}{2}$倍すると
  \begin{align*}
    \begin{pmatrix}
      1&0&\frac{16}{7}\\0&1&-\frac{4}{7}
    \end{pmatrix}
  \end{align*}
  である.
  これを拡大係数行列にもつ連立方程式は,
  \begin{align*}
    \begin{cases}
      x=\frac{16}{7}\\
      y=-\frac{4}{7}
    \end{cases}
  \end{align*}
  である. したがって,
  解の空間$\FFF$は
  \begin{align*}
    \FFF=\Set{
    \begin{pmatrix}
      \frac{16}{7}\\-\frac{4}{7}
    \end{pmatrix}
    }
  \end{align*}
  である.

  解は, 次のように検算できる.
  \begin{align*}
    \begin{pmatrix}
      2&1\\5&6
    \end{pmatrix}
    \begin{pmatrix}
      \frac{16}{7}\\-\frac{4}{7}
    \end{pmatrix}
    =
    \begin{pmatrix}
      \frac{2\cdot 16-1\cdot 4}{7}\\\frac{5\cdot 16-6\cdot 4}{7}
    \end{pmatrix}
    =
    \begin{pmatrix}
      \frac{28}{7}\\\frac{56}{7}
    \end{pmatrix}
    =
    \begin{pmatrix}
      4\\8
    \end{pmatrix}.
  \end{align*}
\end{answerof}


\begin{answerof}{quiz:3:3}

  拡大係数行列
  \begin{align*}
    \begin{pmatrix}-5&1&2\\5&-1&-2\end{pmatrix}
  \end{align*}
  の2行目に1行目の1倍を足すと,
  \begin{align*}
    \begin{pmatrix}-5&1&2\\0&0&0\end{pmatrix}
  \end{align*}
  となる. 1行目を$\frac{-1}{5}$倍すると
  \begin{align*}
    \begin{pmatrix}1&\frac{-1}{5}&\frac{-2}{5}\\0&0&0\end{pmatrix}
  \end{align*}
  となる. つまり,
  \begin{align*}
    \begin{cases}
      x+\frac{-1}{5}y=\frac{-2}{5}\\
      0=0
    \end{cases}
  \end{align*}
  である.
  つまり,
  \begin{align*}
    \begin{cases}
      x=+\frac{1}{5}y+\frac{-2}{5}\\
      0=0
    \end{cases}
  \end{align*}
  である.
  1つ目の式から, $y=t$のとき
  $x=\frac{1}{5}t+\frac{-2}{5}$
  となるので,
  解の空間$\FFF$は,
  \begin{align*}
    \FFF=
    \Set{
     \begin{pmatrix}\frac{1}{5}t+\frac{-2}{5}\\t\end{pmatrix}
        |t\in \RR
        }.
  \end{align*}


  これは次の様に検算できる.
  $t=0$のとき,
  \begin{align*}
      \begin{pmatrix}x\\y\end{pmatrix}
        &=\begin{pmatrix}\frac{-2}{5}\\0\end{pmatrix}
  \end{align*}
  であるが,
  \begin{align*}
    \begin{pmatrix}-5&1\\5&-1\end{pmatrix}
      \begin{pmatrix}\frac{-2}{5}\\0\end{pmatrix}
        =
    \begin{pmatrix}-5\cdot(\frac{-2}{5})+1\cdot 0 \\5\cdot(\frac{-2}{5})-1\cdot 0\end{pmatrix}
        =
    \begin{pmatrix}2 \\-2\end{pmatrix}
  \end{align*}
  となる.
  
  $t=1$のとき,
  \begin{align*}
      \begin{pmatrix}x\\y\end{pmatrix}
        &=\begin{pmatrix}\frac{1}{5}1+\frac{-2}{5}\\1\end{pmatrix}
        =\begin{pmatrix}\frac{-1}{5}\\1\end{pmatrix}
  \end{align*}
  であるが,
  \begin{align*}
    \begin{pmatrix}-5&1\\5&-1\end{pmatrix}
      \begin{pmatrix}\frac{-1}{5}\\1\end{pmatrix}
        =
    \begin{pmatrix}-5\cdot(\frac{-1}{5})+1\cdot 1 \\5 \frac{-1}{5}-1\cdot 1 \end{pmatrix}
        =
    \begin{pmatrix}2 \\-2\end{pmatrix}.
  \end{align*}
  
\end{answerof}

\begin{answerof}{quiz:3:3x}
  拡大係数行列
  \begin{align*}
    \begin{pmatrix}-5&1&2\\5&-1&2\end{pmatrix}
  \end{align*}
  の2行目に1行目の1倍を足すと,
  \begin{align*}
    \begin{pmatrix}-5&1&2\\0&0&4\end{pmatrix}
  \end{align*}
  となる.
  2行目を$\frac{1}{4}$倍すると
  \begin{align*}
    \begin{pmatrix}-5&1&2\\0&0&1\end{pmatrix}
  \end{align*}
  となる.
  1行目を$\frac{-1}{5}$倍すると
  \begin{align*}
    \begin{pmatrix}1&\frac{-1}{5}&\frac{-2}{5}\\0&0&1\end{pmatrix}
  \end{align*}
  となる.
  1行目に2行目の$\frac{5}{2}$倍を加えると
  \begin{align*}
    \begin{pmatrix}1&\frac{-1}{5}&0\\0&0&1\end{pmatrix}
  \end{align*}
  となる.
  つまり,
  \begin{align*}
    \begin{cases}
      x+\frac{-1}{5}y=0\\
      0=1
    \end{cases}
  \end{align*}
  である.
  2つ目の式は$x$, $y$をどの様な値にしても成り立つことはないので,
  この連立方程式は解の空間は空集合である.
  つまり,
  \begin{align*}\FFF=\emptyset\end{align*}
\end{answerof}



\begin{answerof}{quiz:3:8} 
  \begin{align*}
    \xx=
    \begin{pmatrix}x\\y\end{pmatrix}
  \end{align*}
  とすると,
  \begin{align*}
     \begin{pmatrix}0&1\\0&0\end{pmatrix}
      \xx=
      \begin{pmatrix}0&1\\0&0\end{pmatrix}
        \begin{pmatrix}x\\y\end{pmatrix}
      = \begin{pmatrix}y\\0\end{pmatrix}  
  \end{align*}
  したがって,
  \begin{align*}
     \begin{pmatrix}0&1\\0&0\end{pmatrix}
      \xx&=\zzero\\
     \begin{pmatrix}y\\0\end{pmatrix}=\begin{pmatrix}0\\0\end{pmatrix}  
  \end{align*}
  であるので, $y=0$であれば$x$はどのような値でも良い.
  したがって,
  実数解の空間$\FFF$は
  \begin{align*}
    \FFF=
    \Set{\begin{pmatrix}t\\0\end{pmatrix}|t\in\RR}.
  \end{align*}
\end{answerof}

\begin{answerof}{quiz:3:9} 
  \begin{align*}
    \begin{pmatrix}-5&1\\5&-1\end{pmatrix}
  \end{align*}
  の$2$行目に$1$行目の$1$倍を足すと,
 \begin{align*}
    \begin{pmatrix}-5&1\\0&0\end{pmatrix}.
  \end{align*}
 $1$行目を$\frac{-1}{5}$倍すると
 \begin{align*}
    \begin{pmatrix}1&\frac{-1}{5}\\0&0\end{pmatrix}.
  \end{align*}
 したがって,
  \begin{align*}
    \begin{pmatrix}1&\frac{-1}{5}\\0&0\end{pmatrix}
    \xx=\zzero
  \end{align*}
  を解けば良い.
  \begin{align*}
    \xx=
    \begin{pmatrix}x\\y\end{pmatrix}
  \end{align*}
  とすると,
  \begin{align*}
    \begin{pmatrix}1&\frac{-1}{5}\\0&0\end{pmatrix}
      \xx
      =
    \begin{pmatrix}x+\frac{-1}{5}y\\0\end{pmatrix}
  \end{align*}
  であるので,
  \begin{align*}
    \begin{pmatrix}x+\frac{-1}{5}y\\0\end{pmatrix}=
      \begin{pmatrix}0\\0\end{pmatrix}
  \end{align*}
  をとく. $x-\frac{1}{5}y=0$であるので,
  $x=\frac{1}{5}y$であるから,
  $y=t$とすると, $x=\frac{t}{5}$である.
  よって, 実数解の空間$\FFF$は,
  \begin{align*}
    \FFF=
    \Set{\begin{pmatrix}\frac{t}{5}\\t\end{pmatrix}|t\in\RR}
  \end{align*}
  と書くことができる.  
\end{answerof}

\begin{answerof}{quiz:3:10} 
  \begin{align*}
    \begin{pmatrix}1&1\\5&5\end{pmatrix}
  \end{align*}
  の$2$行目に$1$行目の$-5$倍を足すと,
 \begin{align*}
    \begin{pmatrix}1&1\\0&0\end{pmatrix}.
  \end{align*}
 したがって,
  \begin{align*}
    \begin{pmatrix}1&1\\0&0\end{pmatrix}
    \xx=\zzero
  \end{align*}
  を解けば良い.
  \begin{align*}
    \xx=
    \begin{pmatrix}x\\y\end{pmatrix}
  \end{align*}
  とすると,
  \begin{align*}
    \begin{pmatrix}1&1\\0&0\end{pmatrix}
      \xx
      =
    \begin{pmatrix}x+y\\0\end{pmatrix}
  \end{align*}
  であるので,
  \begin{align*}
    \begin{pmatrix}x+y\\0\end{pmatrix}=
      \begin{pmatrix}0\\0\end{pmatrix}
  \end{align*}
  をとく. $x+y=0$であるので,
  $x=-y$であるから,
  $y=t$とすると, $x=-t$である.
  よって, 実数解の空間$\FFF$は,
  \begin{align*}
    \FFF=
    \Set{\begin{pmatrix}-t\\t\end{pmatrix}|t\in\RR}
  \end{align*}
  と書くことができる.  
\end{answerof}

\endinput
\section{行列式と逆行列に関する問題}

\begin{answerof}{quiz:2:1}
  \begin{align*}
    \det(\begin{pmatrix}
      2&1\\5&6
    \end{pmatrix})=2\cdot 6 -1\cdot 5=12-5=7.
\\
    \det(\begin{pmatrix}
      -5&1\\5&-1
    \end{pmatrix})=(-5)\cdot(-1) -1\cdot 5=5-5=0.
\end{align*}
\end{answerof}


\begin{answerof}{quiz:2:2}
  \begin{align*}
    \det(
    \begin{pmatrix}
      -4-x&1\\5&-x
    \end{pmatrix})=
    (-4-x)\cdot (-x) -1\cdot 5=x^2+4x-5=(x+5)(x-1)
  \end{align*}
  であるので
  \begin{align*}
    \det(\begin{pmatrix}
      -4-x&1\\5&-x
    \end{pmatrix})=0
  \end{align*}
  とすると,
  \begin{align*}
    (x+5)(x-1)&=0
  \end{align*}
  である.
  よって, $x\in \Set{-5,1}$.
\end{answerof}




\begin{answerof}{quiz:2:reg:1}
  \begin{align*}
    \det(\begin{pmatrix}
      2&1\\5&6
    \end{pmatrix})=2\cdot 6 -1\cdot 5=12-5=7\neq 0.
  \end{align*}
  であるので,
  \begin{align*}
    \begin{pmatrix}
      2&1\\5&6
    \end{pmatrix}
  \end{align*}
  は正則である.

  これについては, 次のように検算をできる:
  正則であるので逆行列が存在する.
  \begin{align*}
     \begin{pmatrix}
      2&1\\5&6
     \end{pmatrix}^{-1}=
     \frac{1}{7}
     \begin{pmatrix}
      6&-1\\-5&2
    \end{pmatrix}
  \end{align*}
  である.
  したがって,
  \begin{align*}
     \begin{pmatrix}
      2&1\\5&6
     \end{pmatrix}^{-1}
     \begin{pmatrix}
      2&1\\5&6
    \end{pmatrix}
    & =
     \frac{1}{7}
     \begin{pmatrix}
      6&-1\\-5&2
    \end{pmatrix}
    \begin{pmatrix}
      2&1\\5&6
    \end{pmatrix}\\
   & =
    \frac{1}{7}
     \begin{pmatrix}
       6\cdot 2+(-1)\cdot 5&6\cdot 1+(-1)\cdot 6\\
       -5\cdot 2+2\cdot 5&-5\cdot 1+2\cdot 6
    \end{pmatrix}
    =
    \frac{1}{7}
     \begin{pmatrix}
       7&0\\
       0&7
    \end{pmatrix}
    =
     \begin{pmatrix}
       1&0\\
       0&1
    \end{pmatrix}.
  \end{align*}
  
  \begin{align*}
    \det(\begin{pmatrix}
      -5&1\\5&-1
    \end{pmatrix})=(-5)\cdot(-1) -1\cdot 5=5-5=0.
  \end{align*}
  であるので
  \begin{align*}
    \begin{pmatrix}
      -5&1\\5&-1
    \end{pmatrix}
  \end{align*}
  は正則ではない.
\end{answerof}


\begin{answerof}{quiz:2:4}
  \begin{align*}
    A&=\begin{pmatrix}
      2&1\\5&6
    \end{pmatrix},&
    \bbb&=\begin{pmatrix}
      4\\8
    \end{pmatrix}.   
  \end{align*}
  とする.
\begin{enumerate}
\item
  \begin{align*}
    \det(A)=
    \det(\begin{pmatrix}
      2&1\\5&6
    \end{pmatrix})=2\cdot 6 -1\cdot 5=12-5=7\neq 0.
  \end{align*}
  であるので,
  $A$は正則であり,
  \begin{align*}
     A^{-1}=
     \frac{1}{7}
     \begin{pmatrix}
      6&-1\\-5&2
    \end{pmatrix}
  \end{align*}
  である.

  
  これは次のように検算できる.
  もとの行列と求めた行列をかけると
  \begin{align*}
    A^{-1}A
     &=
     \frac{1}{7}
     \begin{pmatrix}
      6&-1\\-5&2
    \end{pmatrix}
    \begin{pmatrix}
      2&1\\5&6
    \end{pmatrix}\\
    &=
    \frac{1}{7}
     \begin{pmatrix}
       6\cdot 2+(-1)\cdot 5&6\cdot 1+(-1)\cdot 6\\
       -5\cdot 2+2\cdot 5&-5\cdot 1+2\cdot 6
    \end{pmatrix}
    =
    \frac{1}{7}
     \begin{pmatrix}
       7&0\\
       0&7
    \end{pmatrix}
    =
     \begin{pmatrix}
       1&0\\
       0&1
    \end{pmatrix}.
  \end{align*}
  となり, $E_2$となる.

\item
  $A$が正則なので解は$A^{-1}\bbb$である.
  したがって,
  \begin{align*}
    \frac{1}{7}\begin{pmatrix}
      6&-1\\-5&2
    \end{pmatrix}
    \begin{pmatrix}
      4\\8
    \end{pmatrix}
    =
    \frac{1}{7}\begin{pmatrix}
      6\cdot 4 -1\cdot 8\\
      -5\cdot 4 +2\cdot 8\\
    \end{pmatrix}
    =
    \frac{1}{7}\begin{pmatrix}
      16\\-4
    \end{pmatrix}
    =
    \begin{pmatrix}
      \frac{16}{7}\\-\frac{4}{7}
    \end{pmatrix}
  \end{align*}

  これは次のように検算できる.
  \begin{align*}
    A\xx=
    \begin{pmatrix}
      2&1\\5&6
    \end{pmatrix}
    \begin{pmatrix}
      \frac{16}{7}\\-\frac{4}{7}
    \end{pmatrix}
    =
    \begin{pmatrix}
      2\frac{16}{7}+1\frac{-4}{7}\\5\frac{16}{7}+6\frac{-4}{10}
    \end{pmatrix}
    =
    \begin{pmatrix}
      \frac{28}{7}\\\frac{56}{7}
    \end{pmatrix}
    =
    \begin{pmatrix}
      4\\8
    \end{pmatrix}
  \end{align*}
  となり$\bbb$となる.
\end{enumerate}
\end{answerof}


\endinput
\section{行列と連立方程式に関する問題}


\begin{answerof}{quiz:3:4}
  係数行列は,
  \begin{align*}
    A=\begin{pmatrix}-5&1\\5&-1\end{pmatrix}
  \end{align*}
  であり,
  拡大係数行列は
  \begin{align*}
    B=\begin{pmatrix}-5&1&a\\5&-1&b\end{pmatrix}
  \end{align*}
  である.
  拡大係数行列
  \begin{align*}
    B=\begin{pmatrix}-5&1&a\\5&-1&b\end{pmatrix}
  \end{align*}
  の2行目に1行目の1倍を足すと,
  \begin{align*}
    \begin{pmatrix}-5&1&a\\0&0&a+b\end{pmatrix}
  \end{align*}
  となる. 1行目を$\frac{-1}{5}$倍すると
  \begin{align*}
    C=\begin{pmatrix}1&\frac{-1}{5}&\frac{-a}{5}\\0&0&a+b\end{pmatrix}
  \end{align*}
  となる.
  係数行列$A$の階数は1であるが,
  拡大係数行列$B$の階数を求めるには,
  $a+b$の値によって場合分けをする必要がある.

  $a+b=0$のとき,
  $C$は,
  \begin{align*}
    C=\begin{pmatrix}1&\frac{-1}{5}&\frac{-a}{5}\\0&0&0\end{pmatrix}
  \end{align*}
  となるので, 拡大係数行列の階数は1となる.
  係数行列の階数と拡大係数行列の階数が等しいので,
  このとき, 考えている連立方程式は解をもつ.

  $a+b\neq 0$のとき,
    $C$の2行目を$\frac{1}{a+b}$倍すると,
  \begin{align*}
    \begin{pmatrix}1&\frac{-1}{5}&\frac{-a}{5}\\0&0&1\end{pmatrix}
  \end{align*}
  である.
  1行目に2行目の$\frac{a}{5}$倍を足すことで,
  \begin{align*}
    \begin{pmatrix}1&\frac{-1}{5}&0\\0&0&1\end{pmatrix}
  \end{align*}
  という被約階段行列を得ることができる.
  したがって拡大係数行列の階数は2であり,
  $\rank(A)<\rank(B)$であるので,
  このときは解を持たない.
\end{answerof}


\begin{answerof}{quiz:3:4x}
  係数行列は,
  \begin{align*}
    A=\begin{pmatrix}2&1\\5&6\end{pmatrix}
  \end{align*}
  である.
  \begin{align*}
    \det(A)=\det(\begin{pmatrix}2&1\\5&6\end{pmatrix})=2\cdot6-1\cdot 5=7\neq 0
  \end{align*}
  であるので, $A$は正則である.
  したがって,
  \begin{align*}
      A^{-1}\begin{pmatrix}a\\b\end{pmatrix}
  \end{align*}
  は連立一次方程式
  \begin{align*}
    \begin{pmatrix}2&1\\5&6\end{pmatrix}
    \begin{pmatrix}x\\y\end{pmatrix}
      =
      \begin{pmatrix}a\\b\end{pmatrix}
  \end{align*}
  の解である.
  したがって,
  いつでも,
  この連立方程式は解を持つ.
\end{answerof}

 

\begin{answerof}{quiz:3:5}
  係数行列は,
  \begin{align*}
    A=\begin{pmatrix}-5&1\\5&a\end{pmatrix}
  \end{align*}
  であり,
  拡大係数行列は
  \begin{align*}
    B=\begin{pmatrix}-5&1&2\\5&a&b\end{pmatrix}
  \end{align*}
  である.

  拡大係数行列
  \begin{align*}
    B=\begin{pmatrix}-5&1&2\\5&a&b\end{pmatrix}
  \end{align*}
  の2行目に1行目の1倍を足すと,
  \begin{align*}
    C=\begin{pmatrix}-5&1&2\\0&a+1&b+2\end{pmatrix}
  \end{align*}
  となる.
  $a+1$の値で場合分けをする必要がある.

  $a+1\neq 0$のとき,
  \begin{align*}
    C=\begin{pmatrix}-5&1&2\\0&a+1&b+2\end{pmatrix}
  \end{align*}
  2行目に$\frac{1}{a+1}$をかけると,
  \begin{align*}
    \begin{pmatrix}-5&1&2\\0&1&\frac{b+2}{a+1}\end{pmatrix}
  \end{align*}
  となる.
  1行目に2行目の$-1$倍を加えると
  \begin{align*}
    \begin{pmatrix}-5&0&2-\frac{b+2}{a+1}\\0&1&\frac{b+2}{a+1}\end{pmatrix}
  \end{align*}
  となる.
  したがって, この場合は$\rank(A)=\rank(B)$であるので,
  考えている方程式は解をもつ.


  $a+1=0$のとき,
  \begin{align*}
    C=\begin{pmatrix}-5&1&2\\0&0&b+2\end{pmatrix}
  \end{align*}
  であるので, $\rank(A)=1$である.
  $\rank(B)$を求めるには, $b+2$の値で更に場合分けをする必要がある.
  $b+2=0$であれば,
  \begin{align*}
    C=\begin{pmatrix}-5&1&2\\0&0&0\end{pmatrix}
  \end{align*}
  であるので, $\rank(B)=1$であり, $\rank(A)=\rank(B)$であるから,
  考えている方程式は解をもつ.
  $b+2\neq 0$であれば,
  $C$の2行目を$\frac{1}{b+2}$倍することで,
  \begin{align*}
    \begin{pmatrix}-5&1&2\\0&0&1\end{pmatrix}
  \end{align*}
  を得る. さらに1行目に2行目の$-1$倍を加え
  \begin{align*}
    \begin{pmatrix}-5&1&0\\0&0&1\end{pmatrix}
  \end{align*}
  という被約階段行列を得る.
  したがって, $\rank(B)=2$である.
  $\rank(A)<\rank(B)$であるので,
  この場合は
  考えている方程式は解を持たない.
\end{answerof}

\begin{answerof}{quiz:3:6}
  \seealsoquiz{quiz:2:2}
  
  係数行列は,
  \begin{align*}
    A=    \begin{pmatrix}-4-a&1\\5&-a\end{pmatrix}
  \end{align*}
  である.
  $A$が正則であれば,
  解は
  $A^{-1}\zzero=\zzero$
  に限られる.
  $A$が正則ではないことと$\det(A)=0$は同値であるので,
  $\det(A)$を計算する.
  \begin{align*}
    \det(A)=\det(\begin{pmatrix}-4-a&1\\5&-a\end{pmatrix})
     =-a(-4-a)-5=a^2+4a-5=(a+5)(a-1)
  \end{align*}
  であるので$\det(A)=0$となるのは$a\in\Set{1,-5}$のときである.

  実際$a=1$のとき,
  \begin{align*}
    A=\begin{pmatrix}-4-1&1\\5&-1\end{pmatrix}
    =\begin{pmatrix}-5&1\\5&-1\end{pmatrix}
  \end{align*}
  であるので, 方程式$A\xx=\zzero$の拡大係数行列は,
  \begin{align*}
    \begin{pmatrix}-5&1&0\\5&-1&0\end{pmatrix}
  \end{align*}
  であるが, 2行目に1行目の$-1$倍を足すことで,
  \begin{align*}
    \begin{pmatrix}-5&1&0\\0&0&0\end{pmatrix}
  \end{align*}
  となる. 1行目を$-\frac{1}{5}$倍することで,
  \begin{align*}
    \begin{pmatrix}1&\frac{-1}{5}&0\\0&0&0\end{pmatrix}
  \end{align*}
  となる.  $x-\frac{1}{5}=0$であるので, $y=t$とすると, $x=\frac{t}{5}$である.
  よって解は
  \begin{align*}
    \begin{pmatrix}\frac{t}{5}\\t\end{pmatrix}
  \end{align*}
  とかける. 特に, $t=1$のとき,
  \begin{align*}
    \begin{pmatrix}\frac{1}{5}\\1\end{pmatrix}
  \end{align*}
  であり, これは第2成分が$0$ではないので, $\zzero$ではない.

  一方
  $a=-5$のとき,
  \begin{align*}
    A=\begin{pmatrix}-4-(-5)&1\\5&-(-5)\end{pmatrix}
    =\begin{pmatrix}1&1\\5&5\end{pmatrix}
  \end{align*}
  であるので, 方程式$A\xx=\zzero$の拡大係数行列は,
  \begin{align*}
    \begin{pmatrix}1&1&0\\5&5&0\end{pmatrix}
  \end{align*}
  であるが, 2行目に1行目の$-5$倍を足すことで,
  \begin{align*}
    \begin{pmatrix}1&1&0\\0&0&0\end{pmatrix}
  \end{align*}
  となる.
  $x+y=0$であるので, $y=t$とすると, $x=-t$である.
  よって解は
  \begin{align*}
    \begin{pmatrix}-t\\t\end{pmatrix}
  \end{align*}
  とかける. 特に, $t=1$のとき,
  \begin{align*}
    \begin{pmatrix}-1\\1\end{pmatrix}
  \end{align*}
  であり, これは第2成分が$0$ではないので, $\zzero$ではない.  
\end{answerof}





\begin{answerof}{quiz:3:7} 
\seealsoquiz{quiz:3:6}
  \begin{align*}
    \begin{pmatrix}-4&1\\5&0\end{pmatrix}
    \begin{pmatrix}x\\y\end{pmatrix}
      =
      a\begin{pmatrix}x\\y\end{pmatrix}
  \end{align*}
  は移項すると,
  \begin{align*}
    \begin{pmatrix}-4&1\\5&0\end{pmatrix}
      \begin{pmatrix}x\\y\end{pmatrix}
        -
        a\begin{pmatrix}x\\y\end{pmatrix}
      =\zzero
  \end{align*}
  となるが,
  \begin{align*}
    \begin{pmatrix}-4&1\\5&0\end{pmatrix}
      \begin{pmatrix}x\\y\end{pmatrix}
        -
        a\begin{pmatrix}x\\y\end{pmatrix}
        &=
    \begin{pmatrix}-4&1\\5&0\end{pmatrix}
      \begin{pmatrix}x\\y\end{pmatrix}
        -
        (aE_2)\begin{pmatrix}x\\y\end{pmatrix}\\
        &=
        (\begin{pmatrix}-4&1\\5&0\end{pmatrix}
        -
        aE_2)\begin{pmatrix}x\\y\end{pmatrix}\\
        &=
        (\begin{pmatrix}-4&1\\5&0\end{pmatrix}
        -
        \begin{pmatrix}a&0\\0&a\end{pmatrix}
          )\begin{pmatrix}x\\y\end{pmatrix}\\
        &=
        \begin{pmatrix}-4-a&1\\5&0-a\end{pmatrix}
          \begin{pmatrix}x\\y\end{pmatrix}        
  \end{align*}
  と変形できるので,
  \begin{align*}
    \begin{pmatrix}-4-a&1\\5&0-a\end{pmatrix}
    \begin{pmatrix}x\\y\end{pmatrix}
      =
      \zzero
  \end{align*}
  と書き直せる.
  したがって,
  求めるべきものは,
  連立方程式
  \begin{align*}
    \begin{pmatrix}-4-a&1\\5&0-a\end{pmatrix}
    \begin{pmatrix}x\\y\end{pmatrix}
      =
      \zzero
  \end{align*}
  が$\zzero$
  以外の
  解を持つための$a$に対する条件である.
  係数行列がが正則であれば,
  解は
  $\zzero$
  に限られる.
  係数行列が正則でなくなるための条件を調べるためにその行列式を計算する.
  \begin{align*}
    \det(\begin{pmatrix}-4-a&1\\5&0-a\end{pmatrix})
     =-a(-4-a)-5=a^2+4a-5=(a+5)(a-1)
  \end{align*}
  であるので$\det(A)=0$となるのは$a\in\Set{1,-5}$のときである.

  実際$a=1$のとき,
  考えている方程式の拡大係数行列は,
  \begin{align*}
    \begin{pmatrix}-5&1&0\\5&-1&0\end{pmatrix}
  \end{align*}
  であるが, 2行目に1行目の$-1$倍を足すことで,
  \begin{align*}
    \begin{pmatrix}-5&1&0\\0&0&0\end{pmatrix}
  \end{align*}
  となる. 1行目を$-\frac{1}{5}$倍することで,
  \begin{align*}
    \begin{pmatrix}1&\frac{-1}{5}&0\\0&0&0\end{pmatrix}
  \end{align*}
  となる.  $x-\frac{1}{5}=0$であるので, $y=t$とすると, $x=\frac{t}{5}$である.
  よって解は
  \begin{align*}
    \begin{pmatrix}\frac{t}{5}\\t\end{pmatrix}
  \end{align*}
  とかける. 特に, $t=1$のとき,
  \begin{align*}
    \begin{pmatrix}\frac{1}{5}\\1\end{pmatrix}
  \end{align*}
  であり, これは第2成分が$0$ではないので, $\zzero$ではない.

  一方
  $a=-5$のとき,
  考えている方程式の拡大係数行列は,
  \begin{align*}
    \begin{pmatrix}1&1&0\\5&5&0\end{pmatrix}
  \end{align*}
  であるが, 2行目に1行目の$-5$倍を足すことで,
  \begin{align*}
    \begin{pmatrix}1&1&0\\0&0&0\end{pmatrix}
  \end{align*}
  となる.
  $x+y=0$であるので, $y=t$とすると, $x=-t$である.
  よって解は
  \begin{align*}
    \begin{pmatrix}-t\\t\end{pmatrix}
  \end{align*}
  とかける. 特に, $t=1$のとき,
  \begin{align*}
    \begin{pmatrix}-1\\1\end{pmatrix}
  \end{align*}
  であり, これは第2成分が$0$ではないので, $\zzero$ではない.  
\end{answerof}


\section{平面の線形変換に関する問題}
\begin{answerof}{quiz:5:1}
  \seealsoquiz{quiz:2:1}
  2つの2項ベクトルの組が
  一次独立かどうかは, 行列式を調べることでわかる.
  
 \begin{align*}
     \det(\begin{pmatrix}-5&1\\5&-1\end{pmatrix})=-5\cdot(-1)-1\cdot(5)=0
 \end{align*}
 であるので
 \begin{align*}
   (\begin{pmatrix}-5\\5\end{pmatrix},\begin{pmatrix}1\\-1\end{pmatrix})
 \end{align*}
 は一次独立ではない.

 \begin{align*}
    \det(\begin{pmatrix}
      2&1\\5&6
    \end{pmatrix})=2\cdot 6 -1\cdot 5=12-5=7\neq 0.
 \end{align*}
 であるので,
 \begin{align*}
   (\begin{pmatrix}2\\5\end{pmatrix},\begin{pmatrix}1\\6\end{pmatrix})
 \end{align*}
は一次独立である.
\end{answerof}


\begin{answerof}{quiz:5:3}  
  2つの2項ベクトルの組が
  一次独立かどうかは, 行列式を調べることでわかる.
  \begin{align*}
    \det(\begin{pmatrix}-5&1\\a&b\end{pmatrix})
    =-5b-1a=-5b-a
  \end{align*}
  である.
  したがって,
  \begin{align*}
    (\begin{pmatrix}-5\\a\end{pmatrix},
    \begin{pmatrix}1\\b\end{pmatrix})
  \end{align*}
  は,
  行列式が$0$でないとき,
  つまり,
  $a\neq -5b$であるとき
  一次独立であり,
  行列式が$0$のとき,
  つまり,
  $a= -5b$であるとき
  一次独立ではない.
\end{answerof}

\begin{answerof}{quiz:5:2}
  \begin{align*}
    A&=\begin{pmatrix}a&b\\a'&b'\end{pmatrix}\\
    \aaa&=\begin{pmatrix}a\\a'\end{pmatrix}\\
    \bbb&=\begin{pmatrix}b\\b'\end{pmatrix}
  \end{align*}
  とする.
  $(\aaa,\bbb)$が$\RR^2$の生成系であるということは,
  どの$\pp\in\RR^2$に対しても,
  $x\aaa+y\bbb=\pp$を満たす$x,y\in\RR$が存在するということであった.
  \begin{align*}
    x\aaa+y\bbb=
    x\begin{pmatrix}a\\a'\end{pmatrix}+y\begin{pmatrix}b\\b'\end{pmatrix}
    =\begin{pmatrix}xa+yb\\xa'+yb'\end{pmatrix}
    =\begin{pmatrix}a&b\\a'&b'\end{pmatrix}\begin{pmatrix}x\\y\end{pmatrix}
    =A\begin{pmatrix}x\\y\end{pmatrix}
  \end{align*}
  と書き換えられるので,
  $(\aaa,\bbb)$が$\RR^2$の生成系であるということは,
  どの$\pp\in\RR^2$に対しても
  $A\xx=\pp$が実数解を持つことと言い換えられる.
  どの$\pp\in\RR^2$に対しても
  $A\xx=\pp$が実数解を持つならば,
  $(\aaa,\bbb)$が$\RR^2$の生成系である. 
  $A\xx=\pp$が実数解を持たないような
  $\pp\in\RR^2$が存在するなら,
  $(\aaa,\bbb)$が$\RR^2$の生成系ではない.
 

  \seealsoquiz{quiz:3:4x}
  \begin{align*}
    A=
    \begin{pmatrix}
      2&1\\5&6
    \end{pmatrix}
  \end{align*}
  とすると
  \begin{align*}
    \det(A)=
    \det(\begin{pmatrix}
      2&1\\5&6
    \end{pmatrix})=2\cdot 6 -1\cdot 5=12-5=7\neq 0.
 \end{align*}
  であるので, $A$は正則である.
  したがって, $\pp\in\RR^2$とすると,
  $A^{-1}\pp$は$A\xx=\pp$の解である.
  したがって,
  \begin{align*}
   (\begin{pmatrix}2\\5\end{pmatrix},\begin{pmatrix}1\\6\end{pmatrix})
  \end{align*}
  は$\RR^2$の生成系である.


  \seealsoquiz{quiz:3:4}
  \begin{align*}
    \begin{pmatrix}
      -5&1\\5&-1
    \end{pmatrix}
    \begin{pmatrix}
      x\\y
    \end{pmatrix}=
    \begin{pmatrix}
      p\\q
    \end{pmatrix}
  \end{align*}
  とする. この方程式の拡大係数行列は
  \begin{align*}
    \begin{pmatrix}
      -5&1&p\\5&-1&q
    \end{pmatrix}
  \end{align*}
  である.
  2行目に1行目の$1$倍を加えると
  \begin{align*}
    \begin{pmatrix}
      -5&1&p\\0&0&p+q
    \end{pmatrix}
  \end{align*}
  1行目を$\frac{-1}{5}$倍すると
  \begin{align*}
    \begin{pmatrix}
      1&\frac{-1}{5}&\frac{-p}{5}\\0&0&p+q
    \end{pmatrix}
  \end{align*}
  となる.
  したがって考えている連立方程式は,
  $p+q=0$なら解をもつ.
  しかし
  $p+q\neq 0$なら,
  さらに2行目を$\frac{1}{p+q}$倍すると,
  \begin{align*}
    \begin{pmatrix}
      1&\frac{-1}{5}&\frac{-p}{5}\\0&0&1
    \end{pmatrix}
  \end{align*}
  となり,
  さらに1行目に2行目の$\frac{p}{5}$倍を加えることで,
  \begin{align*}
    \begin{pmatrix}
      1&\frac{-1}{5}&0\\0&0&1
    \end{pmatrix}
  \end{align*}
  となるので,
  考えている連立方程式は
  解をもたない.
  例えば,
  \begin{align*}
    \begin{pmatrix}
      -5&1\\5&-1
    \end{pmatrix}
    \begin{pmatrix}
      x\\y
    \end{pmatrix}=
    \begin{pmatrix}0\\1\end{pmatrix}
  \end{align*}
  は実数解を持たないので,
  \begin{align*}
    (\begin{pmatrix}-5\\5\end{pmatrix},\begin{pmatrix}1\\-1\end{pmatrix})
  \end{align*}
  は$\RR^2$の生成系ではない.
\end{answerof}

\begin{answerof}{quiz:5:5}
  \begin{align*}
    \aaa=\begin{pmatrix}-1\\-1\end{pmatrix},
    \bbb=\begin{pmatrix}2\\-1\end{pmatrix}
  \end{align*}
  であるとき,

  \begin{align*}
    \Braket{\aaa,\bbb}
    &=\Braket{\begin{pmatrix}-1\\-1\end{pmatrix},\begin{pmatrix}2\\-1\end{pmatrix}}
    =-1\cdot2+(-1)(-1)=-2+1=-1.
    \\
    \Braket{\aaa,\aaa}
    &=\Braket{\begin{pmatrix}-1\\-1\end{pmatrix},\begin{pmatrix}-1\\-1\end{pmatrix}}
    =(-1)^2+(-1)^2=2.
\\
    \Braket{\bbb,\bbb}
    &=\Braket{\begin{pmatrix}2\\-1\end{pmatrix},\begin{pmatrix}2\\-1\end{pmatrix}}
    =(2)^2+(-1)^2=5.
    \\
    \|\aaa\|
    =\sqrt{\Braket{\aaa,\aaa}}
    =\sqrt{2}
\\
    \|\bbb\|
    =\sqrt{\Braket{\bbb,\bbb}}
    =\sqrt{5}
  \end{align*}
  $\Braket{\aaa,\bbb}=\left\|\aaa\right\|\left\|\bbb\right\|\cos(\theta)$であるので,
  \begin{align*}
    \cos(\theta)=\frac{\Braket{\aaa,\bbb}}{\left\|\aaa\right\|\left\|\bbb\right\|}
=\frac{-1}{\sqrt{2}\sqrt{5}}
=-\frac{1}{\sqrt{10}}.
  \end{align*}
\end{answerof}

\begin{answerof}{quiz:5:7}
  \begin{align*}
    \|\aaa-2\bbb\|^2
    &=\Braket{\aaa+2\bbb,\aaa+2\bbb}\\
    &=\Braket{\aaa,\aaa+2\bbb}+\Braket{2\bbb,\aaa+2\bbb}\\
    &=\Braket{\aaa,\aaa}+\Braket{\aaa,2\bbb}+\Braket{2\bbb,\aaa}+\Braket{2\bbb,2\bbb}\\
    &=\Braket{\aaa,\aaa}+2\Braket{\aaa,\bbb}+2\Braket{\bbb,\aaa}+4\Braket{\bbb,\bbb}\\
    &=\|\aaa\|^2+2\Braket{\aaa,\bbb}+2\Braket{\aaa,\bbb}+4\|\bbb\|^2\\
    &=\|\aaa\|^2+4\Braket{\aaa,\bbb}+4\|\bbb\|^2
  \end{align*}
  であるので,
  $\|\aaa-2\bbb\|^2=\|\aaa\|^2+4\Braket{\aaa,\bbb}+4\|\bbb\|^2$
  である.
  $\|\aaa\|=2$,
  $\|\bbb\|=3$,
  $\|\aaa-2\bbb\|=\sqrt{3}$
  であるから,
  \begin{align*}
    \|\aaa-2\bbb\|^2&=\|\aaa\|^2+4\Braket{\aaa,\bbb}+4\|\bbb\|^2\\
    \sqrt{3}^2&=2^2+4\Braket{\aaa,\bbb}+4\cdot 3^2\\
    3&=4+4\Braket{\aaa,\bbb}+36\\
    -4\Braket{\aaa,\bbb}&=4+36-3=37\\
    \Braket{\aaa,\bbb}&=-\frac{37}{4}\\
  \end{align*}
  である.
  $\Braket{\aaa,\bbb}=\left\|\aaa\right\|\left\|\bbb\right\|\cos(\theta)$であるので,
  \begin{align*}
    \cos(\theta)=\frac{\Braket{\aaa,\bbb}}{\left\|\aaa\right\|\left\|\bbb\right\|}
=\frac{37}{4\cdot 2\cdot 3}
=\frac{37}{\sqrt{24}}.
  \end{align*}
  
\end{answerof}

\begin{answerof}{quiz:5:6}
  \begin{align*}
    \|\aaa-2\bbb\|^2
    &=\Braket{\aaa+2\bbb,\aaa+2\bbb}\\
    &=\Braket{\aaa,\aaa+2\bbb}+\Braket{2\bbb,\aaa+2\bbb}\\
    &=\Braket{\aaa,\aaa}+\Braket{\aaa,2\bbb}+\Braket{2\bbb,\aaa}+\Braket{2\bbb,2\bbb}\\
    &=\Braket{\aaa,\aaa}+2\Braket{\aaa,\bbb}+2\Braket{\bbb,\aaa}+4\Braket{\bbb,\bbb}\\
    &=\|\aaa\|^2+2\Braket{\aaa,\bbb}+2\Braket{\aaa,\bbb}+4\|\bbb\|^2\\
    &=\|\aaa\|^2+4\Braket{\aaa,\bbb}+4\|\bbb\|^2
  \end{align*}
  であるので,
  $\|\aaa-2\bbb\|^2=\|\aaa\|^2+4\Braket{\aaa,\bbb}+4\|\bbb\|^2$
  である.
  $\|\aaa\|=2$,
  $\|\bbb\|=3$,
  $\|\aaa-2\bbb\|=\sqrt{3}$
  であるから,
  \begin{align*}
    \|\aaa-2\bbb\|^2&=\|\aaa\|^2+4\Braket{\aaa,\bbb}+4\|\bbb\|^2\\
    \sqrt{3}^2&=2^2+4\Braket{\aaa,\bbb}+4\cdot 3^2\\
    3&=4+4\Braket{\aaa,\bbb}+36\\
    -4\Braket{\aaa,\bbb}&=4+36-3=37\\
    \Braket{\aaa,\bbb}&=-\frac{37}{4}\\
  \end{align*}
  である.

  また,
  \begin{align*}
    \Braket{\aaa+t\bbb,\aaa+2\bbb}
    &=\Braket{\aaa,\aaa+2\bbb}+\Braket{t\bbb,\aaa+2\bbb}\\
    &=\Braket{\aaa,\aaa}+\Braket{\aaa,2\bbb}+\Braket{t\bbb,\aaa}+\Braket{t\bbb,2\bbb}\\
    &=\Braket{\aaa,\aaa}+2\Braket{\aaa,\bbb}+t\Braket{\bbb,\aaa}+2t\Braket{\bbb,\bbb}\\
    &=\|\aaa\|^2+2\Braket{\aaa,\bbb}+t\Braket{\aaa,\bbb}+2t\|\bbb\|^2\\
    &=\|\aaa\|^2+(2+t)\Braket{\aaa,\bbb}+2t\|\bbb\|^2
  \end{align*}
  である.
  $\|\aaa\|=2$,
  $\|\bbb\|=3$,
  $\Braket{\aaa,\bbb}=-\frac{37}{4}$であるので,
  \begin{align*}
    \Braket{\aaa+t\bbb,\aaa+2\bbb}
    &=\|\aaa\|^2+(2+t)\Braket{\aaa,\bbb}+2t\|\bbb\|^2\\
    &=2^2+-\frac{37}{4}(2+t)+ 3^2\cdot2t\\
    &=4-\frac{37}{2}-\frac{37}{4}t+18t\\
    &=\frac{8-37}{2}+\frac{72-37}{4}t\\
    &=\frac{-29}{2}+\frac{35}{4}t
  \end{align*}
  である.
  したがって,
  \begin{align*}
    \frac{-29}{2}+\frac{35}{4}t&=0\\
    \frac{35}{4}t&=\frac{29}{2}\\
    t&=\frac{29\cdot 2}{35}=\frac{58}{35}
  \end{align*}
  のとき,
  $\Braket{\aaa+t\bbb,\aaa+2\bbb}=0$となる.
\end{answerof}

\begin{answerof}{quiz:5:9}
  \begin{align*}
    \aaa=\frac{1}{5}\begin{pmatrix}3\\4\end{pmatrix},\\
    \bbb=\begin{pmatrix}x\\y\end{pmatrix},
  \end{align*}
  とおく.
  正規直交基底であるとすると,
  \begin{align}
    \|\aaa\|^2&=\left(\frac{3}{5}\right)^2+\left(\frac{4}{5}\right)^2=1\\
    \|\bbb\|^2&=x^2+y^2=1     \label{orthonorm:eq1}\\
    \Braket{\aaa,\bbb}&=\frac{3}{5}x+\frac{4}{5}y=0
    \label{orthonorm:eq2}
  \end{align}
  である. したがって, \cref{orthonorm:eq2}から,
  \begin{align*}
    x=-\frac{4}{3}y
  \end{align*}
  がわかるので, \cref{orthonorm:eq1}に代入して,
  \begin{align*}
    \left(-\frac{4}{3}y\right)^2+y^2&=1\\
    \frac{25}{9}y^2&=1\\
    y^2&=\frac{9}{25}
  \end{align*}
  となるので, $y\in\Set{\frac{3}{5},-\frac{3}{5}}$である.

  $y=\frac{3}{5}$のときには,
  $x=-\frac{4}{5}$で,
  \begin{align*}
    (\frac{1}{5}\begin{pmatrix}3\\4\end{pmatrix},
    \frac{1}{5}\begin{pmatrix}-4\\3\end{pmatrix})
  \end{align*}
  という正規直交基底が得られる.

  $y=-\frac{3}{5}$のときには,
  $x=\frac{4}{5}$で,
  \begin{align*}
    (\frac{1}{5}\begin{pmatrix}3\\4\end{pmatrix},
    \frac{1}{5}\begin{pmatrix}4\\-3\end{pmatrix})
  \end{align*}
  という正規直交基底が得られる.
\end{answerof}

\begin{answerof}{quiz:5:10}
  \begin{align*}
    A=\frac{1}{5}\begin{pmatrix}3&x\\4&y\end{pmatrix}
  \end{align*}
  とする.
  $A$が直交行列であるとすると,
  \begin{align*}
    \transposed{A}A=E_2
  \end{align*}
  である.
  \begin{align*}
    \transposed{A}A
    &=\frac{1}{25}\begin{pmatrix}3&4\\x&y\end{pmatrix}\begin{pmatrix}3&x\\4&y\end{pmatrix}\\
    &=\frac{1}{25}\begin{pmatrix}3^2+4^2&3x+4y\\3x+4y&x^2+\end{pmatrix}\\
    &=\begin{pmatrix}25&3x+4y\\3x+4y&x^2+y^2\end{pmatrix}
  \end{align*}
  であるので
  \begin{align}
    \frac{x^2+y^2}{25}&=1\label{orthomat:eq1}\\
    \frac{3}{25}x+\frac{4}{25}y&=0
    \label{orthomat:eq2}
  \end{align}
  である.
  \Cref{orthomat:eq2}より, 
  \begin{align*}
    y=-\frac{3}{4}x
  \end{align*}
  であるので, \cref{orthomat:eq1}に代入すると,
  \begin{align*}
    \frac{x^2+\left(-\frac{3}{4}x\right)^2}{25}&=1\\
    x^2+\left(-\frac{3}{4}x\right)^2&=25\\
    16x^2+9x^2&=400\\
    25x^2&=400\\
    x^2&=16
  \end{align*}
  であるので, $x\in\Set{4,-4}$である.

  $x=4$のときには,
  $y=-3$で,
  \begin{align*}
    \frac{1}{5}\begin{pmatrix}3&-4\\4&3\end{pmatrix}
  \end{align*}
  という直交行列が得られる.

  $x=-4$のときには,
  $y=3$で,
  \begin{align*}
    \frac{1}{5}\begin{pmatrix}3&4\\4&-3\end{pmatrix}
  \end{align*}
  という直交行列が得られる.


  また, $A$が直交行列であることと, $A\transposed{A}=E_2$を満たすことは同値であったので, 次のように解くこともできる.
  \begin{align*}
    A\transposed{A}
    &=\frac{1}{25}\begin{pmatrix}3&x\\4&y\end{pmatrix}\begin{pmatrix}3&4\\x&y\end{pmatrix}\\
    &=\frac{1}{25}\begin{pmatrix}3^2+x^2&3x+xy\\4\cdot 3+xy&4^2+y^2\end{pmatrix}
  \end{align*}
  であるので
  \begin{align}
    \frac{1}{25}(3^2+x^2)&=1\label{orthomat:eq4}\\
    \frac{1}{25}(3\cdot 4+xy)&=0\label{orthomat:eq5}\\
    \frac{1}{25}(4^2+y^2)&=1\label{orthomat:eq6}
  \end{align}
  である.
  \Cref{orthomat:eq4}より, 
  \begin{align*}
    9+x^2&=25\\
    x^2&=16
  \end{align*}
  であるので, $x\in\Set{4,-4}$.
    \Cref{orthomat:eq6}より, 
  \begin{align*}
    16+y^2&=25\\
    y^2&=9
  \end{align*}
  であるので, $y\in\Set{3,-3}$.

  $x=4$, $y=3$のとき,
  $\frac{1}{25}(12+xy)=\frac{24}{25}\neq 0$
  であり,
  \Cref{orthomat:eq5} を満たさない.

  $x=4$,
  $y=-3$のときには,
  \begin{align*}
    \frac{1}{5}\begin{pmatrix}3&-4\\4&3\end{pmatrix}
  \end{align*}
  という直交行列が得られる.

  $x=-4$,
  $y=3$のときには,
  \begin{align*}
    \frac{1}{5}\begin{pmatrix}3&4\\4&-3\end{pmatrix}
  \end{align*}
  という直交行列が得られる.
  
  $x=-4$, $y=-3$のとき,
  $\frac{1}{25}(12+xy)=\frac{24}{25}\neq 0$
  であり,
  \Cref{orthomat:eq5} を満たさない.
\end{answerof}

\begin{answerof}{quiz:5:p:1}
\begin{align*}
  \aaa&=\begin{pmatrix}7\\-4\end{pmatrix}&
  \bbb&=\begin{pmatrix}2\\-3\end{pmatrix}
\end{align*}
とする.
$\aaa$から$L$への直交射影を$\pp$
とすると,
\begin{align*}
  \pp&=t\bbb\\
  \Braket{\aaa-\pp,\bbb}&=0
\end{align*}
を満たすものである.
$\pp=t\bbb$であることから,
\begin{align*}
  \Braket{\aaa-\pp,\bbb}
  &=\Braket{\aaa-t\bbb,\bbb}\\
  &=\Braket{\aaa,\bbb}-\Braket{t\bbb,\bbb}\\
  &=\Braket{\aaa,\bbb}-t\Braket{\bbb,\bbb}\\
  &=\Braket{\begin{pmatrix}7\\-4\end{pmatrix},\begin{pmatrix}2\\-3\end{pmatrix}}-t\Braket{\begin{pmatrix}2\\-3\end{pmatrix},\begin{pmatrix}2\\-3\end{pmatrix}}\\
  &=7\cdot 2+(-4)(-3)-t(2^2+(-3)^2)\\
  &=26-13t
\end{align*}
である.
$\Braket{\aaa-\pp,\bbb}=0$であるので,
$26-13t=0$.
つまり, $t=2$である.
よって直交射影は
\begin{align*}
  \pp=2\begin{pmatrix}2&-3\end{pmatrix}=\begin{pmatrix}4&-6\end{pmatrix}
\end{align*}
である.
垂直成分を$\qq$とすると,
$\aaa=\pp+\qq$であるので,
\begin{align*}
  \qq=\aaa-\pp
  =\begin{pmatrix}7\\-4\end{pmatrix}-\begin{pmatrix}4&-6\end{pmatrix}
  =\begin{pmatrix}3\\2\end{pmatrix}
\end{align*}
である.

これは次の様に考えることもできる:
$\aaa$と$\bbb$のなす角を$\theta$とすると,
$\aaa$から$L$への直交射影$\pp$は,
$\bbb$のスカラー倍でそのノルムが$|\cos(\theta)|\|\aaa\|$
であるものである.
このようなベクトルは
\begin{align*}
  \vv=\frac{\cos(\theta)\|\aaa\|}{\|\bbb\|}\bbb,
  \vv'=-\frac{\cos(\theta)\|\aaa\|}{\|\bbb\|}\bbb
\end{align*}
の2つがあるので,
どちらが$\pp$である.
一方, $\Braket{\aaa,\bbb}=\|\aaa\|\|\bbb\|\cos(\theta)$
であるから,
\begin{align*}
  \cos(\theta)\|\aaa\|=\frac{\Braket{\aaa,\bbb}}{\|\bbb\|},
\end{align*}
である.
したがって,
\begin{align*}
  \vv=\frac{\cos(\theta)\|\aaa\|}{\|\bbb\|}\bbb=
  \frac{\Braket{\aaa,\bbb}}{\|\bbb\|^2}\bbb
\end{align*}
である. 
\begin{align*}
  \Braket{\aaa-\vv,\bbb}
  &=
  \Braket{\aaa,\bbb}-\Braket{\vv,\bbb}\\
  &=\Braket{\aaa,\bbb}-\left\langle\frac{\Braket{\aaa,\bbb}}{\|\bbb\|^2}\bbb,\bbb\right\rangle\\
  &=\Braket{\aaa,\bbb}-\frac{\Braket{\aaa,\bbb}}{\|\bbb\|^2}\Braket{\bbb,\bbb}\\
  &=\Braket{\aaa,\bbb}-\frac{\Braket{\aaa,\bbb}}{\|\bbb\|^2}\|\bbb\|^2\\
  &=\Braket{\aaa,\bbb}-\Braket{\aaa,\bbb}\\
  &=0
\end{align*}
であるから, $\pp=\vv$である.

$\aaa$から原点と$\bbb$を通る直線への直交射影$\pp$は,
\begin{align*}
\pp=\frac{\Braket{\aaa,\bbb}}{\|\bbb\|^2}\bbb
\end{align*}
とかけるので,
\begin{align*}
  \pp&=\frac{\Braket{\aaa,\bbb}}{\|\bbb\|^2}\bbb\\
  &=\frac{\Braket{\begin{pmatrix}7\\-4\end{pmatrix},\begin{pmatrix}2\\-3\end{pmatrix}}}{\|\begin{pmatrix}2\\-3\end{pmatrix}\|^2}\begin{pmatrix}2\\-3\end{pmatrix}\\
  &=\frac{14+12}{4+9}\begin{pmatrix}2\\-3\end{pmatrix}\\
  &=2\begin{pmatrix}2\\-3\end{pmatrix}\\
  &=\begin{pmatrix}4\\-6\end{pmatrix}
\end{align*}
である.
垂直成分$\qq$は,
$\aaa=\pp+\qq$を満たすので
\begin{align*}
  \qq=\aaa-\pp
  =\begin{pmatrix}7\\-4\end{pmatrix}-\begin{pmatrix}4&-6\end{pmatrix}
  =\begin{pmatrix}3\\2\end{pmatrix}
\end{align*}
である.
\end{answerof}

\begin{answerof}{quiz:5:p:2}
  \begin{align*}
    \aaa&=\begin{pmatrix}8\\-3\end{pmatrix}&
    \bbb&=\begin{pmatrix}2\\-3\end{pmatrix}&
    \ccc&=\begin{pmatrix}1\\1\end{pmatrix}
  \end{align*}
  とし, $\aaa$から$L$への直交射影を$\pp$とする.
  
  $L$は原点を通らないので, 原点を通るように平行移動して考える.
  $L$は$\ccc$を通るので,
  \begin{align*}
    \aaa'&=\aaa-\ccc&
    \pp'&=\pp-\ccc&
    L'&=\Set{t\bbb|t\in\RR}
  \end{align*}
  について考える.
  このとき, $\pp'$は$\aaa'$から原点を通る直線$L'$への直交射影になっている.
  $\aaa'$から原点と$\bbb$を通る直線への直交射影$\pp'$は,
  \begin{align*}
    \pp'=\frac{\Braket{\aaa',\bbb}}{\|\bbb\|^2}\bbb
  \end{align*}
  とかけるので,
  \begin{align*}
    \aaa'&=\aaa-\ccc=\begin{pmatrix}8\\-3\end{pmatrix}-\begin{pmatrix}1\\1\end{pmatrix}
    =\begin{pmatrix}8-1\\-3-1\end{pmatrix}=\begin{pmatrix}7\\-4\end{pmatrix}
  \end{align*}
  であることから,
  \begin{align*}
  \pp'&=\frac{\Braket{\aaa',\bbb}}{\|\bbb\|^2}\bbb\\
  &=\frac{\Braket{\begin{pmatrix}7\\-4\end{pmatrix},\begin{pmatrix}2\\-3\end{pmatrix}}}{\|\begin{pmatrix}2\\-3\end{pmatrix}\|^2}\begin{pmatrix}2\\-3\end{pmatrix}\\
  &=\frac{14+12}{4+9}\begin{pmatrix}2\\-3\end{pmatrix}\\
      &=2\begin{pmatrix}2\\-3\end{pmatrix}\\
  &=\begin{pmatrix}4\\-6\end{pmatrix}
  \end{align*}
  である.
  いま, $\pp'=\pp-\ccc$
  であるから,
  \begin{align*}
    \pp=\pp'+\ccc=\begin{pmatrix}4\\-6\end{pmatrix}+\begin{pmatrix}1\\1\end{pmatrix}=\begin{pmatrix}5\\-5\end{pmatrix}
  \end{align*}
  である.  
  垂直成分$\qq$は,
  $\aaa=\pp+\qq$を満たすので
  \begin{align*}
    \qq=\aaa-\pp
    =\begin{pmatrix}8\\-3\end{pmatrix}-\begin{pmatrix}5&-5\end{pmatrix}
    =\begin{pmatrix}3\\2\end{pmatrix}
  \end{align*}
  である.


  これは次の様に考えてもよい:
  $\pp$は
  \begin{align*}
    \pp&=t\bbb+\ccc\\
    \Braket{\aaa-\pp,\bbb}&=0
  \end{align*}
  を満たすものである.
  \begin{align*}
    \Braket{\aaa-\pp,\bbb}&=
    \Braket{\aaa-(t\bbb+\ccc),\bbb}\\
    &=\Braket{\aaa-\ccc-t\bbb,\bbb}\\
    &=\Braket{\aaa,\bbb}-\Braket{\ccc,\bbb}-t\Braket{\bbb,\bbb}\\
    &=\Braket{\begin{pmatrix}8\\-3\end{pmatrix},\begin{pmatrix}2\\-3\end{pmatrix}}-\Braket{\begin{pmatrix}1\\1\end{pmatrix},\begin{pmatrix}2\\-3\end{pmatrix}}-t\Braket{\begin{pmatrix}2\\-3\end{pmatrix},\begin{pmatrix}2\\-3\end{pmatrix}}\\
    &=16+9-2+3-t(4+9)\\
    &=26-13t
  \end{align*}
  となっているので, $26-13t=0$である.
  つまり$t=2$であり,
  \begin{align*}
    \pp&=t\bbb+\ccc\\
    &=2\begin{pmatrix}2\\-3\end{pmatrix}+\begin{pmatrix}1\\1\end{pmatrix}\\
    &=\begin{pmatrix}5\\-5\end{pmatrix}
  \end{align*}
  である.
  垂直成分$\qq$は,
  $\aaa=\pp+\qq$を満たすので
  \begin{align*}
    \qq=\aaa-\pp
    =\begin{pmatrix}8\\-3\end{pmatrix}-\begin{pmatrix}5&-5\end{pmatrix}
    =\begin{pmatrix}3\\2\end{pmatrix}
  \end{align*}
  である.
\end{answerof}

\begin{answerof}{quiz:5:p:3}
  点と直線の距離を求めるには,
  直交射影を考えその垂直成分のノルムを求めればよかった.


  \begin{align*}
    \aaa&=\begin{pmatrix}8\\-3\end{pmatrix}&
    \bbb&=\begin{pmatrix}2\\-3\end{pmatrix}&
    \ccc&=\begin{pmatrix}1\\1\end{pmatrix}
  \end{align*}
  とし, $\aaa$から$L$への直交射影を$\pp$とする.
  
  $L$は原点を通らないので, 原点を通るように平行移動して考える.
  $L$は$\ccc$を通るので,
  \begin{align*}
    \aaa'&=\aaa-\ccc&
    \pp'&=\pp-\ccc&
    L'&=\Set{t\bbb|t\in\RR}
  \end{align*}
  について考える.
  このとき, $\pp'$は$\aaa'$から原点を通る直線$L'$への直交射影になっている.
  $\aaa'$から原点と$\bbb$を通る直線への直交射影$\pp'$は,
  \begin{align*}
    \pp'=\frac{\Braket{\aaa',\bbb}}{\|\bbb\|^2}\bbb
  \end{align*}
  とかけるので,
  \begin{align*}
    \aaa'&=\aaa-\ccc=\begin{pmatrix}8\\-3\end{pmatrix}-\begin{pmatrix}1\\1\end{pmatrix}
    =\begin{pmatrix}8-1\\-3-1\end{pmatrix}=\begin{pmatrix}7\\-4\end{pmatrix}
  \end{align*}
  であることから,
  \begin{align*}
  \pp'&=\frac{\Braket{\aaa',\bbb}}{\|\bbb\|^2}\bbb\\
  &=\frac{\Braket{\begin{pmatrix}7\\-4\end{pmatrix},\begin{pmatrix}2\\-3\end{pmatrix}}}{\|\begin{pmatrix}2\\-3\end{pmatrix}\|^2}\begin{pmatrix}2\\-3\end{pmatrix}\\
  &=\frac{14+12}{4+9}\begin{pmatrix}2\\-3\end{pmatrix}\\
      &=2\begin{pmatrix}2\\-3\end{pmatrix}\\
  &=\begin{pmatrix}4\\-6\end{pmatrix}
  \end{align*}
  である.
  いま, $\pp'=\pp-\ccc$
  であるから,
  \begin{align*}
    \pp=\pp'+\ccc=\begin{pmatrix}4\\-6\end{pmatrix}+\begin{pmatrix}1\\1\end{pmatrix}=\begin{pmatrix}5\\-5\end{pmatrix}
  \end{align*}
  である.  
  垂直成分$\qq$は,
  $\aaa=\pp+\qq$を満たすので
  \begin{align*}
    \qq=\aaa-\pp
    =\begin{pmatrix}8\\-3\end{pmatrix}-\begin{pmatrix}5&-5\end{pmatrix}
    =\begin{pmatrix}3\\2\end{pmatrix}
  \end{align*}
  である.
  よって, $\aaa$と$L$の距離は
  \begin{align*}
    \|\qq\|= \|\begin{pmatrix}3\\2\end{pmatrix}\|=\sqrt{3^2+2^2}=\sqrt{13}
  \end{align*}
  である.

  これは次の様に考えてもよい:
  $\pp$は
  \begin{align*}
    \pp&=t\bbb+\ccc\\
    \Braket{\aaa-\pp,\bbb}&=0
  \end{align*}
  を満たすものである.
  \begin{align*}
    \Braket{\aaa-\pp,\bbb}&=
    \Braket{\aaa-(t\bbb+\ccc),\bbb}\\
    &=\Braket{\aaa-\ccc-t\bbb,\bbb}\\
    &=\Braket{\aaa,\bbb}-\Braket{\ccc,\bbb}-t\Braket{\bbb,\bbb}\\
    &=\Braket{\begin{pmatrix}8\\-3\end{pmatrix},\begin{pmatrix}2\\-3\end{pmatrix}}-\Braket{\begin{pmatrix}1\\1\end{pmatrix},\begin{pmatrix}2\\-3\end{pmatrix}}-t\Braket{\begin{pmatrix}2\\-3\end{pmatrix},\begin{pmatrix}2\\-3\end{pmatrix}}\\
    &=16+9-2+3-t(4+9)\\
    &=26-13t
  \end{align*}
  となっているので, $26-13t=0$である.
  つまり$t=2$であり,
  \begin{align*}
    \pp&=t\bbb+\ccc\\
    &=2\begin{pmatrix}2\\-3\end{pmatrix}+\begin{pmatrix}1\\1\end{pmatrix}\\
    &=\begin{pmatrix}5\\-5\end{pmatrix}
  \end{align*}
  である.
  垂直成分$\qq$は,
  $\aaa=\pp+\qq$を満たすので
  \begin{align*}
    \qq=\aaa-\pp
    =\begin{pmatrix}8\\-3\end{pmatrix}-\begin{pmatrix}5&-5\end{pmatrix}
    =\begin{pmatrix}3\\2\end{pmatrix}
  \end{align*}
  である.

  よって, $\aaa$と$L$の距離は
  \begin{align*}
    \|\qq\|= \|\begin{pmatrix}3\\2\end{pmatrix}\|=\sqrt{3^2+2^2}=\sqrt{13}
  \end{align*}
  である.
\end{answerof}
  
\begin{answerof}{quiz:5:12}
  \begin{align*}
    \aaa&=\begin{pmatrix}3\\4\end{pmatrix},&
    \bbb&=\begin{pmatrix}1\\2\end{pmatrix}.
  \end{align*}
  \begin{align*}
    \|\aaa\|
    &=\left\|\begin{pmatrix}3\\4\end{pmatrix}\right\|
    =\sqrt{3^2+4^2}=5
  \end{align*}
  であるので,
  \begin{align*}
    \overline{\aaa}
    &=\frac{1}{\|\aaa\|}\aaa
    =\frac{1}{5}\begin{pmatrix}3\\4\end{pmatrix}
  \end{align*}
  とおく.

  つぎに,
  \begin{align*}
    \Braket{\overline \aaa,\bbb}=\Braket{\frac{1}{5}\begin{pmatrix}3\\4\end{pmatrix},\begin{pmatrix}1\\2\end{pmatrix}}
    =\frac{1}{5}(3+8)=\frac{11}{5}
  \end{align*}
  であるので,
  \begin{align*}
    \bbb'
    &=\bbb-\Braket{\overline{\aaa},\bbb}\overline{\aaa}\\
    &=
    \begin{pmatrix}1\\2\end{pmatrix}-\frac{11}{5}\frac{1}{5}\begin{pmatrix}3\\4\end{pmatrix}\\
    &=
    \begin{pmatrix}1\\2\end{pmatrix}-\frac{11}{25}\begin{pmatrix}3\\4\end{pmatrix}\\
    &=
    \frac{1}{25}\begin{pmatrix}25-33\\50-44\end{pmatrix}
    &=
    \frac{1}{25}\begin{pmatrix}-8\\6\end{pmatrix}
  \end{align*}
  とおく.

  \begin{align*}
    \|\bbb'\|
    &=\left\|\frac{1}{25}\begin{pmatrix}-8\\6\end{pmatrix}\right\|
    =\frac{1}{25}\left\|\begin{pmatrix}-8\\6\end{pmatrix}\right\|
    =\frac{\sqrt{64+36}}{25}\\
    &=\frac{10}{25}
    =\frac{2}{5}
  \end{align*}
  であるので,
  \begin{align*}
    \overline{\bbb}
    &=\frac{1}{\|\bbb'\|}\bbb'
    =\frac{5}{2}\frac{1}{25}\begin{pmatrix}-8\\6\end{pmatrix}
    =\frac{1}{5}\begin{pmatrix}-4\\3\end{pmatrix}
  \end{align*}
  とおく.

  このとき, $\overline{\aaa}$, $\overline{\bbb}$
  は,
  \begin{align*}
    \|\overline{\aaa}\|=\|\overline{\bbb}\|=1\\
    \Braket{\overline{\aaa},\overline{\bbb}}=0
  \end{align*}
  を満たしている.
\end{answerof}


\begin{answerof}{quiz:5:15}
  \begin{align*}
    f(\begin{pmatrix}x\\y\end{pmatrix})
      =\begin{pmatrix}x+3y\\2x+7y\end{pmatrix}
      =\begin{pmatrix}1&3\\2&7\end{pmatrix}\begin{pmatrix}x\\y\end{pmatrix}
  \end{align*}
  とかける.


  次の様に考えることもできる:
  \begin{align*}
    f(\begin{pmatrix}\ee_1\end{pmatrix})
    =f(\begin{pmatrix}1\\0\end{pmatrix})
    =\begin{pmatrix}1\\2\end{pmatrix},
    f(\begin{pmatrix}\ee_2\end{pmatrix})
    =f(\begin{pmatrix}0\\1\end{pmatrix})
    =\begin{pmatrix}3\\7\end{pmatrix}
  \end{align*}
  である.
  $f(\ee_1)$と$f(\ee_2)$を並べてできる$2$次正方行列を$A$とおくと,
  $f(\xx)=A\xx$となる.
  したがって,
  \begin{align*}
    A=\begin{pmatrix}1&3\\2&7\end{pmatrix}
  \end{align*}
  とおけば$f(\xx)=A\xx$である.
\end{answerof}


\begin{answerof}{quiz:5:16}
  $f(\ee_1)$と$f(\ee_2)$を並べてできる$2$次正方行列を$A$とおくと,
  $f(\xx)=A\xx$となる.
  したがって,
  \begin{align*}
    A=\begin{pmatrix}1&2\\-3&-9\end{pmatrix}
  \end{align*}
  とおけば$f(\xx)=A\xx$である.

  次の様に考えることもできる:
  \begin{align*}
    f(\begin{pmatrix}x\\y\end{pmatrix})
      &=f(x\ee_1+y\ee_2)\\
      &=f(x\ee_1)+f(y\ee_2)\\
      &=xf(\ee_1)+yf(\ee_2)\\
      &=x\begin{pmatrix}1\\-3\end{pmatrix}+y\begin{pmatrix}2\\-9\end{pmatrix}\\
      &=\begin{pmatrix}x\\-3x\end{pmatrix}+\begin{pmatrix}2y\\-9y\end{pmatrix}\\
      &=\begin{pmatrix}x+2y\\-3x-9y\end{pmatrix}
      =\begin{pmatrix}1&2\\-3&-9\end{pmatrix}\begin{pmatrix}x\\y\end{pmatrix}
  \end{align*}
  とかける.
\end{answerof}


\begin{answerof}{quiz:5:17}
  $f(\xx)=A\xx$, $g(\xx)=B\xx$
  であるとき, $(f\circ g)(\xx)=AB\xx$
  と書ける. よって
  \begin{align*}
    (f\circ g)(\xx)
    &=AB\xx\\
    &=\begin{pmatrix}1&3\\2&4\end{pmatrix}\begin{pmatrix}7&8\\-1&3\end{pmatrix}\xx\\
    &=\begin{pmatrix}7-3&8+9\\14-4&16+12\end{pmatrix}\xx\\
    &=\begin{pmatrix}4&17\\10&28\end{pmatrix}\xx.
  \end{align*}
  
  次の様に考えることもできる:
  \begin{align*}
    (f\circ g)(\begin{pmatrix}x\\y\end{pmatrix})
    &=f(g(\begin{pmatrix}x\\y\end{pmatrix}))\\
    &=f(B\begin{pmatrix}x\\y\end{pmatrix})\\
    &=f(\begin{pmatrix}7&8\\-1&3\end{pmatrix}\begin{pmatrix}x\\y\end{pmatrix})\\
    &=f(\begin{pmatrix}7x+8y\\-x+3y\end{pmatrix})\\
    &=A\begin{pmatrix}7x+8y\\-x+3y\end{pmatrix}\\
    &=\begin{pmatrix}1&3\\2&4\end{pmatrix}\begin{pmatrix}7x+8y\\-x+3y\end{pmatrix}\\
    &=\begin{pmatrix}(7x+8y)+3(-x+3y)\\2(7x+8y)+4(-x+3y)\end{pmatrix})\\
    &=\begin{pmatrix}4x+17y\\10x+28y\end{pmatrix}\\
    &=\begin{pmatrix}4&17\\10&28\end{pmatrix}\begin{pmatrix}x\\y\end{pmatrix}.
  \end{align*}
  とかける.
\end{answerof}



\begin{answerof}{quiz:5:18}
  $f(\xx)=A\xx$, 
  であるとき,
  $f$が全単射であることと$A$が正則であることが同値であり,
  $f$が全単射であるときに$f$の逆写像$f^{-1}$は
  $f^{-1}(\xx)=A^{-1}\xx$という形で,
  $A$の逆行列を用いて表すことができた.
  今回の場合,
  \begin{align*}
    \det(A)=
    \det(\begin{pmatrix}1&3\\2&4\end{pmatrix})
      =1\cdot4-3\cdot 2=4-6=-2\neq 0
  \end{align*}
  であるので, $A$は正則である.
  また$A$の逆行列は,
  \begin{align*}
    A^{-1}=\frac{1}{\det(A)}
    \begin{pmatrix}4&-3\\-2&1\end{pmatrix}
      =\frac{1}{-2}\begin{pmatrix}4&-3\\-2&1\end{pmatrix}\\
      =\begin{pmatrix}-2&\frac{3}{2}\\1&frac{-1}{2}\end{pmatrix}
  \end{align*}
  である.
  したがって,
  \begin{align*}
    f^{-1}(\xx)=
    A^{-1}\xx
      =\begin{pmatrix}-2&\frac{3}{2}\\1&frac{-1}{2}\end{pmatrix}\xx
  \end{align*}
  と書ける.

\end{answerof}

\begin{answerof}{quiz:5:p:4}
  \begin{align*}
    \aaa&=\begin{pmatrix}7\\-4\end{pmatrix}\\
    \bbb&=\begin{pmatrix}2\\-3\end{pmatrix}
  \end{align*}
  とする.
  $\aaa$の$L$への直交射影を$\pp$とすると,
  $\aaa$と$f(\aaa)$の中点が$\pp$である.
  つまり$\pp=\frac{1}{2}\aaa+\frac{1}{2}f(\aaa)$である.
  $f(\aaa)=2\pp-\aaa$
  とかけるので,
  まずは直交射影$\pp$を求める.
$\aaa$から原点と$\bbb$を通る直線への直交射影$\pp$は,
\begin{align*}
\pp=\frac{\Braket{\aaa,\bbb}}{\|\bbb\|^2}\bbb
\end{align*}
とかけるので,
\begin{align*}
  \pp&=\frac{\Braket{\aaa,\bbb}}{\|\bbb\|^2}\bbb\\
  &=\frac{\Braket{\begin{pmatrix}7\\-4\end{pmatrix},\begin{pmatrix}2\\-3\end{pmatrix}}}{\|\begin{pmatrix}2\\-3\end{pmatrix}\|^2}\begin{pmatrix}2\\-3\end{pmatrix}\\
  &=\frac{14+12}{4+9}\begin{pmatrix}2\\-3\end{pmatrix}\\
  &=2\begin{pmatrix}2\\-3\end{pmatrix}\\
  &=\begin{pmatrix}4\\-6\end{pmatrix}
\end{align*}
である.
したがって,
\begin{align*}
  f(\aaa)=2\pp-\aaa
  =2\begin{pmatrix}4\\-6\end{pmatrix}-\begin{pmatrix}7\\-4\end{pmatrix}\\
  =\begin{pmatrix}8-7\\-12+4\end{pmatrix}\\
  =\begin{pmatrix}-3\\-8\end{pmatrix}\\
\end{align*}
\end{answerof}

\begin{answerof}{quiz:5:19}
  原点を中心に$\theta$だけ回転させるには,
  \begin{align*}
    \begin{pmatrix}
      \cos(\theta)&-\sin(\theta)\\
      \sin(\theta)&\cos(\theta)
    \end{pmatrix}
  \end{align*}
  をかければよかった.
  今は, $\theta=\frac{\pi}{2}$であるので,
  \begin{align*}
    \begin{pmatrix}
      \cos(\frac{\pi}{2})&-\sin(\frac{\pi}{2})\\
      \sin(\frac{\pi}{2})&\cos(\frac{\pi}{2})
    \end{pmatrix}
    =
    \begin{pmatrix}
      0&-1\\
      1&0
    \end{pmatrix}
  \end{align*}
  であるので,
  \begin{align*}
    f(\xx)&=
    \begin{pmatrix}
      0&-1\\
      1&0
    \end{pmatrix}
    \xx\\
    f(\aaa)&=
    \begin{pmatrix}
      0&-1\\
      1&0
    \end{pmatrix}
    \begin{pmatrix}
      7\\
      -4
    \end{pmatrix}
    =
    \begin{pmatrix}
      4\\7
    \end{pmatrix}
  \end{align*}
  である.


  もし複素平面の扱いの方が慣れているのであれば次のように考えても良い:
  虚数単位を$\sqrt{-1}$と書くことにする.
  絶対値が1で偏角が$\theta$である複素数$z$は,
  $z=\cos(\theta)+\sin(\theta)\sqrt{-1}$
  と書くことができる.
  $x+y\sqrt{-1}$という複素平面上の点は,
  $z$をかけることによって$\theta$だけ回転するから,
  回転後の点は,
  \begin{align*}
    (x+y\sqrt{-1})z&=(x+y\sqrt{-1})(\cos(\theta)+\sin(\theta)\sqrt{-1})\\
    &=x\cos(\theta)+y\cos(\theta)\sqrt{-1}+x\sin(\theta)\sqrt{-1}-y\sin(\theta)\\
    &=(\cos(\theta)x-\sin(\theta)y)+(x\sin(\theta)+\cos(\theta))y\sqrt{-1}
  \end{align*}
  と表すことができる.
  ベクトルを用いて表すと,
  \begin{align*}
    \begin{pmatrix}x\\y\end{pmatrix}
  \end{align*}
  を$\theta$だけ回転させた点は
  \begin{align*}
      \begin{pmatrix}
        \cos(\theta)x-\sin(\theta)y\\x\sin(\theta)+\cos(\theta))y
      \end{pmatrix}\\
      &=
      \begin{pmatrix}
        \cos(\theta)&-\sin(\theta)\\\sin(\theta)&\cos(\theta))
      \end{pmatrix}
      \begin{pmatrix}x\\y\end{pmatrix}
  \end{align*}
  とかける.
  今は, $\theta=\frac{\pi}{2}$であるので,
  \begin{align*}
    f(\xx)&=
    \begin{pmatrix}
      \cos(\frac{\pi}{2})&-\sin(\frac{\pi}{2})\\
      \sin(\frac{\pi}{2})&\cos(\frac{\pi}{2})
    \end{pmatrix}
    \xx=
    \begin{pmatrix}
      0&-1\\
      1&0
    \end{pmatrix}
    \xx
  \end{align*}
  とかける.  また,
  \begin{align*}
    f(\aaa)&=
    \begin{pmatrix}
      0&-1\\
      1&0
    \end{pmatrix}
    \begin{pmatrix}
      7\\
      -4
    \end{pmatrix}
    =
    \begin{pmatrix}
      4\\7
    \end{pmatrix}
  \end{align*}
  である.
  
\end{answerof}


\begin{answerof}{quiz:5:20}
  $R$を回転行列とすると,
  $\theta$をつかって
  \begin{align*}
    R&=\begin{pmatrix}\cos(\theta)&-\sin(\theta)\\\sin(\theta)&\cos(\theta)\end{pmatrix}
  \end{align*}
  とかける.
  また,
  $B$を
  上半三角行列とすると,
  $a,b,c$を使って,
  \begin{align*}
    R&=\begin{pmatrix}a&b\\0&c\end{pmatrix}
  \end{align*}
  とかける.
  $\theta,a,b,c$を求める.

  \begin{align*}
    RB=
    \begin{pmatrix}a\cos(\theta)&b\cos(\theta)-c\sin(\theta)\\
      a\sin(\theta)&b\sin(\theta)+c\cos(\theta)\end{pmatrix}
  \end{align*}
  である.
  1列目のみを考えると
  \begin{align*}
    \begin{pmatrix}a\cos(\theta)\\a\sin(\theta)\end{pmatrix}
    =a\begin{pmatrix}\cos(\theta)\\\sin(\theta)\end{pmatrix}
  \end{align*}
  となる.
  このベクトルのノルムは, $|a|$である.
  $A=RB$を満たすという条件から,
  \begin{align*}
    a\begin{pmatrix}\cos(\theta)\\\sin(\theta)\end{pmatrix}
    =\begin{pmatrix}3\\4\end{pmatrix}    
  \end{align*}
  が得られるが, 
 \begin{align*}
     \left\|\begin{pmatrix}3\\4\end{pmatrix}\right\|=5
  \end{align*}
 であるので, $a\in\Set{5,-5}$ である.

 まず$a=5$のときについて考える.
 このとき,
 \begin{align*}
    \begin{pmatrix}\cos(\theta)\\\sin(\theta)\end{pmatrix}
    =\frac{1}{a}\begin{pmatrix}3\\4\end{pmatrix}    
    =\frac{1}{5}\begin{pmatrix}3\\4\end{pmatrix}    
  \end{align*}
 であるので,
 \begin{align*}
   \cos(\theta)&=\frac{3}{5}&
   \sin(\theta)&=\frac{4}{5}
 \end{align*}
 となる.
 このとき, 
 \begin{align*}
   R&=\begin{pmatrix}\cos(\theta)&-\sin(\theta)\\\sin(\theta)&\cos(\theta)\end{pmatrix}\\
   &\begin{pmatrix}\frac{3}{5}&-\frac{4}{5}\\\frac{4}{5}&\frac{3}{5}\end{pmatrix}
   &\frac{1}{5}\begin{pmatrix}3&-4\\4&3\end{pmatrix}
 \end{align*}
 である.
 一方$RB$の2列目は,
 \begin{align*}
   R\begin{pmatrix}b\\c\end{pmatrix}
   =\frac{1}{5}\begin{pmatrix}3&-4\\4&3\end{pmatrix}\begin{pmatrix}b\\c\end{pmatrix}
\end{align*}
 であるので, $A$の2列目と比較し,
 \begin{align*}
\frac{1}{5}\begin{pmatrix}3&-4\\4&3\end{pmatrix}\begin{pmatrix}b\\c\end{pmatrix}=\begin{pmatrix}5\\5\end{pmatrix}
 \end{align*}
 となる.  $R$の逆行列は,
 \begin{align*}
   R^{-1}=
 \begin{pmatrix}\frac{3}{5}&\frac{4}{5}\\-\frac{4}{5}&\frac{3}{5}\end{pmatrix}
=\frac{1}{5} \begin{pmatrix}3&4\\-4&3\end{pmatrix}
 \end{align*}
 であるので,
\begin{align*}
  \begin{pmatrix}b\\c\end{pmatrix}
    &=R^{-1}\begin{pmatrix}5\\5\end{pmatrix}
    &=\frac{1}{5} \begin{pmatrix}3&4\\-4&3\end{pmatrix}\begin{pmatrix}5\\5\end{pmatrix}\\
    &=\begin{pmatrix}3+4\\-4+3\end{pmatrix}\\
    &=\begin{pmatrix}7\\-1\end{pmatrix}
 \end{align*}
となる.
よって,
\begin{align*}
  R&=\frac{1}{5}\begin{pmatrix}3&-4\\4&3\end{pmatrix}\\
  B&=\begin{pmatrix}5&7\\0&-1\end{pmatrix}
\end{align*}
 とすると$A=RB$を満たす.


 次に$a=-5$のときについて考える.
 このとき,
 \begin{align*}
    \begin{pmatrix}\cos(\theta)\\\sin(\theta)\end{pmatrix}
    =\frac{1}{a}\begin{pmatrix}3\\4\end{pmatrix}    
    =\frac{1}{-5}\begin{pmatrix}3\\4\end{pmatrix}    
  \end{align*}
 であるので,
 \begin{align*}
   \cos(\theta)&=-\frac{3}{5}&
   \sin(\theta)&=-\frac{4}{5}
 \end{align*}
 となる.
 このとき, 
 \begin{align*}
   R&=\begin{pmatrix}\cos(\theta)&-\sin(\theta)\\\sin(\theta)&\cos(\theta)\end{pmatrix}\\
   &\begin{pmatrix}-\frac{3}{5}&\frac{4}{5}\\-\frac{4}{5}&-\frac{3}{5}\end{pmatrix}
   &\frac{-1}{5}\begin{pmatrix}3&-4\\4&3\end{pmatrix}
 \end{align*}
 である.
 一方$RB$の2列目は,
 \begin{align*}
   R\begin{pmatrix}b\\c\end{pmatrix}
   =\frac{-1}{5}\begin{pmatrix}3&-4\\4&3\end{pmatrix}\begin{pmatrix}b\\c\end{pmatrix}
\end{align*}
 であるので, $A$の2列目と比較し,
 \begin{align*}
\frac{-1}{5}\begin{pmatrix}3&-4\\4&3\end{pmatrix}\begin{pmatrix}b\\c\end{pmatrix}=\begin{pmatrix}5\\5\end{pmatrix}
 \end{align*}
 となる.  $R$の逆行列は,
 \begin{align*}
   R^{-1}=
 \begin{pmatrix}-\frac{3}{5}&-\frac{4}{5}\\\frac{4}{5}&-\frac{3}{5}\end{pmatrix}
=\frac{-1}{5} \begin{pmatrix}3&4\\-4&3\end{pmatrix}
 \end{align*}
 であるので,
\begin{align*}
  \begin{pmatrix}b\\c\end{pmatrix}
    &=R^{-1}\begin{pmatrix}5\\5\end{pmatrix}
    &=\frac{-1}{5} \begin{pmatrix}3&4\\-4&3\end{pmatrix}\begin{pmatrix}5\\5\end{pmatrix}\\
    &=-\begin{pmatrix}3+4\\-4+3\end{pmatrix}\\
    &=\begin{pmatrix}-7\\1\end{pmatrix}
 \end{align*}
となる.
よって,
\begin{align*}
  R&=\frac{-1}{5}\begin{pmatrix}3&-4\\4&3\end{pmatrix}\\
  B&=-\begin{pmatrix}5&7\\0&-1\end{pmatrix}
\end{align*}
 とすると$A=RB$を満たす.
\end{answerof}


\section{固有値と固有ベクトルに関する問題}

\begin{answerof}{quiz:6:1}
固有多項式とは$\det(A-xE_2)$のことである.
  \begin{align*}
    A-xE_2&=\begin{pmatrix}-4-x&1\\5&-x\end{pmatrix}\\
    \det(A-xE_2)&=(-4-x)(-x)-1\cdot 5=x^2+4x-5=(x-1)(x+5)
  \end{align*}
  であるので, $A$の固有多項式は$x^2+4x-5$である.

  $A$の固有値とは, $A\xx=\lambda\xx$をみたす$\xx\neq\zzero$が存在するような$\lambda$のことである.
  したがって, $\lambda$が$A$の固有値であることと, 
  $\lambda$が$x$に関する方程式$\det(A-xE_2)=0$の解であることは同値である.
  したがって, $\det(A-xE_2)=(x-1)(x+5)$であるので,
  \begin{align*}
    (x-1)(x+5)=0
  \end{align*}
  を解けばよい.  $A$の固有値は$1$, $5$である.  
\end{answerof}


\begin{answerof}{quiz:6:2}
  $2$次正方行列$A$が対角化できることと,
  一次独立な固有ベクトルの組が存在することは同値であった.
  また, 相異なる固有値に属する固有ベクトルの組は一次独立であった.
  
  まず$A$の固有値を求める.
  そのために, $\det(A-xE_2)=0$という$x$に関する方程式をとく.
  \begin{align*}
    A-xE_2&=\begin{pmatrix}-4-x&1\\5&-x\end{pmatrix}\\
    \det(A-xE_2)&=(-4-x)(-x)-1\cdot 5=x^2+4x-5=(x-1)(x+5)
  \end{align*}
  であるので, $(x-1)(x+5)=0$を解けばよく,
  $1$と$-5$が固有値である.
  したがって,
  固有値$1$に属する固有ベクトル$\vv$と,
  固有値$-5$に属する固有ベクトル$\ww$をとってくると,
  $(\vv,\ww)$は一次独立である.
  したがって, $A$は対角化できる.
\end{answerof}

\begin{answerof}{quiz:6:3}
  $2$次正方行列$A$が対角化できることと,
  一次独立な固有ベクトルの組が存在することは同値であった.

  まず$A$の固有値を求める.
  そのために, $\det(A-xE_2)=0$という$x$に関する方程式をとく.
  \begin{align*}
    A-xE_2&=\begin{pmatrix}2-x&1\\0&2-x\end{pmatrix}\\
    \det(A-xE_2)&=(2-x)^2-1\cdot 0=(2-x)^2
  \end{align*}
  であるので, $(2-x)^2=0$を解けばよく,
  固有値は$2$のみであることがわかる.

  次に固有値$2$に属する$A$の固有値を求める.
  つまり$A\xx=2\xx$となる$\xx$について考える.
  \begin{align*}
    A\xx&=2\xx\\
    A\xx-2\xx&=\zzero \\
    (A-2E_2)\xx&=\zzero
  \end{align*}
  と変形できる.
  \begin{align*}
    A-2E_2=\begin{pmatrix}2-2&1\\0&2-2\end{pmatrix}
    =\begin{pmatrix}0&1\\0&0\end{pmatrix}
  \end{align*}
  であるので, $\xx$に関する方程式
  \begin{align*}
    \begin{pmatrix}0&1\\0&0\end{pmatrix}\xx&=\zzero
  \end{align*}
  を解けばよい.
  \begin{align*}
    \begin{pmatrix}0&1\\0&0\end{pmatrix}
      \begin{pmatrix}x\\y\end{pmatrix}
      &=\zzero\\
      \begin{pmatrix}y\\0\end{pmatrix}
      &=\zzero\\
  \end{align*}
  であるから, 解は変数$t$を用いて
  \begin{align*}
    \xx&=\begin{pmatrix}t\\0\end{pmatrix} =t\ee_1
  \end{align*}
  と表せる.
  $A$の固有値は$2$のみであるから,
  $\vv$と$\ww$を$A$の固有ベクトルとすると,
  $a\neq 0\neq b$を使って
  \begin{align*}
    \vv&=\begin{pmatrix}a\\0\end{pmatrix}=a\ee_1 \\
    \ww&=\begin{pmatrix}b\\0\end{pmatrix}=b\ee_2
  \end{align*}
  と書ける. $(a,b)\neq(0,0)$であるにも関わらず
  $b \vv-a\ww = ab\ee_1-ab\ee_1=\zzero$となるので
  $(\vv,\ww)$は一次独立ではない.

  どんな固有ベクトルを2つ選んできても一次独立にはならない.
  つまり, 一次独立な固有ベクトルの組は存在しない.
  したがって, $A$は対角化できない.
\end{answerof}


\begin{answerof}{quiz:6:4}
  まず$A$の固有値を求める.
  そのために, $\det(A-xE_2)=0$という$x$に関する方程式をとく.
  \begin{align*}
    A-xE_2&=\begin{pmatrix}-4-x&1\\5&-x\end{pmatrix}\\
    \det(A-xE_2)&=(-4-x)(-x)-1\cdot 5=x^2+4x-5=(x-1)(x+5)
  \end{align*}
  であるので, $(x-1)(x+5)=0$を解けばよく,
  $1$と$-5$が固有値である.

  固有値$1$に属する固有ベクトルを求める.
  つまり$A\xx=\xx$となる$\xx$について考える.
  これは$(A-E_2)\xx=\zzero$
  を解けばよい.
  \begin{align*}
    A-E_2=\begin{pmatrix}-4-1&1\\5&-1\end{pmatrix}
    =\begin{pmatrix}-5&1\\5&-1\end{pmatrix}
  \end{align*}
  であるので,
  \begin{align*}
    \begin{pmatrix}-5&1\\5&-1\end{pmatrix}
    \begin{pmatrix}x\\y\end{pmatrix}=
    \begin{pmatrix}0\\0\end{pmatrix}
  \end{align*}
  をとく.
  \begin{align*}
    \begin{pmatrix}-5&1\\5&-1\end{pmatrix}
  \end{align*}
  の$2$行目に$1$行目の$1$倍を足すと,
 \begin{align*}
    \begin{pmatrix}-5&1\\0&0\end{pmatrix}.
  \end{align*}
 $1$行目を$\frac{-1}{5}$倍すると
 \begin{align*}
    \begin{pmatrix}1&\frac{-1}{5}\\0&0\end{pmatrix}.
  \end{align*}
 したがって,
  \begin{align*}
    \begin{pmatrix}1&\frac{-1}{5}\\0&0\end{pmatrix}
      \begin{pmatrix}x\\y\end{pmatrix}
=
    \begin{pmatrix}0\\0\end{pmatrix}
  \end{align*}
  を解けば良い.
  左辺を計算すると
  \begin{align*}
    \begin{pmatrix}x+\frac{-1}{5}y\\0\end{pmatrix}=
      \begin{pmatrix}0\\0\end{pmatrix}
  \end{align*}
  となるので,
   $x-\frac{1}{5}y=0$であるので,
  $y=t$とすると, $x=\frac{t}{5}$である.
  よって, 解は, 変数$t$を使って
  \begin{align*}
    \xx=\begin{pmatrix}\frac{t}{5}\\t\end{pmatrix} 
  \end{align*}
  と書くことができる.
  $t=0$のときには, $\zzero$になり固有ベクトルではないので,
  固有値$1$に属する固有ベクトルは,
  \begin{align*}
    &\begin{pmatrix}\frac{t}{5}\\t\end{pmatrix}
       &&(t\neq 0)
  \end{align*}
  と書くことができる.

  固有値$-5$に属する固有ベクトルを求める.
  つまり$A\xx=-5\xx$となる$\xx$について考える.
  これは$(A+5E_2)\xx=\zzero$
  を解けばよい.
  \begin{align*}
    A-E_2=\begin{pmatrix}-4+5&1\\5&5\end{pmatrix}
    =\begin{pmatrix}1&1\\5&5\end{pmatrix}
  \end{align*}
  であるので,
  \begin{align*}
    \begin{pmatrix}1&1\\5&5\end{pmatrix}
    \begin{pmatrix}x\\y\end{pmatrix}=
    \begin{pmatrix}0\\0\end{pmatrix}
  \end{align*}
  をとく.
  \begin{align*}
    \begin{pmatrix}1&1\\5&5\end{pmatrix}
  \end{align*}
  の$2$行目に$1$行目の$-5$倍を足すと,
 \begin{align*}
    \begin{pmatrix}1&1\\0&0\end{pmatrix}.
  \end{align*}
 したがって,
  \begin{align*}
    \begin{pmatrix}1&1\\0&0\end{pmatrix}
      \begin{pmatrix}x\\y\end{pmatrix}
=
    \begin{pmatrix}0\\0\end{pmatrix}
  \end{align*}
  を解けば良い.
  左辺を計算すると
  \begin{align*}
    \begin{pmatrix}x+y\\0\end{pmatrix}=
      \begin{pmatrix}0\\0\end{pmatrix}
  \end{align*}
  となるので,
   $x+y=0$であるので,
  $y=t$とすると, $x=-t$である.
  よって, 解は, 変数$t$を使って
  \begin{align*}
    \xx=\begin{pmatrix}-t\\t\end{pmatrix} 
  \end{align*}
  と書くことができる.
  $t=0$のときには, $\zzero$になり固有ベクトルではないので,
  固有値$-5$に属する固有ベクトルは,
  \begin{align*}
    &\begin{pmatrix}-t\\t\end{pmatrix}
       &&(t\neq 0)
  \end{align*}
  と書くことができる.

  
  次に$A$を対角化する.
  \begin{align*}
    &\begin{pmatrix}1\\5\end{pmatrix}
  \end{align*}
  は固有値$1$に属する固有ベクトルである.
  また
  \begin{align*}
    &\begin{pmatrix}1\\-1\end{pmatrix}
  \end{align*}
  は固有値$-5$に属する固有ベクトルである.
  これらを並べて得られる行列を$P$とすると
  \begin{align*}
    P=&\begin{pmatrix}1&1\\5&-1\end{pmatrix}
  \end{align*}
  である.
  $\det(P)=-1-5=-6\neq 0$であるので$P$は正則であり
  \begin{align*}
    P^{-1}=\frac{1}{-6}\begin{pmatrix}-1&-1\\-5&1\end{pmatrix}
  \end{align*}
  となる.
  このとき,
  \begin{align*}
    P^{-1}AP=\begin{pmatrix}1&0\\0&-5\end{pmatrix}
  \end{align*}
  となる.


  最後に$A^n$を求める.
  \begin{align*}
    (P^{-1}AP)^n=(P^{-1}AP)(P^{-1}AP)\cdot (P^{-1}AP)=P^{-1}A^nP
  \end{align*}
  であるが, 一方で,
  \begin{align*}
    (P^{-1}AP)^n=\begin{pmatrix}1&0\\0&-5\end{pmatrix}^n
    =\begin{pmatrix}1^n&0\\0&(-5)^n\end{pmatrix}
    =\begin{pmatrix}1&0\\0&(-5)^n\end{pmatrix}
  \end{align*}
  でもある.
  したがって,
  \begin{align*}
    P^{-1}A^nP&=\begin{pmatrix}1&0\\0&(-5)^n\end{pmatrix}\\
    A^n
    &=P\begin{pmatrix}1&0\\0&(-5)^n\end{pmatrix}P^{-1}\\
    &=\begin{pmatrix}1&1\\5&-1\end{pmatrix}\begin{pmatrix}1&0\\0&(-5)^n\end{pmatrix}(\frac{1}{-6}\begin{pmatrix}-1&-1\\-5&1\end{pmatrix})\\
    &=\frac{1}{-6}\begin{pmatrix}1&1\\5&-1\end{pmatrix}\begin{pmatrix}1&0\\0&(-5)^n\end{pmatrix}\begin{pmatrix}-1&-1\\-5&1\end{pmatrix}\\
    &=\frac{1}{-6}\begin{pmatrix}1&1\\5&-1\end{pmatrix}\begin{pmatrix}-1&-1\\(-5)^{n+1}&(-5)^n\end{pmatrix}\\
    &=\frac{1}{-6}\begin{pmatrix}-1+(-5)^{n+1}&-1+(-5)^n\\-5-(-5)^{n+1}&-5-(-5)^n\end{pmatrix}
  \end{align*}
  となる.
\end{answerof}


\begin{answerof}{quiz:6:5}
  \begin{align*}
    \transposed{A}A&=
    (\frac{1}{\sqrt 5}
    \begin{pmatrix}
      2&1\\
      -1&2
    \end{pmatrix})
    (\frac{1}{\sqrt 5}
    \begin{pmatrix}
      2&-1\\
      1&2
    \end{pmatrix})\\
    &=
    \frac{1}{5}
    \begin{pmatrix}
      2&1\\
      -1&2
    \end{pmatrix}
    \begin{pmatrix}
      2&-1\\
      1&2
    \end{pmatrix}\\
    &=
    \frac{1}{5}
    \begin{pmatrix}
      2\cdot 2+1\cdot 1&2\cdot (-1)+1\cdot 2\\
      -1\cdot 2+2\cdot 1&(-1)(-1)+2\cdot 2
    \end{pmatrix}\\
    &=
    \frac{1}{5}
    \begin{pmatrix}
      5&0\\
      0&5
    \end{pmatrix}\\
    &=
    \begin{pmatrix}
      1&0\\
      0&1
    \end{pmatrix}
  \end{align*}
  
\end{answerof}

\begin{answerof}{quiz:6:6}
  \begin{align*}
    A=
    \begin{pmatrix}
      2&2\\2&-1
    \end{pmatrix}
  \end{align*}
  とすると, $A$は対称行列であるので,
  直交行列により必ず対角化できる.
  基本的には通常と同様に対角化すればよいが,
  直交化に使う正則行列$P$を決める際に注意が必要である.

  まず固有値を求める.
  そのために$\det(A-xE_2)=0$という$x$に関する方程式を解く.
  \begin{align*}
    \det(A-xE_2)
    &=\det(
    \begin{pmatrix}
      2-x&2\\2&-1-x
    \end{pmatrix})\\
    &=(2-x)(-1-x)-2\cdot 2\\
    &=x^2-x-6
    =(x-3)(x+2)
  \end{align*}
  であるので, $(x-3)(x+2)=0$とすると, $x\in\Set{3,-2}$である.
  よって, $A$の固有値は$3$, $-2$である.

  次に, 固有ベクトルを求める.

  固有値$3$に属する固有ベクトルは,
  $\xx$に関する方程式$(A-3E_2)\xx=\zzero$を解けばよい:
  \begin{align*}
    A-3E_2=
    \begin{pmatrix}
      2-3&2\\2&-1-3
    \end{pmatrix}
    =
    \begin{pmatrix}
      -1&2\\2&-4
    \end{pmatrix}
  \end{align*}
  である.  1行目を$-1$倍すると,
  \begin{align*}
    \begin{pmatrix}
      1&-2\\2&-4
    \end{pmatrix}.
  \end{align*}
  2行目に1行目の$-2$倍を加えると,
  \begin{align*}
    \begin{pmatrix}
      1&-2\\0&0
    \end{pmatrix}
  \end{align*}
  となる.
  \begin{align*}
    \begin{pmatrix}
      1&-2\\0&0
    \end{pmatrix}
    \begin{pmatrix}
      x\\y
    \end{pmatrix}=
    \begin{pmatrix}
      0\\0
    \end{pmatrix}
  \end{align*}
  とすると, $x-2y=0$である.
  つまり$x=2y$であるので, 考えている方程式の実数解は,
  \begin{align*}
    \xx=
    \begin{pmatrix}
      2t\\t
    \end{pmatrix}&&(t\in\RR)
  \end{align*}
  である.
  よって固有値$3$に属する固有ベクトルは,
  \begin{align*}
    \begin{pmatrix}
      2t\\t
    \end{pmatrix}&&(t\neq 0)
  \end{align*}
  と書ける.

  
  固有値$-2$に属する固有ベクトルは,
  $\xx$に関する方程式$(A-(-2)E_2)\xx=\zzero$を解けばよい:
  \begin{align*}
    A-(-2)E_2=
    \begin{pmatrix}
      2+2&2\\2&-1+2
    \end{pmatrix}
    =
    \begin{pmatrix}
      4&2\\2&1
    \end{pmatrix}
  \end{align*}
  である.  1行目を$\frac{1}{4}$倍すると,
  \begin{align*}
    \begin{pmatrix}
      1&\frac{1}{2}\\2&1
    \end{pmatrix}.
  \end{align*}
  2行目に1行目の$-2$倍を加えると,
  \begin{align*}
    \begin{pmatrix}
      1&\frac{1}{2}\\0&0
    \end{pmatrix}
  \end{align*}
  となる.
  \begin{align*}
    \begin{pmatrix}
      1&\frac{1}{2}\\0&0
    \end{pmatrix}
    \begin{pmatrix}
      x\\y
    \end{pmatrix}=
    \begin{pmatrix}
      0\\0
    \end{pmatrix}
  \end{align*}
  とすると, $x+\frac{1}{2}y=0$である.
  つまり$x=-\frac{1}{2}y$であるので, 考えている方程式の実数解は,
  \begin{align*}
    \xx=
    \begin{pmatrix}
      -\frac{1}{2}t\\t
    \end{pmatrix}&&(t\in\RR)
  \end{align*}
  である.
  よって固有値$3$に属する固有ベクトルは,
  \begin{align*}
    \begin{pmatrix}
      -\frac{1}{2}t\\t
    \end{pmatrix}&&(t\neq 0)
  \end{align*}
  と書ける.

  次に, 対角化するために必要となる正則行列$P$を考える.
  これは, 固有ベクトルを2つ選んできて並べることで作る.
  今回は$P$が直交行列であるようにしたいので,
  ノルムが$1$である固有ベクトルを選ぶ必要がある.
  固有ベクトルはスカラー倍をしても再び固有ベクトルとなるので,
  固有ベクトルを選んできてそれをノルムの逆数によりスカラー倍すればよい.
  

  \begin{align*}
    \vv=
    \begin{pmatrix}
      2\\1
    \end{pmatrix}
  \end{align*}
  とすると,
  固有値$3$に属する固有ベクトルである.
  $\|\vv\|=\sqrt{2^2+1^2}=\sqrt{5}$であるので,
  \begin{align*}
    \vv'=
    \frac{1}{\sqrt{5}}
    \begin{pmatrix}
      2\\1
    \end{pmatrix}
  \end{align*}
  とすれば, $\vv'$は
  固有値$3$に属する固有ベクトルであり,
  $\|\vv'\|=1$である.

  
  \begin{align*}
    \ww=
    \begin{pmatrix}
      -\frac{1}{2}\\1
    \end{pmatrix}
  \end{align*}
  とすると,
  固有値$-2$に属する固有ベクトルである.
  $\|\ww\|=\sqrt{\frac{1}{2^2}+1^2}=\frac{\sqrt{5}}{2}$であるので,
  \begin{align*}
    \vv'=
    \frac{2}{\sqrt{5}}
    \begin{pmatrix}
      -\frac{1}{2}\\1
    \end{pmatrix}
    =
    \frac{1}{\sqrt{5}}
    \begin{pmatrix}
      -1\\2
    \end{pmatrix}
  \end{align*}
  とすれば, $\ww'$は
  固有値$-2$に属する固有ベクトルであり,
  $\|\ww'\|=1$である.

  
  $\vv'$と$\ww'$を並べた行列を$P$とすると,
  \begin{align*}
    P=\frac{1}{\sqrt{5}}
    \begin{pmatrix}
      2&-1\\
      1&2
    \end{pmatrix}
  \end{align*}
  である.  対称行列の相異なる固有値に属する固有ベクトルは直交するので,
  $\Braket{\vv',\ww'}=0$である.
  また, $\|\vv'\|=\|\ww'\|=1$であったから,
  $P$は直交行列である.
  $P^{-1}=\transposed{P}$
  であるから,
  \begin{align*}
    P^{-1}=\frac{1}{\sqrt{5}}
    \begin{pmatrix}
      2&1\\
      -1&2
    \end{pmatrix}
  \end{align*}
  である.
  この直交行列$P$に対し,
  \begin{align*}
    P^{-1}AP=
    (\frac{1}{\sqrt{5}}
    \begin{pmatrix}
      2&1\\
      -1&2
    \end{pmatrix})
    \begin{pmatrix}
      2&2\\2&-1
    \end{pmatrix}
    (\frac{1}{\sqrt{5}}
    \begin{pmatrix}
      2&-1\\
      1&2
    \end{pmatrix})
    =
    \begin{pmatrix}
      3&0\\0&-2
    \end{pmatrix}
  \end{align*}
  となる.
  
\end{answerof}






\begin{answerof}{quiz:6:7}
$a_{0}=0$として$0$項目を付け加えても, 漸化式をみたすので,
問題ないから, $a_0$を付け加えて考える.

\begin{align*}
  \vv_n=\begin{pmatrix}a_{n+1}\\a_n\end{pmatrix}
\end{align*}
とおく.
このとき,
\begin{align*}
  \vv_0=\begin{pmatrix}a_{1}\\a_0\end{pmatrix}=\begin{pmatrix}1\\0\end{pmatrix}
\end{align*}
である.
また,
\begin{align*}
  \vv_{n+1}
  &=\begin{pmatrix}a_{n+2}\\a_{n+1}\end{pmatrix}\\
  &=\begin{pmatrix}a_{n+1}+a_n\\a_{n+1}\end{pmatrix}\\
  &=\begin{pmatrix}1&1\\1&0\end{pmatrix}\begin{pmatrix}a_{n+1}\\a_n\end{pmatrix}\\
  &=\begin{pmatrix}1&1\\1&0\end{pmatrix}\vv_n
\end{align*}
とできる.
\begin{align*}
A=\begin{pmatrix}1&1\\1&0\end{pmatrix}
\end{align*}
とおくと,
\begin{align*}
  \vv_{n+1}
  &=A\vv_{n}
\end{align*}
となる.
この等式は$n>0$であればどの$n$でも成り立つので,
\begin{align*}
  \vv_{n}
  &=A\vv_{n-1}.
\end{align*}
と書き直しても良い.
繰り返しこの等式を使うことで,
\begin{align*}
  \vv_{n}&=A\vv_{n-1}=A^2\vv_{n-2}
  =\cdots=A^{n}\vv_0
\end{align*}
となることがわかる.

$\vv_0$はもうすでにわかっているので$A^n$を計算する.

まず$A$の固有値を求める.
\begin{align*}
  A-xE_2=\begin{pmatrix}1&1\\1&0\end{pmatrix}-\begin{pmatrix}x&0\\0&x\end{pmatrix}
=\begin{pmatrix}1-x&1\\1&-x\end{pmatrix}
\end{align*}
であるので,
\begin{align*}
  \det(A-xE_2)&=\det(\begin{pmatrix}1-x&1\\1&-x\end{pmatrix})\\
    &=-(1-x)x-1\\
    &=x^2-x-1\\
    &=\left(x-\frac{1}{2}\right)^2-\frac{1}{4}-1\\
    &=\left(x-\frac{1}{2}\right)^2-\frac{5}{4}\\\
    &=\left(x-\frac{1}{2}\right)^2-\left(\frac{\sqrt{5}}{2}\right)^2\\
    &=\left(x-\frac{1}{2}+\frac{\sqrt{5}}{2}\right)\left(x-\frac{1}{2}-\frac{\sqrt{5}}{2}\right)\\
    &=\left(x-\frac{1-\sqrt{5}}{2}\right)\left(x-\frac{1+\sqrt{5}}{2}\right)
\end{align*}
である.
したがって, $x$に関する方程式
\begin{align*}
  \det(A-xE_2)&=0
\end{align*}
は,
\begin{align*}
\lambda&=\frac{1-\sqrt{5}}{2}&
\mu&=\frac{1+\sqrt{5}}{2}
\end{align*}
とおけば,
\begin{align*}
  \left(x-\lambda\right)\left(x-\mu\right)&=0
\end{align*}
と書けるので,
解は, $\lambda$と$\mu$
の$2$つである.
したがって,
$A$の固有値も,
$\lambda$と
$\mu$
の$2$つである.


なお,
$\lambda$と$\mu$は,
$\lambda+\mu=1$,
$\lambda\mu=-1$
を満たしているので,
\begin{align*}
  1-\lambda&=\mu\\
  1-\mu&=\lambda\\
\frac{1}{\lambda}&=-\mu\\
\frac{1}{\mu}&=-\lambda
\end{align*}
である.
また,
$\lambda-\mu=\frac{1-\sqrt{5}}{2}-\frac{1-\sqrt{5}}{2}=-\sqrt{5}$である.



まず,
固有値$\lambda$に属する固有ベクトルを求める.
\begin{align*}
  A-\lambda E_2 \xx=\zzero 
\end{align*}
の$\zzero$以外の解を求めれば,
それが$\lambda$に属する固有ベクトルである.
\begin{align*}
  A-\lambda E_2
  & =
  \begin{pmatrix}1-\lambda&1\\1&-\lambda\end{pmatrix}  \\
  & =
    \begin{pmatrix}\mu&1\\1&-\lambda\end{pmatrix}
\end{align*}
である. したがって, $\xx$に関する方程式
\begin{align*}
  A-\lambda E_2 \xx=\zzero 
\end{align*}
の拡大係数行列は,
\begin{align*}
  \begin{pmatrix}
   \mu&1&0\\
   1&-\lambda&0
  \end{pmatrix}
\end{align*}
である. $1$行目を$\frac{1}{\mu}$倍すると,
\begin{align*}
  \begin{pmatrix}
    1&\frac{1}{\mu}&0\\
    1&-\lambda&0
  \end{pmatrix}
 =
  \begin{pmatrix}
    1&-\lambda&0\\
    1&-\lambda&0
  \end{pmatrix}
\end{align*}
となる.
$2$行目に$1$行目の$-1$倍を足すと,
\begin{align*}
  \begin{pmatrix}
    1&-\lambda&0\\
    0&0&0
  \end{pmatrix}
\end{align*}
となる.
したがって,
\begin{align*}
  (A-\lambda E_2) \xx=\zzero 
\end{align*}
の解の空間は,
\begin{align*}
\Set{t\begin{pmatrix}\lambda\\1\end{pmatrix}|t\text{は数}}
\end{align*}
と書ける.
したがって, 例えば
\begin{align*}
  \begin{pmatrix}\lambda\\1\end{pmatrix}
\end{align*}
は固有値$\lambda$に属する固有ベクトルである.


次に,
固有値$\mu$に属する固有ベクトルを求める.
\begin{align*}
  A-\mu E_2 \xx=\zzero 
\end{align*}
の$\zzero$以外の解を求めれば,
それが$\mu$に属する固有ベクトルである.
\begin{align*}
  A-\mu E_2
  & =
  \begin{pmatrix}1-\mu&1\\1&-\mu\end{pmatrix}  \\
  & =
    \begin{pmatrix}\lambda&1\\1&-\mu\end{pmatrix}
\end{align*}
である. したがって, $\xx$に関する方程式
\begin{align*}
  (A-\lambda E_2) \xx=\zzero 
\end{align*}
の拡大係数行列は,
\begin{align*}
  \begin{pmatrix}
    \lambda&1&0\\
    1&-\mu&0
  \end{pmatrix}
\end{align*}
である. $1$行目を$\frac{1}{\lambda}$倍すると,
\begin{align*}
  \begin{pmatrix}
    1&\frac{1}{\lambda}&0\\
    1&-\mu&0
  \end{pmatrix}
  =
  \begin{pmatrix}
    1&-\mu&0\\
    1&-\mu&0
  \end{pmatrix}
\end{align*}
となる.
$2$行目に$1$行目の$-1$倍を足すと,
\begin{align*}
  \begin{pmatrix}
    1&-\mu&0\\
    0&0&0
  \end{pmatrix}
\end{align*}
となる.
したがって,
\begin{align*}
  A-\mu E_2 \xx=\zzero 
\end{align*}
の解の空間は,
\begin{align*}
\Set{t\begin{pmatrix}\mu\\1\end{pmatrix}|t\text{は数}}
\end{align*}
と書ける.
例えば
\begin{align*}
  \begin{pmatrix}\mu\\1\end{pmatrix}
\end{align*}
は固有値$\mu$に属する固有ベクトルである.


固有ベクトルが求められたので対角化をする.
\begin{align*}
  P=
  \begin{pmatrix}
    \lambda&\mu\\
    1&1    
  \end{pmatrix}
\end{align*}
とすると
\begin{align*}
  \det(P)=
  \det(
  \begin{pmatrix}
    \lambda&\mu\\
    1&1    
  \end{pmatrix})
  =\lambda-\mu=-\sqrt{5}\neq 0
\end{align*}
であるので$P$は正則であり,
\begin{align*}
  P^{-1}=
  \frac{1}{-\sqrt{5}}
  \begin{pmatrix}
    1&-\mu\\
    -1&\lambda    
  \end{pmatrix}
\end{align*}
である.
%%%%%%%%%%%%%%%%%%%%%%%%%%%%%%%%%%%%%
%%%%%%@@@@@@@@@@@@@@@@@@@@@@@@@@

このとき,
\begin{align*}
  P^{-1}AP=
  \begin{pmatrix}
    \lambda&0\\
    0&\mu
  \end{pmatrix}
\end{align*}
である.
したがって,
\begin{align*}
  (P^{-1}AP)^n=
  \begin{pmatrix}
    \lambda^n&0\\
    0&\mu^n
  \end{pmatrix}
\end{align*}
である.
一方, \cref{ex:diagonalizedpow}でみたように,
\begin{align*}
  (P^{-1}AP)^n
  &=P^{-1}A^nP
\end{align*}
であるので,
\begin{align*}
P^{-1}A^nP&=
  \begin{pmatrix}
    \lambda^n&0\\
    0&\mu^n
  \end{pmatrix}
\\
A^n&=
P
  \begin{pmatrix}
    \lambda^n&0\\
    0&\mu^n
  \end{pmatrix}
  P^{-1}
\end{align*}
とできる.
したがって,
\begin{align*}
  A^n
  &=
P
  \begin{pmatrix}
    \lambda^n&0\\
    0&\mu^n
  \end{pmatrix}
  P^{-1}\\
&=
  \begin{pmatrix}
    \lambda&\mu\\
    1&1    
  \end{pmatrix}
  \begin{pmatrix}
    \lambda^n&0\\
    0&\mu^n
  \end{pmatrix}
(  \frac{-1}{\sqrt{5}}
  \begin{pmatrix}
    1&-\mu\\
    -1&\lambda    
  \end{pmatrix}
)\\
&=\frac{-1}{\sqrt{5}}
  \begin{pmatrix}
    \lambda&\mu\\
    1&1    
  \end{pmatrix}
  \begin{pmatrix}
    \lambda^n&0\\
    0&\mu^n
  \end{pmatrix}
  \begin{pmatrix}
    1&-\mu\\
    -1&\lambda    
  \end{pmatrix}
\\
&=\frac{-1}{\sqrt{5}}
  \begin{pmatrix}
    \lambda&\mu\\
    1&1    
  \end{pmatrix}
  \begin{pmatrix}
    \lambda^n&-\lambda^n\mu\\
    -\mu^n&\lambda\mu^n    
  \end{pmatrix}
\\
&=\frac{-1}{\sqrt{5}}
  \begin{pmatrix}
    \lambda\lambda^n-\mu^n\mu&-\lambda^n\mu\lambda+\lambda\mu^n\mu\\
    \lambda^n-\mu^n&-\lambda^n\mu+\lambda\mu^n        
  \end{pmatrix}\\
&=\frac{-1}{\sqrt{5}}
  \begin{pmatrix}
    \lambda^{n+1}-\mu^{n+1}&\lambda^n-\mu^n\\
    \lambda^n-\mu^n&\lambda^{n-1}-\mu^{n-1}        
  \end{pmatrix}
\end{align*}
となる.

\begin{align*}
  \begin{pmatrix}a_{n+1}\\a_n\end{pmatrix}
    &=
    \vv_n\\
    &=
    A^n\vv_0\\
    &=
    \frac{-1}{\sqrt{5}}
    \begin{pmatrix}
      \lambda^{n+1}-\mu^{n+1}&\lambda^n-\mu^n\\
      \lambda^n-\mu^n&\lambda^{n-1}-\mu^{n-1}        
    \end{pmatrix}
    \begin{pmatrix}1\\0\end{pmatrix}\\
    &=
    \frac{-1}{\sqrt{5}}
    \begin{pmatrix}
      \lambda^{n+1}-\mu^{n+1}\\
      \lambda^n-\mu^n
    \end{pmatrix}
\end{align*}
である.
したがって,
\begin{align*}
  a_n&=
    \frac{-1}{\sqrt{5}}
    (\lambda^n-\mu^n)\\
    &=
    \frac{\mu^n-\lambda^n}{\sqrt{5}}
    \\
    &=\frac{1}{\sqrt{5}}\left(\left(\frac{1+\sqrt{5}}{2}\right)^n
    -\frac{1}{\sqrt{5}}\left(\frac{1-\sqrt{5}}{2}\right)^n\right)
\end{align*}
である.
\end{answerof}

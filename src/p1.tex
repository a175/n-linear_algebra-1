\title{線形代数学I (沼田担当) 講義補足資料}
\author{}
\date{}
\maketitle

\begin{abstract}
  この資料の目的は, 次の2つである
  被役階段行列を係数行列とする連立一次方程式の解法について
  補足説明をすること;
  講義中に証明を省略した命題(\cref{thm:main})について,
証明を与えること.
\end{abstract}

\section{被役階段行列を係数行列とする連立一次方程式の解法について}
$A$を$(m,n)$行列とし, $\bbb$を$(m,1)$行列とする.
このとき, $\xxx$に関する方程式
$A\xxx=\bbb$
を考えることができる.
$A$が被約階段行列である場合に,
解の集合を求める方法について,
具体例で説明する.

ここでは次の例を考える:
\begin{align*}
  A&=
  \begin{pmatrix}
    0&1&0&0&5&0&3\\
    0&0&0&1&0&0&9\\
    0&0&0&0&0&1&0
  \end{pmatrix}
  &
  \bbb=
  \begin{pmatrix}
    4\\
    6\\
    8
  \end{pmatrix}
\end{align*}
という場合を考える.
$A$は$(3,7)$行列であり,
$\bbb$は$(3,1)$行列である.
また, $A$は被約階段行列である.

$A\xxx=\bbb$という方程式を考える.
このとき未知数$x_i$の個数は$A$の列数である$7$でないといけない.
未知数の個数が$7$個であれば, $\xxx$は$(7,1)$行列であり,
$A\xxx$という積が意味をもつ.
$A\xxx=\bbb$を具体的に書き下すと,
\begin{align*}
  \begin{pmatrix}
    0&1&0&0&5&0&3\\
    0&0&0&1&0&0&9\\
    0&0&0&0&0&1&0
  \end{pmatrix}
  \begin{pmatrix}
    x_1\\x_2\\x_3\\x_4\\x_5\\x_6\\x_7
  \end{pmatrix}
  &
  =
  \begin{pmatrix}
    4\\
    6\\
    8
  \end{pmatrix}
\end{align*}
となる.
$A$と$\bbb$の行数は揃っているが,
$\xxx$はそうではないことを念頭に入れておいてほしい.
さて, この右辺を具体的に計算すると以下のようになる.
\begin{align*}
  \begin{pmatrix}
    0x_1+1x_2+0x_3+0x_4+5x_5+0x_6+3x_7\\
    0x_1+0x_2+0x_3+1x_4+0x_5+0x_6+9x_7\\
    0x_1+0x_2+0x_3+0x_4+0x_5+1x_6+0x_7
  \end{pmatrix}
  =
  \begin{pmatrix}
    4\\
    6\\
    8
  \end{pmatrix}
\end{align*}
左辺を整理すると, 以下のようになる.
\begin{align*}
  \begin{pmatrix}
    x_2+5x_5+3x_7\\
    x_4+9x_7\\
    x_6
  \end{pmatrix}
  =
  \begin{pmatrix}
    4\\
    6\\
    8
  \end{pmatrix}
\end{align*}
行ごとの成分を比較することで次の連立方程式となる:
\begin{align*}
  \begin{cases}
    x_2+5x_5+3x_7=4\\
    x_4+9x_7=6\\
    x_6=8
  \end{cases}
\end{align*}
$A$が被約階段行列なので,
各方程式に現れる未知数のうち番号の一番若いものに着目すると,
その係数は1であり,
また, 他の方程式には現れないという, とても良い状況になっている.

そこで,
各方程式に現れる未知数のうち番号の一番若いものを左辺に残し,
それ以外を右辺に移項する.
\begin{align*}
  \begin{cases}
    x_2=4-(5x_5+3x_7)\\
    x_4=6-9x_7\\
    x_6=8
  \end{cases}
\end{align*}
いま,
各方程式に現れる未知数のうち番号の一番若いものを左辺に残し,
それ以外を右辺に移項した.\footnote{重要なので2回書いた.}
この変形により左辺に現れる$x_2$, $x_4$, $x_6$は,
右辺によって値が定まる.
一方これ以外の未知数の値はどのような値でもよい.
左辺に現れる$x_2$, $x_4$, $x_6$以外の未知数というのは,
以下の4つである.
\begin{align*}
  x_1,x_3,x_5,x_7
\end{align*}
である.
$x_1$の係数は$0$であったためどの方程式にも現れないが,
もともとの$A\xxx=\bbb$という方程式にはあった未知数であるので,
これも忘れてはいけないことに注意する.

$x_7=c_1$,
$x_5=c_2$,
$x_3=c_3$, 
$x_1=c_4$, 
であるとき,
$x_6=8$,
$x_4=6-9c_1$,
$x_2=3-(5c_2+3c_1)$
となる.
したがって$A\xxx=\bbb$の実数解の集合は,
\begin{align*}
  \Set{\begin{pmatrix}
      c_4\\
      3-(5c_2+3c_1)\\
      c_3\\
      6-9c_1\\
      c_2\\
      8\\
      c_1
    \end{pmatrix}|c_1,c_2,c_3,c_4\in\RR}
\end{align*}
となる.  これで解の集合は求められたことになる.

加法とスカラー倍の定義から,
\begin{align*}
  \begin{pmatrix}
      c_4\\
      3-(5c_2+3c_1)\\
      c_3\\
      6-9c_1\\
      c_2\\
      8\\
      c_1
  \end{pmatrix}
  =
  \begin{pmatrix}
      0\\
      3\\
      0\\
      6\\
      0\\
      8\\
      0
  \end{pmatrix}
  +c_1
  \begin{pmatrix}
      0\\
      -3\\
      0\\
      -9\\
      0\\
      0\\
      1
  \end{pmatrix}
  +c_2
  \begin{pmatrix}
      0\\
      -5\\
      0\\
      0\\
      1\\
      0\\
      0
  \end{pmatrix}
  +c_3
  \begin{pmatrix}
      0\\
      0\\
      1\\
      0\\
      0\\
      0\\
      0
  \end{pmatrix}
  +c_4
  \begin{pmatrix}
      1\\
      0\\
      0\\
      0\\
      0\\
      0\\
      0
  \end{pmatrix}
\end{align*}
であるので,
$A\xxx=\bbb$の実数解の集合は,
\begin{align*}
  \Set{  \begin{pmatrix}
      0\\
      3\\
      0\\
      6\\
      0\\
      8\\
      0
  \end{pmatrix}
  +c_1
  \begin{pmatrix}
      0\\
      -3\\
      0\\
      -9\\
      0\\
      0\\
      1
  \end{pmatrix}
  +c_2
  \begin{pmatrix}
      0\\
      -5\\
      0\\
      0\\
      1\\
      0\\
      0
  \end{pmatrix}
  +c_3
  \begin{pmatrix}
      0\\
      0\\
      1\\
      0\\
      0\\
      0\\
      0
  \end{pmatrix}
  +c_4
  \begin{pmatrix}
      1\\
      0\\
      0\\
      0\\
      0\\
      0\\
      0
  \end{pmatrix}
|c_1,c_2,c_3,c_4\in\RR}
\end{align*}
とも書ける.



\section{被約階段行列について}
講義で紹介したが,
証明は与えていない次の命題について,
証明を与える
\begin{theorem}
  \label{thm:main}
  $A$, $B$を被約階段行列としサイズは$(m,n)$とする.
  また, $P$を$m$次正則行列とする.
  $PA=B$ならば$A=B$
\end{theorem}
この命題を証明するために, いくつか補題を用意する.






$P$を$m$次正方行列とする.
$\xxx$を$(m,1)$行列とする.
$\zzero$を零行列$O_{m,1}$とする.
このとき次が成り立つことは既に見た:
\begin{lemma}
  \label{lem:1}
  $\xxx\neq\zzero$かつ$P\xxx=\zzero$ならば,
  $P$は正則ではない.
\end{lemma}



$i=1,2,\ldots,m$に対し,
  \begin{align*}
    \eee_i=\begin{pmatrix}0\\\vdots\\0\\1\\0\\\vdots\\0\end{pmatrix}
  \end{align*}
という$i$行目だけ$1$で, 他は$0$である$(m,1)$行列とする.
\begin{lemma}
  \label{lem:2}
  $k\geq 1$とする.
  \begin{align*}
    P\eee_1&=\eee_1,\\
    P\eee_2&=\eee_2,\\
    &\vdots\\
    P\eee_{k-1}&=\eee_{k-1},\\
    P\eee_k&=\alpha_1\eee_1+\alpha_2\eee_2+\cdots+\alpha_{k-1}\eee_{k-1}\\
    &(\text{$k=1$のときには, $P\eee_1=P\eee_k=0$ という条件として読む})    
  \end{align*}
  ならば, $P$は正則ではない.
\end{lemma}
\begin{proof}
$\eee_k-(\alpha_1\eee_1+\alpha_2\eee_2+\cdots+\alpha_{k-1}\eee_{k-1})$
  は第$k$成分が1なので, $\zzero$ではない.
  \begin{align*}
    &\phantom{{}={}}P(\eee_k-(\alpha_1\eee_1+\alpha_2\eee_2+\cdots+\alpha_{k-1}\eee_{k-1}))\\
    &=
    P\eee_k-(\alpha_1P\eee_1+\alpha_2P\eee_2+\cdots+\alpha_{k-1}P\eee_{k-1}))\\
    &=
    P\eee_k-(\alpha_1\eee_1+\alpha_2\eee_2+\cdots+\alpha_{k-1}\eee_{k-1}))\\
    &=
    P\eee_k-P\eee_k\\
    &=\zzero.
  \end{align*}
  したがって\cref{lem:1}より$P$は正則ではない.
\end{proof}


$A=(a_{i,j})_{\substack{i=1,\ldots,m\\j=1,\ldots,n}}$
$(m,n)$行列とする.
また,
$A=(\aaa_1|\aaa_2|\cdots|\aaa_n)$
と列ベクトル表示されているとする.
つまり,
\begin{align*}
  \aaa_j=\begin{pmatrix}a_{j,1}\\a_{j,2}\\\vdots\\a_{j,m}\end{pmatrix}
\end{align*}
である. さらに, $A$は階数$r$の被約階段行列で,
そのpivotsは$(1,p_1),(2,p_r),\ldots,(r,p_r)$であるとする.

$A$か階数$r$の階段行列であることから次がわかる:
\begin{lemma}
  \label{lem:3}
  \begin{align*}
    \aaa_{j}&=a_{1,j}\eee_1+a_{2,j}\eee_2+\cdots+a_{r,j}\eee_r\\
    &=\sum_{i=1}^r a_{i,j}\eee_i.
  \end{align*}
\end{lemma}
また, $(l+1,p_{l+1})$がpivotであることから, 次がわかる:
\begin{lemma}
  \label{lem:4}
  $j<p_{l+1}$ならば,
  \begin{align*}
    \aaa_{j}&=a_{1,j}\eee_1+a_{2,j}\eee_2+\cdots+a_{l,j}\eee_l\\
    &=\sum_{i=1}^l a_{i,j}\eee_i.
  \end{align*}
\end{lemma}


\begin{lemma}
  \label{lem:5}
    \begin{align*}
    P\eee_1&=\eee_1,\\
    P\eee_2&=\eee_2,\\
    &\vdots\\
    P\eee_{r}&=\eee_{r}
  \end{align*}
  ならば, $PA=A$.
\end{lemma}
\begin{proof}
  \Cref{lem:3}より
  \begin{align*}
    \aaa_{j}&=a_{1,j}\eee_1+a_{2,j}\eee_2+\cdots+a_{r,j}\eee_r.
  \end{align*}
  であるので,
  \begin{align*}
    P\aaa_{j}
    &=a_{1,j}P\eee_1+a_{2,j}P\eee_2+\cdots+a_{r,j}P\eee_r\\
    &=a_{1,j}\eee_1+a_{2,j}\eee_2+\cdots+a_{r,j}\eee_r\\
    &=\aaa_j
  \end{align*}
  したがって
  \begin{align*}
    PA&=P(\aaa_1|\aaa_2|\cdots|\aaa_n)\\
    &=(P\aaa_1|P\aaa_2|\cdots|P\aaa_n)\\
    &=(\aaa_1|\aaa_2|\cdots|\aaa_n)\\
    &=A.
  \end{align*}
\end{proof}

 被約階段行列の定義から次がわかる:
\begin{lemma}
  \label{lem:6}
  $A$が被約階段行列で,
  $(i,p_i)$がpivotならば,
  $\aaa_{p_i}=\eee_i$.
\end{lemma}

$X$, $Y$を$(m,n)$行列とし,
$X=(\xxx_1|\xxx_2|\cdots|\xxx_n)$,
$Y=(\yyy_1|\yyy_2|\cdots|\yyy_n)$
と列ベクトル表示されているとする.
さらに$X$も$Y$も
被約階段行列であることを仮定する.

\begin{lemma}
  \label{lem:7}
  $X$と$Y$のpivotが完全に一致するとする.
  このとき$PX=Y$ならば$X=Y$.
\end{lemma}
\begin{proof}
  $X$と$Y$のpivotsが
  $(1,p_1),(2,p_r),\ldots,(r,p_r)$であるとする.
  \begin{align*}
    PX&=P(\xxx_1|\cdots|\xxx_n)=(P\xxx_1|\cdots|P\xxx_n)\\
    Y&=(\yyy_1|\cdots|\yyy_n)
  \end{align*}
  であり, $PX=Y$であるので,
  \begin{align*}
    (P\xxx_1|\cdots|P\xxx_n)=(\yyy_1|\cdots|\yyy_n).
  \end{align*}
  Pivotである$p_i$列目について考えると,
  \Cref{lem:6}より
  \begin{align*}
    P\xxx_i&=P\eee_i,\\
    \yyy_i&=\eee_i
  \end{align*}
  であるので$P\eee_i=\eee_i$である.
  $i\in\Set{1,\ldots,r}$に対し$P\eee_i=\eee_i$となるので,
  \Cref{lem:5}より$X=PX=Y$.  
\end{proof}

\begin{lemma}
  \label{lem:8}
  $X$のpivotsは$(1,p_1),\ldots,(r,p_r)$,
  $Y$のpivotsは$(1,p_1),\ldots,(l,q_l)$であるとする.
  $k\leq r$とする.
  次を仮定する.
  \begin{align*}
    i<k \implies p_i=q_i
  \end{align*}
  $PX=Y$をみたすとする.
  $P$が正則行列ならば
  $k\leq l$かつ$p_k\geq q_k$. 
\end{lemma}
\begin{proof}
  対偶となる,
  $k> l$または$p_k< q_k$ならば$P$は正則ではない
  ということを示す.
  
  \begin{align*}
    &PX=(P\xxx_1|\cdots|P\xxx_n)\\
    ={}&Y=(\yyy_1|\cdots|\yyy_n)
  \end{align*}
  であるので, $\yyy_j=P\xxx_j$.
  $j=p_t$のときには, \cref{lem:6}より,
  \begin{align}
    \yyy_{p_t}=P\xxx_{p_t}=P\eee_t.
    \label{eq:x}
  \end{align}

  まず$t\in\Set{1,\ldots,k-1}$のときについて考えておく.
  $p_t=q_t$であるので,
  \cref{lem:6}より,
  \begin{align}
    \yyy_{p_t}=\yyy_{q_t}=\eee_t.
    \label{eq:xx}
  \end{align}
  よって, \eqref{eq:x}と\eqref{eq:xx}より,
  \begin{align*}
    P\eee_t=\xxx_{q_t}=\eee_t.
  \end{align*}


  つぎに, $t=k$のときについて考える.
  $k>l$ならば, \cref{lem:3}を使うと,
  \begin{align*}
    \yyy_{p_k}=\yyy_{1p_k}\eee_1+\cdots+y_{k-1,p_k}\eee_{k-1}
  \end{align*}
  と書ける.
  また, $p_k<q_k$ならば, \cref{lem:4}を使うと,
  \begin{align*}
    \yyy_{p_k}=\yyy_{1p_k}\eee_1+\cdots+y_{k-1,p_k}\eee_{k-1}
  \end{align*}
  と書ける.
  したがって,
  次が成り立つ:
  \begin{align*}
    k>l \text{ or } p_k<q_k
    \implies
    \yyy_{p_k}=\yyy_{1p_k}\eee_1+\cdots+y_{k-1,p_k}\eee_{k-1}.
  \end{align*}
  $k>l$または$p_k<q_k$が成り立つとする.
  このとき, \cref{eq:x}より,
  \begin{align*}
    P\eee_k
    =\yyy_{p_k}
    =\yyy_{1p_k}\eee_1+\cdots+y_{k-1,p_k}\eee_{k-1}
  \end{align*}
  となるが,
  \cref{lem:2}より$P$は正則ではない.
\end{proof}


ここで一旦記号に対する前提を白紙に戻し,
本来示したい主張について述べる.
\begin{theorem}[\Cref{thm:main}]
  $A$, $B$を被約階段行列としサイズは$(m,n)$とする.
  また, $P$を$m$次正則行列とする.
  $PA=B$ならば$A=B$
\end{theorem}
\begin{proof}
  $PA=B$であるので, $B=P^{-1}A$でもある.
  
  $A$にpivotsがないとき:
  このときは$A=O_{m,n}$である.
  $B=PA=PO_{m,n}=O_{m,n}$
  であるので, $A=B$.

  $B$にpivotsがないとき:
  このときは$B=O_{m,n}$である.
  $A=P^{-1}B=PO_{m,n}=O_{m,n}$
  であるので, $A=B$.

  $A$にも$B$にもpivotsがあるとき:
  $A$のpivotsは$(1,p_1),\ldots,(r,p_r)$,
  $B$のpivotsは$(1,p_1),\ldots,(l,q_l)$であるとし,
  まず, pivotsが一致することを示す.
  
  $k=1$として,
  \cref{lem:8}を$PA=B$に使うと$p_1\geq q_1$.
  \cref{lem:8}を$P^{-1}B=A$に使うと$q_1\geq p_1$.
  よって$p_1=q_1$.

  $k=2$として,
  \cref{lem:8}を$PA=B$に使うと$p_2\geq q_2$.
  \cref{lem:8}を$P^{-1}B=A$に使うと$q_2\geq p_2$.
  よって$p_2=q_2$.

  以下同様に繰り返すと,
  \begin{align*}
    r&=l\\
    p_1&=q_1\\
    p_2&=q_2\\
    &\ \vdots\\
    p_r&=q_r
  \end{align*}
  がわかる.
  Pivotsが一致しているので,
  \cref{lem:7}を使うことで$A=B$がわかる.
\end{proof}
